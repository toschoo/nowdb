\section{Outline}
The python client consists of a bunch
of modules all starting with prefix \term{now}.
The main module which defines
the \term{Connection} and the \term{Result} class
is actually called just \tech{now.py}.
Support functions (which are also available)
for server-side python are defined in
\tech{nowutil.py}.

The modules use the package \term{dateutil}
which must be installed on the system in order
to use the \term{now} modules.
(Please refer to chapter
\ref{chpt_install} for details.)

A client program needs to import at least
the \term{now} module. The import may have
any form (\ie\ \term{import now} or
\term{from now import}).

\section{Connections}
The \term{Connection} class represents
a \acronym{tcp/ip} connection to the
database. It provides the following 
methods:
\subsubsection{\_\_init\_\_}
The constructor takes four arguments,
which are all strings:
\begin{itemize}
\item Address:
used to determine the address of the server.
The parameter is passed to the system service
\tech{getaddrinfo} and allows:
an \acronym{ip}v4 address,
an \acronym{ip}v6 address or
a hostname.
Examples: ``127.0.0.1'', ``localhost'',
``myserver.mydomain.org''.

\item Port:
used to determine the port of the server.
The parameter is passed to the system service
\tech{getaddrinfo} and allows:
a port number or a known service name.

\item User:
\comment{currently not used}
\item Password:
\comment{currently not used}
\end{itemize}

\subsubsection{execute(statement)}
The method sends the string \term{statement} to the database
and waits for the database response.
The method retuns an instance of the \term{Result} class
or, on internal errors, raises one of the exceptions
\term{ClientError} or \term{DBError}.

\subsubsection{close()}
The method closes the connection and
releases the C objects allocated by
the constructor. If the connection
cannot be closed for any reason,
the exception \term{ClientError} is raised.

\subsubsection{Resource Manager}
\term{Connection} is a \term{resource manager}.
The method \term{close} is therefore
rarely necessary. Instead, \term{Connection} can
be used with the \term{with} statement, \ie:

\begin{python}
\begin{lstlisting}
with Connection("localhost", "55505", None, None) as conn:
    # here goes your code
    # refer to the connect as 'conn'
\end{lstlisting}
\end{python}

\section{Results}
\subsection{Status}
\subsection{Cursors}
\subsection{Rows}
\subsection{Reports}
\section{Support Functions}
\section{Exceptions}
\subsubsection{ClientError}
\subsubsection{DBError}
\subsubsection{WrongType}
