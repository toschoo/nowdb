\section{Outline}
The python client consists of a bunch
of modules all starting with prefix \term{now}.
The main module which defines
the \term{Connection} and the \term{Result} class
is actually called just \tech{now.py}.
Support functions (which are also available)
for server-side python are defined in
\tech{nowutil.py}.

The modules use the package \term{dateutil}
which must be installed on the system in order
to use the \term{now} modules.
(Please refer to chapter
\ref{chpt_install} for details.)

A client program needs to import at least
the \term{now} module. The import may have
any form (\ie\ \term{import now} or
\term{from now import}).

\section{Connections}
The \term{Connection} class represents
a \acronym{tcp/ip} connection to the
database. It provides the following 
methods:
\subsubsection{\_\_init\_\_}
The constructor takes four arguments,
which are all strings:
\begin{itemize}
\item Address:
used to determine the address of the server.
The parameter is passed to the system service
\tech{getaddrinfo} and allows:
an \acronym{ip}v4 address,
an \acronym{ip}v6 address or
a hostname.
Examples: ``127.0.0.1'', ``localhost'',
``myserver.mydomain.org''.

\item Port:
used to determine the port of the server.
The parameter is passed to the system service
\tech{getaddrinfo} and allows:
a port number or a known service name.

\item User:
\comment{currently not used}
\item Password:
\comment{currently not used}
\end{itemize}

\subsubsection{execute(statement)}
The method sends the string \term{statement} to the database
and waits for the database response.
The method retuns an instance of the \term{Result} class
or, on internal errors, raises one of the exceptions
\term{ClientError} or \term{DBError}.

\subsubsection{close()}
The method closes the connection and
releases the C objects allocated by
the constructor. If the connection
cannot be closed for any reason,
the exception \term{ClientError} is raised.

\subsubsection{Resource Manager}
\term{Connection} is a \term{resource manager}.
The method \term{close} is therefore
rarely necessary. Instead, \term{Connection} can
be used with the \term{with} statement, \ie:

\begin{python}
\begin{lstlisting}
with Connection("localhost", "55505", None, None) as conn:
    # here goes your code
    # refer to the connect as 'conn'
\end{lstlisting}
\end{python}

\section{Results}
The \term{Result} class represents
the dynamic types described in chapter \ref{chpt_sql}.
Instances of \term{Result} are
created and returned by
the \term{Connection.execute()} method.

The \term{Result} class has the following methods:

\subsubsection{rType()}
The method returns the result type,
either STATUS, REPORT, ROW or CURSOR.

\subsubsection{ok()}
The method return \term{True}
if the instance does not represent
an error and \term{False} otherwise.

\subsubsection{release()}
The method releases the C objects
allocated with the \term{Result} object.
Since \term{Result} is a resource manager,
it is rarely necessary to use explicitly.

\subsubsection{Resource Manager}
Result is a resource manager.
\term{Result} can therefore
be used with the \term{with} statement, \eg:

\begin{python}
\begin{lstlisting}
with conn.execute("select count(*) from sales") as cur:
    # here goes your code
    # 'conn' is a previously created connection
    # 'cur' is the result and
    # if no error has occurred, cur is a cursor
\end{lstlisting}
\end{python}

\subsection{Status}
If \term{Result} is a \term{Status},
two more methods are available:

\subsubsection{code()}
The method returns the \nowdb\ error code.
The error code may be
\begin{itemize}
\item 0: Success, no error has occurred;
\item positive:
An error in the database occurred;
chapter \ref{chpt_errors} provides a list
of server-side error codes.
\item negative:
An error in the client library occurred;
chapter \ref{chpt_llc} provides a list
of client-side error codes.
\end{itemize}

\subsubsection{details()}
The method returns
detailed information on the error
(or \term{None} if no error has occurred.

\subsection{Cursors}
If the result is a cursor,
four more methods are available:

\subsubsection{fetch()}
The method fetches the next bulk
of rows from the database.
After successful completion,
the cursor contains the bulk
of rows, which can be obtain by means of
the method \term{row()}.
Note that the first bulk of rows
is available immediately with
Cursor creation.

\subsubsection{row()}
The method obtains the current
bulk of rows from the cursor.
It returns a \term{Row} result.

\subsubsection{close()}
The method closes the cursor.
Since cursors are server-side
resource, it needs to be closed
explicitly; the method, however,
is usually not called directly,
but implicitly, when the result
is created with a \term{with} statement.

\subsubsection{eof()}
The method returns \term{True}
if the error state of the cursor
is \term{end-of-file} and \term{False}
otherwise.

\subsubsection{Iterator}
\term{Cursor} is an iterator.
Usually, \term{fetch()}, \term{row()}
and \term{eof()} do not need to be called
explicitly in the code.
Instead a \term{for}-loop  can be used, \eg:

\begin{python}
\begin{lstlisting}
with conn.execute("select * from sales") as cur:
    if not cur.ok():
        print "ERROR: %s " % cur.details()
        return
    for row in cur:
        # process cursor
        # row holds a single row
\end{lstlisting}
\end{python}

\subsection{Rows}
If the result is a row,
two more methods are available:

\subsubsection{field(n)}
The method returns the value
of the $n$th field (starting to count
from 0 for the first field).
The value is an \sql\ base type
converted to Python.
Conversion takes place in the obvious way
(integer and unsigned integer to int,
 float to float, text to string, \etc).
An exception are \term{time} fields.
One could expect \term{field} to return
a \term{datetime}, but that is not a case.
Time is returned as a (signed) integer
representing a \acronym{unix} timestamp
with nanosecond precision.
It can be converted with \term{now2db}
(please refer to section \ref{sec_supp}
for details).

\subsubsection{nextRow()}
The method advances to the next row.
If there was one more row, the method
returns \term{True} and the next call
to \term{field(n)} will return the
$n$th field of the next row.
Otherwise, if no more rows are available,
the method returns \term{False}.

Note that, working with a cursor
as iterator, it is not necessary
to use this method. The iterator
produces single rows.
The main use case of \term{nextRow()}
is for user-defined procedures that
return bulks of rows. An example
for this usage is:

\begin{python}
\begin{lstlisting}
with conn.execute("exec myproc()") as row:
    if not row.ok():
        print "ERROR: %s " % row.details()
        return
    while True:
        # use row.field(n) to access fields
        if not row.nextRow():
            break
\end{lstlisting}
\end{python}

\subsection{Reports}
If the result is a report,
three more methods are available:

\subsubsection{affected()}
The method returns the number of affected rows.

\subsubsection{errors()}
The method returns the number of errors.
Note that \acronym{ddl} statements will
return a status result if an error occurred.
In practice, only \acronym{dml} statements provide
this information.

\subsubsection{runTime()}
The method returns the running time of 
the \acronym{dml} or \acronym{dll} statement.

A usage example is

\begin{python}
\begin{lstlisting}
with conn.execute("load '/opt/import/client.csv into client") as rep:
    if not rep.ok():
        print "ERROR: %s " % rep.details()
        return
    print "affected: %d, errors: %d, running time: %dus" % 
             (rep.affected(), rep.errors(), rep.runTime())
\end{lstlisting}
\end{python}

\section{Exceptions and Errors}
\subsubsection{ClientError}
Is raised when an error in the client library occurred
(\eg\ a socket error).

\subsubsection{DBError}
Is raised when an error in the database occurred.

\subsubsection{WrongType}
Is raised in case of type mismatch, \ie\
trying to call a method not available for
this specific return type.

\subsubsection{explain(err)}
The function \term{explain} returns
a description of the error code passed in.

\section{Support Functions}\label{sec_supp}
Support functions are provided by module
\term{nowutil}. Currently the following
support functions are available:

\subsubsection{dt2now(dt)}
The function expects a \term{datetime}
object and converts it to a \nowdb\ timestamp.

\subsubsection{now2dt(tp)}
The function expects a \nowdb\ timestamp
and converts it to a \term{datetime} object.

\subsubsection{utc}
This is a global variable defined in \term{nowutil}.
It is the \acronym{utc} timezone object needed
for many datetime constructors and conversion functions.
It is provided for convenience.

\subsubsection{TIMEFORMAT and DATEFORMAT}
These are global constants providing the
standard \nowdb\ time and date format
strings that can be used, for instance,
in the \term{datetime}
\term{strftime} and \term{strptime} methods, \eg:

\begin{python}
\begin{lstlisting}
    stmt = "select count(*) from sales\
             where timestamp = '" + tp.strftime(TIMEFORMAT) + "'"
\end{lstlisting}
\end{python}
