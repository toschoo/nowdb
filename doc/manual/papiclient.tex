\section{Outline}
The \nowdb\ \acronym{db api} implements
\acronym{pep} 249, the Python \acronym{db api}.
Its structure is similar to the simple \acronym{api},
but there are some differences.
It is in some cases more convenient to use
than the simple \acronym{api}, but, more importantly,
it can be used with Python packages that rely on
\acronym{pep} 249, in particular \term{Pandas}.
The main disadvantage of this \acronym{api} is
that it is significantly slower
than the simple \acronym{api}
when it comes to queries with large result sets.

The \acronym{api} lives in the module \term{nowapi.py}
and provides a Connection class,
a Cursor class (which differs conceptionally
from cursors in the simple \acronym{api}),
a rich exception hierarchy and type constructors
to address database-specific types,
in particular timestamps.

\section{Connections}
Connection is conceptionally equivalent
to the Connection class in the simple \acronym{api}
and can be used in exactly the same way (\eg\
as a resource manager).
There is a difference in the constructor
\term{connect}, which expects one more argument:
a database name. If this argument is given,
connect, on success, issues a \term{use} statement.
Example:

\begin{python}
\begin{lstlisting}
with connect('localhost', '55505', 'usr', 'pwd', 'retail') as c:
    # here goes your code
    # refer to the connection as 'c'
\end{lstlisting}
\end{python}

When execution leaves the \term{with} block,
the connection's \term{close} method is invoked internally,
which works exactly as for the simple \acronym{api}.

A peculiarity of the Connection class is the \term{cursor}() method,
which creates and returns a cursor object.
The cursor concept is quite different in this \acronym{api}.
A cursor is a dedicated means to execute \sql\ statements.
\acronym{pep} 249 does not foresee
an \term{execute} method as part of the Connection class
like the simple \acronym{api};
instead statements are executed through cursors.
However, we provide such a method for convenience.
Internally, it creates a cursor, executes the statement
through this cursor and finally returns it to the caller
or, on error, raises an exception.

The \term{execute} method accepts up to three arguments:
the statement (which is mandatory), a list of parameters for this
statement (which is optional) and the row format (which, too,
is optional).

The parameters argument is expected to be a list of parameters,
which will be added to the statement according
to usual string formatting in Python.
\comment{We will need server-side parameter binding in the future.}
A simple example is:

\begin{python}
\begin{lstlisting}
with connect('localhost', '55505', 'usr', 'pwd', 'retail') as c:
     with c.execute("select count(*) from buys \
                      where stamp >= '%s'", ['2018-10-01']) as cur:
       # process cursor here
\end{lstlisting}
\end{python}

The row format is relevant only for \term{select}
statements and can be either:

\begin{itemize}
\item \term{dictrow}: 
      the rows in the result set will be presented as dictionaries
      with the \emph{name} of the fields as keys.
      This is the default row format.
      The field names are derived from the statement as follows:
      \begin{itemize}
      \item for a projection clause of the form \keyword{select} *,
            the names of the fields are obtained from the database via
            the \term{describe} statement;
      \item fields that have an alias
            (\keyword{select} \identifier{x} \keyword{as} \identifier{y})
            are named after the alias;
      \item fields that have no alias, are named as they are selected.
            For instance, the expression \keyword{count}(*) is
            literally called ``count(*)''; 
      \end{itemize}
\item \term{tuplerow}:
      the rows are presented as tuples and can be referenced by their index
      starting, as usual, from 0.
\item \term{listrow}:
      the rows are presented as lists and can be referenced by their index.
\end{itemize}

A basic usage example is

\begin{python}
\begin{lstlisting}
import nowapi as na
with na.connect('localhost', '55505', 'usr', 'pwd', 'retail') as c:
  # cursor is an iterator
  for row in c.execute("select client_key as k,\
                               client_name as n \
                          from client"):
      print("%d: %s" % (row['k'], row['n']))
\end{lstlisting}
\end{python}

And an example involving Pandas:

\begin{python}
\begin{lstlisting}
import nowapi as na
import pandas as pd
with na.connect('localhost', '55505', 'usr', 'pwd', 'retail') as c:
  sales = pd.read_sql("select origin, stamp, quanity, price \
                         from buys", c)
  # do something with dataframe 'sales'
\end{lstlisting}
\end{python}

\section{Cursors}
A cursor is a device to send \sql\ statements to the database.
The \term{execute} method of the Connection class is just a
convenience interface that internally creates and uses a
cursor to execute a statement. The explicit way to handle statements
is to create a cursor by means of the method \term{cursor}, \eg:

\begin{python}
\begin{lstlisting}
import nowapi as na
with na.connect('localhost', '55505', 'usr', 'pwd', 'retail') as c:
  cur = c.cursor()
  # do something with 'cur'
\end{lstlisting}
\end{python}

Cursors provide the \term{close} method, which
will finally be called by the \acronym{gc},
but, since cursors bind resources on the server,
they should be closed
as soon as possible.
However, cursors are resource managers.
They can, thus, be used in a \term{with} statement
which will close the cursor when leaving the block.
As we have seen, they are also iterators.
When used as iterator, the cursor is closed
after the final iteration.
There is, hence, rarely the need to call \term{close}
explicitly.

The Cursor class provides methods to
execute statements and fetch rows from result sets.
The \term{execute} method takes up to two arguments:
the \sql\ statement and a list of parameters.
It does not accept the row format argument like
the Connection method. Instead the row format
must be set explicitly \emph{before} \term{execute}
is called. For this purpose Cursor provides the
method \term{setRowFormat} which accepts one argument,
namely the row format. If the row format is not set,
it defaults to \term{dictrow}.

The rows can be fetched from the cursor by either
\term{fetchone},
\term{fetchmany} or
\term{fetchall}.
The method \term{fetchone} returns the next row
in the cursor according to the defined row format.
\term{fetchmany} returns a list of the next $n$ rows,
where $n$ is passed in as parameter. If \term{fetchmany}
is called without parameter, $n$ defaults to 1.
Finally, \term{fetchall} fetches all rows at once.

As a rationale to provide these methods,
the \acronym{pep} 249 specification refers
to performance reasons. It should be noted
that, in the \nowdb\ implementation of
\acronym{pep} 249, \term{fetchmany} has no performance
advantage over \term{fetchone}.
The method does not send a \term{fetch}
to the database for each row.
This is already handled in the underlying
C library, which always receives a bunch of rows
from which single rows are then fetched locally
in the client.

Furthermore, since Cursor is an iterator,
there is no need to use these methods at all;
the common \term{for} loop implements an optimal
row fetching mechanism.

In general, processing cursors with this \acronym{api}
is slower than with the simple one.
The reason is that \acronym{pep} 249 forces
database clients to transform all rows into Python
data types, \viz\ dictionaries or tuples,
which is very expensive and, in most cases,
unnecessary.
Using the values directly to compute the final result
is often much more efficient.
This leads to the somewhat paradoxical situation
that we create expensive Python objects
in order to create Pandas structures (like series and
dataframes), which we want to use because they
are much faster than native Python objects in the first place.

Cursors have two attributes that provide information
about the data. The most important is the \term{description},
a list of tuples of the form (field name, type code).
\term{field name} is the chosen name of the $nth$ field
and the type code the \nowdb\ type of the $nth$ field where
$n$ is the $nth$ element in the list.
The description is available, when a \acronym{dql} statement
has been executed and \term{None} otherwise.

The second attribute is the \term{rowcount}, which
is $-1$ if no statements has been executed.
For a \acronym{dml} statement, \term{rowcount}
reflects the number of affected rows;
for a \acronym{dql} statement, \term{rowcount} shall,
according to \acronym{pep} 249, reflect
the number of rows in the result set.
Here, the \nowdb\ implementation deviates from the specification.
The \term{rowcount} is not written immediately after
executing the statement. At this time the \term{rowcount}
is not yet known. Instead, the value is incremented
as the user goes along fetching rows from the cursor.

\section{Exceptions}
In the \acronym{db api}, errors are  always
communicated as exceptions. The specification
foresees a rich hierarchy of exceptions, which
is completely implemented in the \nowdb\ client,
but not all are actually raised. This section
discusses the exceptions that are currently used.

\subsection{Error}
This is the base class for other exceptions
and is not raised itself.

\subsection{InterfaceError}
This exception is raised for errors on the level
of the \acronym{api}, \eg\ parameter errors, missing
connection, fetching from a cursor that has not yet been executed, \etc\

\subsection{DatabaseError}
This exception is raised for errors in the database server.
The errors may be related to \sql\ parsing errors,
data inconsistency (\eg\ duplicated key) or non-existing
entities (\eg\ inserting into a non-existing table).

\subsection{InternalError}
This is a ``panic'' error and indicates a software bug in \nowdb.

\subsection{NotSupportedError}
The requested feature is not available.

\section{Type Constructors}
The \acronym{pep} 249 specification foresees a number
of type conversions to deal with types that are specific
for the database in question.
The \nowdb\ client implements the constructors
\term{Date} and \term{Timestamp}.
Both create a \acronym{utc} \term{datetime} object that can be used
as parameters in the \term{execute} variants.

\term{Date} expects three integer arguments:
year, month and day of month;
\term{Timestamp} expects six arguments:
year, month, day, hour, minute and second,
for instance:

\begin{python}
\begin{lstlisting}
import nowapi as na
with na.connect('localhost', '55505', 'usr', 'pwd', 'retail') as c:
  for row in c.execute("select * from buys \
                         where stamp = '%s'", \
                       [Date(2018, 1, 15)]):
    # process rows here
\end{lstlisting}
\end{python}
