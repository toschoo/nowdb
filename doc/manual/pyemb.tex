\section{Outline}
The embedded python library consists mainly
of the \term{nowdb} module, which defines
the \term{execute} function and the \term{Result} class.
Support functions (which are also available
for client-side python) are defined in
\tech{nowutil.py}.

The module uses the package \term{dateutil}
which must be installed on the system in order
to use the \term{nowdb} module.
(Please refer to chapter
\ref{chpt_install} for details.)

An embedded Python program needs to import at least
the \term{nowdb} module. The import must have
the form \term{import nowdb}, because the database
interacts with this module through the user-defined
module.

The user may define a module-level function
\term{cleanup()} in his or her module.
This function is called by the database
before the module is unloaded at the end of
the session. A meaningful use case for this
function is provided below in section \ref{sec_func}.

The embedded \acronym{api} is very similar
to that provided by the client library,
but there are also some differences.
The most obvious difference is that
there is no \term{Connection} class.
Instead, there is a module-level function
to execute \sql\ statements.

There are also model-level functions
to create results. In the client library
this is not necessary, because clients
do not need to return results.
Likewise, the \term{Result} class
has some more methods to deal with
the specifics of creating results
and returning them to the database.

\comment{
Currently, we only discuss procedures --
in the future, there will also be functions
(which can be called in \sql\ context)
and those will be discussed here as well.
}

\section{Module-level Functions}

\subsubsection{execute(statement)}
The function calls the parser against the statement and, on success,
executes the parsing result in the database.
The function returns an instance of the \term{Result} class
or, on internal errors, raises the exception \term{DBError}.

\subsubsection{success()}
The function creates a \term{Status} result
that represents \acronym{ok}.

\subsubsection{makeError(code, msg)}
The function creates a \term{Status} result
that represents an error event with error code
\term{code} and detailed information \term{msg}.

\subsubsection{makeRow()}
The function creates a \term{Row} result.
The row is initially empty.

\subsubsection{convert(typ, value)}
The function converts \term{value} according
to the \nowdb\ \sql\ type \term{typ}.
Available types are
\acronym{text}, \acronym{time},
\acronym{float}, \acronym{int}, \acronym{uint} and
\acronym{bool},
which are defined as global constants.

\section{Results}
The \term{Result} class represents
the dynamic types described in chapter \ref{chpt_sql}.
Instances of \term{Result} are
created and returned by
the \term{execute()} function
and by the \term{Result} creators discussed above.

The \term{Result} class has the following methods:

\subsubsection{rType()}
The method returns the result type,
either \acronym{status},
\acronym{report}, \acronym{row} or \acronym{cursor}.

\subsubsection{ok()}
The method returns \term{True}
if the instance does not represent
an error and \term{False} otherwise.

\subsubsection{toDB()}
The method returns a \term{void} pointer
to the underlying C structure.
The database expects such a pointer
as result from a procedure call and
wouldn't know what to do
with a Python result instance.
Any result returned to the database
must therefore call \term{toDB()} 
and return the result of that method.
The following example would return
the result of a statement as is:

\begin{python}
\begin{lstlisting}
with execute("select count(*) from sales") as cur:
    return cur.toDB()
\end{lstlisting}
\end{python}

\subsubsection{release()}
The method releases the C objects
allocated with the \term{Result} object.
Since \term{Result} is a resource manager,
it is rarely necessary to use it explicitly.

\subsubsection{Resource Manager}
Result is a resource manager.
\term{Result} can therefore
be used with the \term{with} statement, \eg:

\begin{python}
\begin{lstlisting}
with execute("select count(*) from sales") as cur:
    # here goes your code
    # 'cur' is the result and
    # if no error has occurred, cur is a cursor
\end{lstlisting}
\end{python}

\subsection{Status}
If \term{Result} is a \term{Status},
two more methods are available:

\subsubsection{code()}
The method returns the \nowdb\ error code.
The error code may be 0,
which means that the call was successful,
no error has occurred;
or it may be one of the error codes listed in
chapter \ref{chpt_errors}.

\subsubsection{details()}
The method returns
detailed information about the error
(or \term{None} if no error has occurred).

\subsection{Cursors}
If the result is a cursor,
four more methods are available:

\subsubsection{fetch()}
The method fetches the next bulk
of rows from the database.
After successful completion,
the cursor contains this bulk
of rows, which can be obtained by means of
the method \term{row()}.
Note that the first bulk of rows
is available immediately after
Cursor creation.

\subsubsection{row()}
The method obtains the current
bulk of rows from the cursor.
It returns a \term{Row} result.

\subsubsection{close()}
The method closes the cursor.
Cursors need to be closed
explicitly; the method, however,
is usually not called directly,
but implicitly, when the result
is created by means of a \term{with} statement.

\subsubsection{eof()}
The method returns \term{True}
if the error state of the cursor
is \term{end-of-file} and \term{False}
otherwise.

\subsubsection{Iterator}
\term{Cursor} is an iterator.
Usually, \term{fetch()}, \term{row()}
and \term{eof()} do not need to be called
explicitly.
Instead a \term{for}-loop  can be used, \eg:

\begin{python}
\begin{lstlisting}
with execute("select * from sales") as cur:
    if not cur.ok():
        print "ERROR: %s " % cur.details()
        return
    for row in cur:
        # process cursor
        # row holds a single row
\end{lstlisting}
\end{python}

\subsection{Rows}
If the result is a row,
four more methods are available:

\subsubsection{field(n)}
The method returns the value
of the $n$th field (starting to count
from 0 for the first field).
The value is an \sql\ base type
converted to Python.
Conversion takes place in the obvious way
(integer and unsigned integer to int,
 float to float, text to string, \etc).
An exception are \term{time} fields.
One could expect \term{field} to return
a \term{datetime}, but that is not the case.
Time is returned as a (signed) integer
representing a \acronym{unix} timestamp
with nanosecond precision.
It can be converted with \term{now2dt}
(please refer to chapter \ref{chpt_pythonclient}
for details).

\subsubsection{nextRow()}
The method advances to the next row.
If there was one more row, the method
returns \term{True} and the next call
to \term{field(n)} will return the
$n$th field of that row.
Otherwise, if no more rows are available,
the method returns \term{False}.

Note that, working with a cursor
as iterator, it is not necessary
to use this method. The iterator
produces single rows.
The main use case of \term{nextRow()}
is for user-defined procedures that
return bulks of rows. An example
for this usage is:

\begin{python}
\begin{lstlisting}
with execute("exec myproc()") as row:
    if not row.ok():
        print "ERROR: %s " % row.details()
        return
    while True:
        # use row.field(n) to access fields
        if not row.nextRow():
            break
\end{lstlisting}
\end{python}

\subsubsection{add2Row(typ, value)}
The method adds \term{value}
as \nowdb\ \sql\ type \term{typ} to
the row. This method is intended
for rows that are created in user code.
A usage example is:

\begin{python}
\begin{lstlisting}
with execute("select origin from edge where edge = 'buys'") as cur:
    our = None
    for row in cur:
       if our is None:
          our = makeRow() # create a row result
       else:
          our = closeRow() # terminate row

       with execute("select client_name from client\
                      where client_key = " +
                      str(row.field(0))) as cur2:
            for cow in cur2:
                our.add2Row(UINT, cur.field(0))
                our.add2Row(TEXT, cow.field(0))

    return our.toDB() 
\end{lstlisting}
\end{python}

This is a na\"ive join implementation:
For each edge, the function executes a \term{select}
on the \term{client} with the key found in \term{origin}.
The function creates a row ``our'' which is initialised
to \term{None}. Once it was created, it is closed,
so that each iteration will write to a new row.
Each of the rows contains two fields: the \term{origin}
from edge and the \term{client\_key} from \term{client}.
Finally, the row is returned (calling $toDB()$).

\subsubsection{closeRow()}
The method inserts an \term{end-of-row} marker into the row.
It is intended to be used with bulks of rows created by
user-defined procedures. Note that it is not necessary
to close a single row result. The marker separates
rows and is therefore needed only for bulks.

\subsection{Reports}
If the result is a report,
three more methods are available:

\subsubsection{affected()}
The method returns the number of affected rows.

\subsubsection{errors()}
The method returns the number of errors.
Note that \acronym{dml} statements will
return a \term{Status} result if an error occurred.
In practice, only \acronym{dll} statements provide
this information.

\subsubsection{runTime()}
The method returns the running time of 
the \acronym{dml} or \acronym{dll} statement.

A usage example is

\begin{python}
\begin{lstlisting}
with execute("load '/opt/import/client.csv into client") as rep:
    if not rep.ok():
        print "ERROR: %s " % rep.details()
        return
    
    print "affected: %d, errors: %d, running time: %dus" % 
             (rep.affected(), rep.errors(), rep.runTime())
\end{lstlisting}
\end{python}

\section{Exceptions and Errors}
\subsubsection{DBError}
Is raised when an error in the database occurred.

\subsubsection{WrongType}
Is raised in case of type mismatch, \ie\
trying to call a method not available for
this specific return type.

\subsubsection{explain(err)}
The function \term{explain} returns
a description of the error code passed in.

\section{Functional Cursor Library}\label{sec_func}
\comment{tbc}
