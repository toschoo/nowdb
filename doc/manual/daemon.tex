\section{Outline}
\nowdb\ implements a classic client/server approach.
The \nowdb\ daemon acts as the server side,
that is, the daemon process implements the server;
the terms \term{daemon} and \term{server}
are therefore interchangeable.
One server may run one or more databases,
which reside in the same base directory
on the server machine.
Several servers may run on the same machine,
but need to use different database directories
and, of course, different ports.

Clients connect to the server
through \acronym{tcp/ip}.
A \acronym{tcp/ip} connection
constitutes a \term{session}.
It is always possible to execute \sql\
statements within a session and
it may be possible to execute
server-side Lua or Python scripts,
depending on the command line options
passed to the process on start-up.

For Lua and Python, the session also
defines the scope and lifetime
of global and module-wide variables.
This makes it possible to implement
stateful server modules that maintain
state for as long as the session exists.
This feature allows implementing
a wide range of patterns,
such as complex queries using internally
one or more cursors or partial computation results,
but also long running publish \& subscribe
services.

The \nowdb\ daemon is a \acronym{posix}-compliant
command line program optimised and tested
for 64bit machines running under Linux.
Clients are available for a wide range
of languages (C, Python, Lua, Go, \etc)
and platforms (Linux, Apple, Windows).
In the future, the \nowdb\ daemon may
address other platforms as well.
But, currently, this is not a priority.

The behaviour of the daemon process
is influenced by a number
of command line parameters and
environment variables.
Configuration files are (\comment{currently})
not necessary.
In the following two sections,
we will discuss available
command line parameters and
environment variables.
The last section will present
the \nowdb\ docker.
The docker eases the installation
especially of the server-side components
and the maintenance of \nowdb\ servers.

\section{Command Line Parameters}
\begin{minipage}{\textwidth}
The program accepts the following
command line options:

\begin{itemize}
\item \tech{-b}: The base directory,
where databases are stored.
(default is the current working directory).

\item \tech{-s}: the binding domain, default:
any. If set to a host or a domain, the server
will accept only connections from that host
or domain. Example: \tech{-s localhost} does
only accept connections from the server.
The host or domain can be given as name (\term{localhost})
or as \acronym{ip} address (\tech{127.0.0.1});

\item \tech{-p}: the port to which
the server will listen; default is 55505,
but any other (free) port may be used.
 
\item \tech{-c}: number of connections accepted at the same time.
If the argument is 0, indefinitely many
simultaneous connections are accepted
(until the server is not able to serve more requests);
otherwise, for \tech{-c n},
$n$ being a positive integer,
the server accepts
up to $n$ connections at the same time and
will refuse to accept more if this number is reached.
The default value for this parameter is 128;

It should be noted that, up to a certain threshold
(which is not configurable) sessions are
reused, that is, sessions enter a connection pool
from which they are fetched, when new connection
requests arrive.

\item \tech{-q}: runs in quiet mode
(\ie\ no debug messages are printed to standard error);
\item \tech{-n}: does not print the starting banner;
\item \tech{-y}: activates server-side Python support;
\item \tech{-l}: activates server-side Lua support;
\item \tech{-V}: prints version information to standard output;
\item \tech{-?} or \tech{-h}: prints a help message to standard error.
\end{itemize}
\end{minipage}

\section{Environment Variables}
The following environment variables have influence
on the behaviour of the program:

\tech{LD\_LIBRARY\_PATH}\\
This is the Linux search path for shared libraries.
If libraries needed by the daemon are not in
the standard path \tech{/usr/lib}, the path
where those libraries were actually installed
must be added.

\begin{minipage}{\textwidth}
Non-standard libraries used by the daemon are:
\begin{itemize}
\item The \term{tsalgo} library that provides
      fundamental algorithms and datastructures;
\item The \term{beet} library that provides
      on-disk B$^+$Trees needed for indices;
\item The \term{zstd} and \term{lz4} libraries
      both implementing compression algorithms;
\item The \term{icu} library providing \acronym{utf}-8
      handling for C;
\item The \term{python} 3.5 library;
\item The \term{lua} 5.3 library.
\end{itemize}
\end{minipage}

\tech{PYTHONPATH}\\
This is the search path used by the Python interpreter
to find imported modules and packages.

\tech{LUA\_PATH}\\
This is the search path used by the Lua interpreter
to find imported modules.
The syntax for this variable is special.
For details, please refer to the Lua 5.3 documentation.

\tech{LUA\_CPATH}\\
This is the search path used by the Lua interpreter
to find C libraries that are used in Lua modules.
The syntax for this variable is special.
For details, please refer to the Lua 5.3 documentation.

\tech{NOWDB\_LUA\_PATH}\\
This is the search path used by \nowdb\ to find top-level
user modules. The syntax for this variable is special
and is discussed in detail in section \ref{chpt_luaemb}.

\tech{NOWDB\_HOME}\\
This variable shall hold the path to \nowdb\ resources,
such as the Lua and Python \nowdb\ modules. Where this
path is located on your system depends on the installation.
Please refer to \ref{chpt_install} for details.

\section{The \nowdb\ Docker}
The \nowdb\ docker provides a ready-made environment
for the server to run. All libraries are installed,
relevant environment variables are set and the daemon
is started with a consistent parameter setting.
All decisions can be reviewed by the user and alternative
settings can be chosen.

Within the docker, \nowdb\ is started
by means of a shell script.
One possible way to start the docker is:

\cmdline{
docker run --rm -p 55505:55505 $\backslash$\\
\hspace*{2cm} -v /opt/dbs:/dbs -v /var/log:/log $\backslash$\\
\hspace*{2cm} -d nowdbdocker /bin/bash -c "/nowstart.sh"
}

This command creates the docker container and starts it.
The parameters are
\begin{itemize}
\item \tech{--rm}
instructs the docker daemon to remove
the container immediately after it will have stopped.

\item \tech{-p 55505:55505} binds the host port \term{55505}
to the same docker port.

\item \tech{-v} maps the host path
\tech{/opt/dbs} to the docker path \tech{/dbs} and
the host path \tech{/var/log} to the docker path \tech{/log}.

\item \tech{-d} means the docker runs in the background
(\term{detached}).

\item \tech{nowdbdocker} is the name of the docker image.

\item \tech{/bin/bash} is the command to be executed
within the docker; \tech{-c} passes a command
to be executed to \tech{bash},
namely \tech{/nowstart.sh}.
\end{itemize}

For more information on how to start, stop and manage
dockers, please refer to the docker documentation at
\url{https://docs.docker.com/}.

The script \tech{nowstart.sh}, contains the
instructions to start the \nowdb\ daemon:

\cmdline{
export LD\_LIBRARY\_PATH=/lib:/usr/lib:/usr/local/lib \\
export NOWDB\_HOME=/usr/local \\
export LUA\_PATH=/lua/?.lua \\
export LUA\_CPATH=/lua/?.so \\
export NOWDB\_LUA\_PATH=$\star$:/lua \\
nowdbd -b /dbs -l 2>/log/nowdbd.log
}

The script sets all relevant environment variables,
namely the \tech{LD\_LIBRARY\_PATH},
\tech{NOWDB\_HOME},
\tech{LUA\_PATH} and \tech{LUA\_CPATH}
as well as the \tech{NOWDB\_LUA\_PATH}.
Note that the \tech{PYTHONPATH} is not set.
This is because Lua is considered the
default server-side language.

Then, the \nowdb\ daemon itself is started.
The script passes two options to the daemon:
the base directory where all databases
managed by this particular daemon live (\tech{-b /dbs})
and the \tech{-l} switch, which activates
server-side Lua support.

The docker also contains the \nowdb\ command line client
and the client libraries for C, Python, Lua and Go.
This way the docker can be used to work with the server
interactively without the need to additionally install
programs on the server or the client; the docker can
also be used to prototype a complete solution including
server and client applications.
