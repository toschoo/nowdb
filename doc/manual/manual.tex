%% =======================================================
%% (c) 2018 Tobias Schoofs
%% =======================================================
%% NowDB
%% =======================================================

% Plain Style
\documentclass{scrreprt}

%% =======================================================
%% (c) Tobias Schoofs
%% =======================================================
%% Commands 4 Programmers
%% =======================================================

%include lhs2TeX.fmt
%include lhs2TeX.sty

\usepackage[pdftex]{graphicx}
\usepackage{ucs}
\usepackage[utf8x]{inputenc} 
\usepackage{tabto}
\usepackage[russian,portuguese,german,english]{babel}
\usepackage{CJK}
\usepackage{amsfonts}
\usepackage{amsfonts}

\usepackage{amsmath}
\usepackage{amssymb}
\usepackage{amsthm}
\usepackage{amscd}

\usepackage{siunitx}

\usepackage{listings}
\usepackage{longtable}

\usepackage{tikz}
\usepackage{pgfplots}

\usepackage{relsize}
\usepackage{xcolor}

\long\def\ignore#1{}

\newcommand{\acronym}[1]{\textsc{#1}}

\newcommand{\term}[1]{\textit{#1}}
\newcommand{\tech}[1]{{\ttfamily #1}}
\newcommand{\latin}[1]{\textit{#1}}
\newcommand{\speech}[1]{\textit{#1}}

\newcommand{\ie}{\textit{i.e.}}
\newcommand{\eg}{\textit{e.g.}}
\newcommand{\etc}{\textit{etc.}}
\newcommand{\viz}{\textit{viz.}}
\newcommand{\vs}{\textit{vs.}}

\newcommand{\sql}{\acronym{sql}}

\newcommand{\code}[1]{{\ttfamily #1}}
\newcommand{\cmdline}[1]{{\ttfamily #1}}

\newenvironment{sqlcode}{
\small
\begin{minipage}{\textwidth}
\lstset{language=sql,
        keepspaces=true,
        showspaces=false,
        showstringspaces=false}
}{
\end{minipage}
}

\newenvironment{python}{
\small
\begin{minipage}{\textwidth}
\lstset{language=python,
        keepspaces=true,
        showspaces=false,
        showstringspaces=false}
}{
\end{minipage}
}

\newcommand{\keyword}[1]{\textbf{#1}}
\newcommand{\identifier}[1]{\textit{#1}}

\newcommand{\Rom}[1]{\uppercase\expandafter{\romannumeral #1\relax}}

\newcommand{\CC}{C\nolinebreak[4]\hspace{-.05em}\raisebox{.3ex}{\relsize{-2}{\textbf{++}}}}
\newcommand\csharp{C\nolinebreak[4]\hspace{-.02em}\raisebox{.3ex}{\relsize{-1}{\#}}}

\newcommand{\comment}[1]{\textcolor{red}{#1}}

\newcommand{\nowdb}{\textsc{n}o\textsc{wdb}}

\newcommand{\connect}[2]{
  \draw [-,color=black] (#1) to (#2)
}

% tikz
% \newcommand{\drawDataPoint}{\draw circle (0.2)}
% \drawDataGroup

% \renewcommand{\gcd}{\textsc{gcd}}


\usepackage{authblk}
\usepackage[toc,page]{appendix}
\usepackage{url}
\usepackage{hyperref}
\usepackage{algorithmic}
\usepackage{nicefrac}

\begin{document}
\setlength{\parindent}{0pt}
\setlength{\parskip}{8pt}

\title {NoWDB}
\author {tobias.schoofs@gmx.net}
\date{\today}
%\includegraphics[scale=0.75]{sonic.jpg}\\[24pt]\today}
\maketitle
\tableofcontents

\chapter{Introduction}\label{chpt_intro} 
\nowdb\ is a kind of database.
It merges the concepts
of \term{graph} and \term{timeseries} database.
Timeseries databases typically have simple
data models centred around timelines
consisting of pairs of the form
$(timestamp,value)$
with additional $tags$ to
distinguish thematically different timelines.
An example may be a weather forecast
application with timelines describing
temperature, humidity and air pressure
at certain locations. The values would
reflect these measurements and timestamps
would refer to the points in time when
the measurements were taken. Tags would
be used to distinguish timelines
(temperature, humidity, pressure) and
to identify the location from where the
respective measurement comes.

Graph databases replace the set-theoretic
fundaments of relational databases by
graph theory. Graph databases do not deal
with relations over sets, but with
sets of vertices that are connected
by edges. In Twitter-like applications,
vertices may represent users.
The connections between users such as
\term{following} could then be expressed
in terms of edges between vertices.
Applications built on top of graph databases
typically focus on finding relations
between vertices; a goal may be to decide
whether a user $A$ belongs to the network
of a user $B$ where a follower of a
follower of $B$ is considered part of
$B$'s network. Another challenge may be
to compute how many users have seen
a certain tweet or how many users
see tweets of user $A$ in general.

While timeseries databases stress
fast processing of large volumes of
data with similar and, typically, simple structure,
graph databases focus on efficiently handling
data with growing complexity.

\nowdb\ aims to provide the
performance advantages of timeseries databases
using concepts from graph databases
to allow more complex data models
than usually seen with timeseries databases.
\nowdb\ can thus be applied to a wider
range of applications than pure
timeseries databases without losing
their performance advantage.

\nowdb\ is in particular strong with
data that can be organised in some variant of the
\term{star schema}. Relations between
fact tables and dimensional tables
are expressed in terms of
weighted and timestamped edges
between dimensional data that
are represented as complex vertices.
In contrast to traditional
\term{star schema} applications,
\nowdb\ is not limited to data analysis.
Many features stress real-time data
processing providing \term{publish and subscribe},
online data filtering and integration with big-data
infrastructure.

\newpage
This manual documents the main features
of the database and discusses important
use cases. The next chapter provides
a \term{Quick Start} tutorial that helps
understanding the concepts behind \nowdb\
and introduces the most important tools.
The chapter will close
with an overview of the remaining
chapters of this document.

\comment{
Throughout the document, the reader will encounter
red comments like this one.
These comments aim to clarify the current state
of the prototype. They, in particular, draw
attention to features that are not yet available
or to shortcomings of their current implementation.
This way, the manual also serves as an agenda
for the months to come. The goal is indeed
to get rid of all those comments. When the last
red line has gone, \nowdb\ is ready for the first
release.
}

\chapter{Quick Start}\label{chpt_quickst}
The easiest way to get started
is to use the \nowdb\ docker containing
the database server.

\comment{more instructions of how to get it $\dots$}

The docker does not start the database by itself.
You need to start it explicitly. There is a script
called \tech{nowstart.sh} in the root of the docker
that does that.
(You may want to adapt this script to your specific needs!)
Here is a reasonable way to start the docker:

\code{
docker run --rm -p 55505:55505 $\backslash$\\
\hspace*{2cm} -v /opt/dbs:/dbs -v /var/log:/log $\backslash$\\
\hspace*{2cm} -d nowdbdocker /bin/bash -c "/nowstart.sh"
}

The docker command is \term{run}.
It create a docker container and starts it.
We add a number of parameters:
\begin{itemize}
\item \tech{--rm}
instructs the docker daemon to remove
the docker immediately after it will have been stopped.

\item \tech{-p 55505:55505} binds the host port \term{55505}
to the same docker port.

\item \tech{-v} maps a host path
(namely, the path \tech{/opt/dbs}) to a docker path
(namely, to \tech{/dbs}) and
the path \tech{/var/log} to \tech{/log}.

\item \tech{-d} means the docker runs in the background
(\term{detached}).

\item \tech{nowdbdocker} is the name of the docker.

\item \tech{/bin/bash} is the command to be executed
within the docker; \tech{-c} passes a command
to be executed,
namely \tech{/nowstart.sh},
to the \tech{bash}.
\end{itemize}

The script \tech{nowstart.sh}, contains the
instructions to start the nowdb daemon (whose name is
\tech{nowdbd}).

It first sets some environment variables:

\code{
export LD\_LIBRARY\_PATH=/lib:/usr/lib:/usr/local/lib
}

This sets the search path for shared libraries.

\code{
export PYTHONPATH=/pynow:\$PYTHONPATH
}

This sets the search path for Python modules
(we will discuss that later).

Then the script starts the daemon iself:

\code{
nowdbd -b /dbs -y 2>/log/nowdbd.log
}

The script passes two options to the daemon:
the base directory where all databases
managed by this daemon live (\tech{-b /dbs})
and the \tech{-y} switch, which activates
server-side Python support.

Now the daemon is listening to port 55505
and ready to respond to database requests.
The daemann starts by printing a welcome banner
to standard output:

\begingroup
\small
\begin{verbatim}
+---------------------------------------------------------------+ 
 
  UTC 2018-10-09T12:03:21.631000000
 
  The server is ready
 
    - with python support enabled
 
+---------------------------------------------------------------+ 
  nnnn   nnnn          nnnn    nnnnnn       nnnnnn       nnnnnn  
    wi  i   wi       i      i    iw           iw           er   
    wi i     wi     n        n    iw         e  wi        e    
    wii      wi    wi        iw    iw       e    wi      e        
    wi       wi    wi        iw     iw     e      wi    e        
    wi       wi     n        n       iw   e        wi  e        
    wi       wi      i      i         iw e          wie          
   nnnn     nnnn       nnnn             n            n            
+---------------------------------------------------------------+ 

connections: 128
port       : 55505
domain     : any
\end{verbatim}
\endgroup

Here are some more options provided by
the \term{nowdbd} tool:

\begin{itemize}
\item \tech{-b} The base directory,
where databases are stored.
(default is the current working directory).

\item \tech{-s} the binding domain, default:
any. If set to a host or a domain, the server
will accept only connections from that host
or domain. Example: \tech{-s localhost} does
only accept connections from the server.
The host or domain can be given as name (\term{localhost})
or as \acronym{ip} address (\tech{127.0.0.1});

\item \tech{-p} the port the server to which
the server will listen; default is 55505,
but any other (free) port may be used.
 
\item \tech{-c} size of the connection pool.
If the argument is 0, the connection pool grows
indefinitely; otherwise, for \tech{-c n},
$n$ being a positive integer,
the server will create a connection pool
up to $n$ sessions and then refuse to accept
more connections. The default pool size is 128;
\comment{Notice that due to a memory leak
in the Python interpreter, there is no way
to instruct the server to accept indefinitely
many connections, but to only maintain $n$
in the pool.}

\item \tech{-q} runs in quiet mode
(\ie\ no debug messages are printed to standard error);
\item \tech{-n} does not print the starting banner;
\item \tech{-y} activate server-side Python support;
\item \tech{-l} activates server-side Lua support;
\item \tech{-V} prints version information to standard output;
\item \tech{-?} or \tech{-h} prints a help message to standard output.
\end{itemize}

Once \term{nowdbd} is running, we can connect to pass
queries to the server. There is a tool to do this from the commandline
called \term{nowclient}.

\comment{how to install the client tools???}

To send a query to the server
the client tool can be used like this:

\code{
nowclient -d retail -Q "select count(*) from sales where customer=12345"
}

In this form, the client will try to connect to a server
running on the same host and listening to port 55505.
Furthermore, it will request to use the database \term{retail}
and send the query indicated by the \tech{-Q} parameter.

If the connection is successful, the client will print some
processing information to standard error and the query result
to standard output, \eg:

\code{
executing "use retail" \\
OK \\
executing "select count(*) from sales where customer=12345" \\
59
}

The option \tech{-q} would suppress the processing information.
We would then only see:

\code{
59
}

Here are more options supported by the client tool:
\begin{itemize}
\item \tech{-s} 
The server address or name, \eg\ \tech{myserver.mydomain.org} or
\tech{127.0.0.1}. Default is \tech{127.0.0.1};

\item \tech{-p}
the port to which the database is listening. Default: 55505;

\item \tech{-d}
the database to which we want to connect. 
Default: no database at all, which means
that we cannot send queries without naming a database.
Below we will look at alternatives to using this parameter;

\item \tech{-Q}
the query we want the database to process;

\item \tech{-t}
print some (server-side) timing information to standard error;

\item \tech{-q}
quiet mode: don't print processing information to standard error.

\item \tech{-V} prints version information to standard output;
\item \tech{-?} or \tech{-h} prints a help message to standard output.
\end{itemize}

The client tool is able to read from standard input;
this way, more than one query can be processed at a time.
The following command processes the same query as the one above,
but uses standard input instead of the options \tech{-d} and \tech{-Q}:

\code{
echo "use retail;select count(*) from sales where customer=12345;" $| \backslash$ \\
\hspace*{2cm} nowclient
}

Notice that, using standard input,
we need to terminate single \sql\ statements
by a semicolon. This is even true for the last statement.
Leaving the semicolon out would lead to an error.

Of course, we can do much more useful things than just
getting rid of the options.
The main point of reading from standard input is
that we can put \sql\ statements into a file and
cat it to nowclient. A useful example may be:

\begin{sqlcode}
\begin{lstlisting}
drop schema retail if exists;
create schema retail; use retail;

create large table sales set stress=constant;
create table statistics;

create medium index idx_sales_eds on sales (edge, destin);
create index idx_sales_eor on sales (edge, origin);

create type product (
        prod_key uint primary key,
        prod_desc text,
        prod_price float
);
create type client (
        client_key uint primary key,
        client_name text
);
create edge buys (
        origin client,
        dest product,
        weight uint,
        weight2 float
);
load '/opt/data/products.csv' into vertex use header as product;
load '/opt/data/clients.csv' into vertex use header as client;
load '/opt/data/sales.csv' into sales;
\end{lstlisting}
\end{sqlcode}

Let's assume we had this code in a file called
\tech{create\_retail.sql}; then we could
send it to the nowdbclient:

\code{
cat create\_retail.sql | nowclient
}

which would create the retail database.

The script shows some of the peculiarities of \nowdb.
The beginning is quite regular \sql:

\begin{sqlcode}
\begin{lstlisting}
drop schema retail if exists;
create schema retail; use retail;
\end{lstlisting}
\end{sqlcode}

The first line drops the database retail,
\ie\ it removes all its data physically
from disk. The \term{if exists} part
is included to avoid an error 
(and hence the termination of the script)
in the case
the database does indeed exist.

The keyword \term{schema} is intended to 
make the \nowdb\ \sql\ dialect more
approachable for people with experience
in other \sql\ dialects. Instead of
\term{schema}, one could also say
\term{database}. 
Yet another way to say the same thing
in \nowdb\ is \term{scope}, which
means the scope in which a set of
vertices is valid.

In the second line the schema
(or database or scope) `retail'
is created. Since we have dropped it
if it already had existed before,
this statement will not cause
an error when the database already exists.
An alternative way of doing things is:

\begin{sqlcode}
\begin{lstlisting}
create schema retail if not exists;
\end{lstlisting}
\end{sqlcode}

In that case, we would maintain the database
if it exists already. The script, however,
would continue, because no error would occur.

The third statement (still in the second line)
instructs the \nowdb\ to use the newly created
schema `retail' in everything that follows.

The next line is a bit uncommon:

\begin{sqlcode}
\begin{lstlisting}
create large table sales set stress=constant;
\end{lstlisting}
\end{sqlcode}

The statement creates a table called `sales';
however, since \nowdb\ is not a relational,
but a graph database, there are no 
`tables' with the meaning of that term
in the relational world.
In fact, tables are just units of storage 
for edges. \comment{A better naming convention
would probably be `tablespace'.}

The statement explicitly says that we
want a \emph{large} table and that there
will be \emph{constant} stress (\ie\ ingestion load
on that table). The \term{create table} statement,
hence, is much more oriented to storage
and processing details than to the logical structure
of the table (as it would be in a relational database).

A similar is true for the index creation:

\begin{sqlcode}
\begin{lstlisting}
create medium index idx_sales_eds on sales (edge, destin);
\end{lstlisting}
\end{sqlcode}

This statement creates a \emph{medium} index on table sales
with the fields `edge' and `destination' as index keys.
Index sizing is a difficult topic and will be discussed
in chapter \ref{chpt_sizing}.
As a rule of thumb, it is almost always better to assume
small sizing. Large indices have large storage nodes
and those nodes are very often read from and written
to disk. Only when we know that our index will have
many keys and many data points per key, it is advisable
to use index sizing. The default sizing (used in the next
line) indicatively is \term{small}.

The next two blocks of code create the types
`product' and `client'. This should look a lot like
what you probably know as \term{tables}. Indeed types
are vertices stored as \term{column-oriented} tables.
\comment{Well, not entirely true at the moment.
In fact, completely column-oriented tables would cause
some more work in the underlying mechanisms. Anyway,
that is the direction to go and sooner or later
the statement will be true $\dots$}

Notice that every type needs a primary key that consists
of one attribute.

The attribute types are \term{uint}, \term{float} and \term{text}.
The first is a 64bit unsigned integer;
the second is a 64bit floating point number
(a.k.a. \term{double} in languages like C);
\term{text} is a string of up to 255 \acronym{utf8} characters.
For more information on \sql\ types, please refer to chapter
\ref{chpt_sql}.

The next block defines an edge. Remember
that edges, contrary to vertices, have a fixed layout.
However, we can define the types of the fields of an edge
and we can rename them. \comment{Renaming is not yet possible.}
Here, we define four of the fields of the edge.
The origin has type \term{client} (which we have defined just above)
and the destination has type \term{product}.
The edge has two weights: one is uint and the other float.

Finally, the script loads data from three different \acronym{csv}s
into the database. Loaders have the same effect as the \sql\
\term{insert} statement, but are much more efficient.
The drawback of \term{insert} is that each statement
needs the whole cycle of \sql\ parsing and execution,
while loaders only need one cycle. Since a data source
can contain millions or even billions of rows,
loading is way more efficient than inserting in most cases.

The first two \acronym{csv}s in the script contain vertices.
As such they need to have a header and we need to instruct
the loader of how to interpret the data (\tech{use header as client}).
Since edges always have the same format, they don't need a header
and we do not give any instructions of how to interpret the data.

\acronym{csv} is only one format supported by the database loader.
There are many more and the loader even allows users to define
their own (binary) formats using Apache Avro. Loading binary data 
can be much faster than loading textual formats such as \acronym{csv}.
Using a serialisation system like Avro eases
interoperability of the database with other
external systems and applications and it
significantly eases version management should data formats
change over time (what they always do!). 
For more details on data loaders, please refer to chapters
\ref{chpt_sql} and \ref{chpt_loader}.

The alternative to loading data is, of course, the conventional
\term{insert} statement:

\begin{sqlcode}
\begin{lstlisting}
insert into client(9000001, 'Popeye the Sailor');
insert into product(100001, 'Spinach, 450g net', 1.99);
\end{lstlisting}
\end{sqlcode}

These two statements insert a client and a product respectively.
We can also name the attributes explicitly, like:

\begin{sqlcode}
\begin{lstlisting}
insert into product(prod_key, prod_desc, prod_price)
              (100002, 'Candy Cigarettes, 20', 2.49);
\end{lstlisting}
\end{sqlcode}

Now we insert a bunch of edges:

\begin{sqlcode}
\begin{lstlisting}
insert into sales (edge, origin, destin, timestamp, weight, weight2)
           ('buys', 9000001, 100001, '1929-01-17T09:35:12', 1, 1.99)
insert into sales (edge, origin, destin, timestamp, weight, weight2)
           ('buys', 9000001, 100001, '1929-01-19T10:15:01', 2, 3.98)
insert into sales (edge, origin, destin, timestamp, weight, weight2)
           ('buys', 9000001, 100001, '1929-01-20T17:12:55', 3, 5.97)
insert into sales (edge, origin, destin, timestamp, weight, weight2)
           ('buys', 9000001, 100001, '1929-01-22T08:27:32', 1, 1.99)
insert into sales (edge, origin, destin, timestamp, weight, weight2)
           ('buys', 9000001, 100001, '1929-01-25T12:09:59', 1, 1.99)
insert into sales (edge, origin, destin, timestamp, weight, weight2)
           ('buys', 9000001, 100001, '1929-01-26T21:19:44', 2, 3.98)
insert into sales (edge, origin, destin, timestamp, weight, weight2)
           ('buys', 9000001, 100002, '1929-01-22T08:27:51', 1, 2.49)
\end{lstlisting}
\end{sqlcode}

Worth noticing here is the time format,
which follows \acronym{iso}-8601.

The format is
\begin{itemize}
\item 4 digits for the year and hyphen
\item 2 digits for the month and hyphen
\item 2 digits for the day of month
\item `T' to mark the beginning of the time section
\item 2 digits for the hour and colon
\item 2 digits for the minute and colon
\item 2 digits for the second and,
\item if finer grain is necessary,
a dot followed by up to 9 digits
for the nanosecond
\end{itemize}

This format can be used anywhere in \nowdb\ \sql.
However, it is also possible to define custom
date and time formats. How to do this is discussed
in chapter \ref{chpt_sql}.

Now that we have inserted some data
into our database, we are able to perform selects, \eg:

\code{
nowclient -d retail -Q
"select count(*) from sales $\backslash$ \\
\hspace*{4.7cm} where edge='buys' $\backslash$ \\
\hspace*{4.7cm} and origin=9000001"
}

which would give us 7 and would count Popeye's visits to the supermarket.
But we could also count how often Spinach was bought:

\begin{sqlcode}
\begin{lstlisting}
select count(*) from sales
 where edge='buys'
   and destin=100001
\end{lstlisting}
\end{sqlcode}

which would give us 6.
Or we could ask how much Popeye bought and paid per type of product:

\begin{sqlcode}
\begin{lstlisting}
select edge, destin, count(*), sum(weight), sum(weight2)
  from sales 
 where edge='buys' 
   and origin=9000001 
 group by edge, destin
\end{lstlisting}
\end{sqlcode}

which would give us:

\begin{verbatim}
buys;100001;6;10;19.9000
buys;100002;1;1;2.4900
\end{verbatim}

Two aspects are worth mentioning here:
First, the statement with the group clause above
would use the index we defined on sales
with fields $(edge, destin)$.
To force the use of an index in a group clause
there must be an exact match between the fields
in the group clause and the index; also the order
of keys in the index and in the clause must be the same.
For instance, \code{group by destin, edge} would
not have forced the use of the index.
Whether using an index or not for grouping
is an advantage 
depends on many factors.
For a detailed discussion of indices and
optimising queries please refer to chapter \ref{chpt_opt}.

Second, you may have noticed that the output
of the client tool does not resemble the classical
pretty-printed output produced by most database
clients today. The advantage of such output is
that it is easier for humans to read.
The \acronym{csv}-like output shown above, however,
is easer to read for machines, especially
when you want to send the output through pipes
to other programs, like this:

\code{
nowclient -q -d retail -Q "select * from sales" | cut -d";" -f2 | $\dots$
}

On the other hand, there are tools that
produce more readable output from \acronym{csv} input,
such as \tech{csvlook} from the \tech{csvkit} package.\footnote{Have
a look at
\url{https://github.com/jeroenjanssens/data-science-at-the-command-line}}
To obtain a traditional pretty-printed output you could do:

\code{
cat query.sql | nowclient | csvformat -d";" | $\backslash$ \\
    header -a 'edge,product,count,quantity,price' | csvlook
}

and would obtain for the grouping query used above:

\begin{verbatim}
|----------+---------+-------+----------+---------|
|  edge    | product | count | quantity | price   |
|----------+---------+-------+----------+---------|
|  buys    | 100001  | 6     | 10       | 19.9000 |
|  buys    | 100002  | 1     | 1        | 2.4900  |
|----------+---------+-------+----------+---------|
\end{verbatim}

Here is a more typical time series query illustrating
the advantage of the pretty printer:

\begin{sqlcode}
\begin{lstlisting}
select destin, timestamp, weight, weight2
  from sales 
 where edge='buys' 
   and origin=9000001 
 order by timestamp
\end{lstlisting}
\end{sqlcode}

which, with the same technique used above, shows:

\begin{verbatim}
|----------+---------------------+----------+---------|
|  product | timestamp           | quantity | price   |
|----------+---------------------+----------+---------|
|  100001  | 1929-01-17T09:35:12 | 1        | 1.9900  |
|  100001  | 1929-01-19T10:15:01 | 2        | 3.9800  |
|  100001  | 1929-01-20T17:12:55 | 3        | 5.9700  |
|  100001  | 1929-01-22T08:27:32 | 1        | 1.9900  |
|  100002  | 1929-01-22T08:27:51 | 1        | 2.4900  |
|  100001  | 1929-01-25T12:09:59 | 1        | 1.9900  |
|  100001  | 1929-01-26T21:19:44 | 2        | 3.9800  |
|----------+---------------------+----------+---------|
\end{verbatim}

We can also select from vertices,
but instead of a table, we use the type:

\begin{sqlcode}
\begin{lstlisting}
select prod_price from product
 where prod_key = 100001;
\end{lstlisting}
\end{sqlcode}

would give
\begin{verbatim}
1.99
\end{verbatim}

and

\begin{sqlcode}
\begin{lstlisting}
select client_name from client
 where client_key = 9000001;
\end{lstlisting}
\end{sqlcode}

would give
\begin{verbatim}
Popeye the Sailor
\end{verbatim}

\comment{
This is as it should be.
Unfortunately, there are some inconsistencies stemming 
from the fact that vertices are column-oriented.
The syntax is currently `select field from vertex as type',
which is ugly and pointless.
Even worse, grouping, ordering and aggregates
are not yet implemented for vertices.
That must be corrected \acronym{asap}.
}

Much more typical for \nowdb, however,
is to use vertices together with edges. Edges connect vertices
and can therefore be seen as the `relations' in \nowdb.
What we typically want is either find the vertex
at the other end of the edge (\eg\ find the destination
for a given origin) or to look at edges with the attributes
of the vertices added to them.

Both patterns are, in \sql, instances of \term{joins}.
An instance of the first pattern would be:

\begin{sqlcode}
\begin{lstlisting}
select timestamp, prod_desc, prod_price
  from sales join product
    on prod_key = destin
 where edge = 'buys'
   and origin = 9000001;
\end{lstlisting}
\end{sqlcode}

\begin{verbatim}
|---------------------+----------------------+---------|
| timestamp           | product              | price   |
|---------------------+----------------------+---------|
| 1929-01-17T09:35:12 | Spinach, 450g net    | 1.9900  |
| 1929-01-19T10:15:01 | Spinach, 450g net    | 1.9900  |
| 1929-01-20T17:12:55 | Spinach, 450g net    | 1.9900  |
| 1929-01-22T08:27:32 | Spinach, 450g net    | 1.9900  |
| 1929-01-25T12:09:59 | Spinach, 450g net    | 1.9900  |
| 1929-01-26T21:19:44 | Spinach, 450g net    | 1.9900  |
| 1929-01-22T08:27:51 | Candy Cigarettes, 20 | 2.4900  |
|---------------------+----------------------+---------|
\end{verbatim}

Note the difference in the price column.
In the previous query we used \term{weight2} of sales,
which (as you may have realised)
is the multiplication of the product price and the value
in \term{weight}, which, in its turn,
represents the number of items.
Here, however, we use the value in \term{prod\_price}
which is the base price of one unit of that product.

We can, of course, combine joins with grouping:

\begin{sqlcode}
\begin{lstlisting}
select edge, destin, count(*), sum(prod_price)
  from sales join product
    on prod_key = destin
 where edge = 'buys'
   and origin = 9000001
 group by edge, destin;
\end{lstlisting}
\end{sqlcode}

\begin{verbatim}
|----------+---------+-------+---------|
|  edge    | product | count | price   |
|----------+---------+-------+---------|
|  buys    | 100001  | 6     | 11.9400 |
|  buys    | 100002  | 1     | 2.4900  |
|----------+---------+-------+---------|
\end{verbatim}

Note again the sum of the price which, here,
is just the sum of the base price per unit of the product.

The point about the first joining pattern
is that it joins only one of the two vertices
with the edge. The second pattern is somewhat more complex:
it joins both vertices, like:

\begin{sqlcode}
\begin{lstlisting}
select timestamp, prod_desc, prod_price, client_name
  from sales
  join product
    on prod_key = destin
  join client
    on client_key = origin
 where edge = 'buys'
   and origin = 9000001;
\end{lstlisting}
\end{sqlcode}

The result of this query would be:

\begin{verbatim}
|---------------------+----------------------+---------+-------------------|
| timestamp           | product              | price   | client            |
|---------------------+----------------------+---------+-------------------|
| 1929-01-17T09:35:12 | Spinach, 450g net    | 1.9900  | Popeye the Sailor |
| 1929-01-19T10:15:01 | Spinach, 450g net    | 1.9900  | Popeye the Sailor |
| 1929-01-20T17:12:55 | Spinach, 450g net    | 1.9900  | Popeye the Sailor |
| 1929-01-22T08:27:32 | Spinach, 450g net    | 1.9900  | Popeye the Sailor |
| 1929-01-25T12:09:59 | Spinach, 450g net    | 1.9900  | Popeye the Sailor |
| 1929-01-26T21:19:44 | Spinach, 450g net    | 1.9900  | Popeye the Sailor |
| 1929-01-22T08:27:51 | Candy Cigarettes, 20 | 2.4900  | Popeye the Sailor |
|---------------------+----------------------+---------+-------------------|
\end{verbatim}

\comment{
Unfortunately, joins are not yet available :-(
}

Edges have one more important field that we have not yet discussed,
namely the \term{label}. The \term{label} is intended to create a connection
between edges that are inherently related. In our example, we see
that at one day Popeye bought two different things:
spinach and candy cigarettes.
That was at Jan, 22.

These two edges, hence, relate to the same visit at the supermarket.
We could link these edges using a label. The edge would then be created
as, for instance:

\begin{sqlcode}
\begin{lstlisting}
create edge buys (
        origin client,
        dest product,
        label text,
        weight uint,
        weight2 float
);
\end{lstlisting}
\end{sqlcode}

The label text would correspond to a token generated by
the cashpoint when a customer starts the check-out.
We would then insert edges as follows:

\begin{sqlcode}
\begin{lstlisting}
insert into sales (edge, origin, destin, timestamp,
                            label, weight, weight2)
   ('buys', 9000001, 100001, '1929-01-22T08:27:32',
                      'tx-19290122082731', 1, 1.99)
insert into sales (edge, origin, destin, timestamp,
                            label, weight, weight2)
   ('buys', 9000001, 100002, '1929-01-22T08:27:51', 
                      'tx-19290122082731', 1, 2.49)
\end{lstlisting}
\end{sqlcode}

We could now select edges according to this label:

\begin{sqlcode}
\begin{lstlisting}
select destin, timestamp, weight, weight2
  from sales 
 where label = 'tx-19290122082731'
 order by timestamp
\end{lstlisting}
\end{sqlcode}

This query would result in:

\begin{verbatim}
|----------+---------------------+----------+---------|
|  product | timestamp           | quantity | price   |
|----------+---------------------+----------+---------|
|  100001  | 1929-01-22T08:27:32 | 1        | 1.9900  |
|  100002  | 1929-01-22T08:27:51 | 1        | 2.4900  |
|----------+---------------------+----------+---------|
\end{verbatim}

Until here we always used the client \emph{tool}
to perform queries.
That is certainly an important use case.
Much more typical, however, is to develop application code
that needs a client \acronym{api} to connect to the database.

\nowdb\ comes with a native client \acronym{api}
that is available in different languages, among others
C, \CC, \csharp, Go, Python and Lua.
\comment{Only Python is currently available.
There is a low-level \acronym{api} for C,
but that is not for developing applications,
but for developing \acronym{api}s.}
Here, we will have a quick look at the Python \acronym{api}.

The module implementing the Python \nowdb\ \acronym{api}
is called \term{now.py} and must be imported into the client program.
For the Python interpreter to find this module,
it must be in a directory in the \acronym{pythonpath}.
There may be different ideas on how to install python modules.
The \nowdb\ installation will copy all \nowdb-related modules
to one specific folder and add this folder to the
\acronym{pythonpath}. But you also may install
the \nowdb\ Python \acronym{api} using \term{pip}.
Then, everything is handled by the Python environment
and you don't need to care about these things.
For more details on installation, please refer
to chapter \ref{chpt_install}.

Anyway, here is a simple Python program:

\begin{python}
\begin{lstlisting}
import now

with now.Connection("localhost", "55505", None, None) as c:
   with c.execute("use retail") as r:
       if not r.ok():
          print "cannot use retail: %s" % r.details()
          exit(1)

   with c.execute("select count(*), sum(weight), avg(weight) \
                     from sales where edge='buys'") as cur:
       if not cur.ok():
          print "ERROR: %s" % cur.details()
          exit(1)
       for row in cur:
           print "count: %d, sum: %d, avg: %.2f" %
                 (row.field(0), row.field(1), row.field(2))
\end{lstlisting}
\end{python}

The program first creates a \term{Connection}
to a database listening on port 55505 on `localhost'.
It then executes a \term{use} statement on this connection
to indicate the database towards which
the following statements are directed.

The result of a \term{use} statement is a \term{status}.
A status is either \acronym{ok} or an error.
The details of the can be obtained by means of the method
\term{details()}, which returns a string.
Results also support the method \term{code()},
which would return a numeric error code.
It is often useful to know the error code to decide
what to do programmatically (abandon the program,
retry, try something else, \etc)

Results are also resource managers. That means
that they can be used inside a \term{with} statement.
\term{with} assures that all resources (in this case
the result) are freed before the control leaves
the \term{with}-block even if an exception is raised.

The program then executes a query.
This time the execution returns a \term{cursor} (`cur').
The program checks whether the cursor is
in a good state. The statement may have
failed on the server side. The cursor would
then be in a state that is not \acronym{ok}.
In that case, the program prints the error details
and exits with return code 1.

Otherwise, if the result was fine,
it iterates over the cursor
printing for each row the fields 1-3.
Cursors, indeed, are iterators
that allow simple iteration using \term{for}.

Here is an example of an \term{insert} statement
(we assume that the connection, `c', was already established):

\begin{python}
\begin{lstlisting}
with c.execute("insert into edge (edge, origin, product, timestamp) \
                                 ('complains', 9000001, 100002, \
                                  '1929-01-23T08:45:00')") as rep:
       if not rep.ok():
          print "ERROR: %s" % cur.details()
          exit(1)
       else:
          print "rows affected: %d" % rep.affected()
          print "running time : %d" % rep.runTime()
\end{lstlisting}
\end{python}

In the case of data manipulation (\acronym{dml})
or data loading (\acronym{dll}) statements,
the result may be a status (as with all statements),
namely when the statement fails,
or a report.
A report has the methods \term{affected()},
which indicates the number of rows affected by this statement,
\term{runTime()},
which indicates the running time of the statement in microseconds,
and \term{errors()},
which, in the case of \term{load},
indicates the number of rows that resulted in an error.

For more details on the Python client,
please refer to chapter \ref{chpt_pythonclient}.
The other client \acronym{api}s are described
in chapters \ref{chpt_ccpp}, \ref{chpt_goclient} and
\ref{chpt_luaclient}.

\ignore{
- python in the DB
}


\chapter{SQL}\label{chpt_sql}
\section{Outline}
\sql\ is a language to store, manipulate
and query data in a database; traditionally
\sql\ is used with relational databases.
In recent years, however, people have
started to use \sql\ also in other context,
such as \term{graph} and \term{timeseries}
databases and new patterns are evolving
in the language to better address those
data models.

\sql\ consists of statements that,
in their turn, consist of clauses.
A statement is a piece of \sql\ code
that by itself constitutes a meaningful
action in the database. Statements are
distinguished in

\begin{itemize}
\item \acronym{ddl}:
Statements that manipulate entities
in the database that hold or define
data
like tables, types, edges, indices,
functions or procedures.

\item \acronym{dml}:
Statements that manipulate data,
\eg\ \term{insert}, \term{update} and
\term{delete}.

\item \acronym{dll}:
Statements that load large volumes of data into the database
or retrieve large volumes of data from the database.

\item \acronym{dql}:
Statements that read data from the database.

\item Miscellaneous:
Statements that do not fall into any
those categories, in particular
\term{use} and \term{exec}.
\end{itemize}

Clauses are parts of statements;
a \acronym{dql} statement, for instance,
typically has a \term{select} clause and
a \term{from} clause and may have
additional clauses (\term{where},
\term{order by}, \etc).

Some clauses can appear in more than
one type of statement. \term{update}
and \term{delete} statements, typically,
have a \term{where} clause, but no
\term{select} clause.

Clauses can be seen as logical building blocks
of \sql. But they cannot live alone.
It is not possible to execute an isolated \term{where}
clause or an isolated \term{from} clause.
The smallest executable unit is therefore the statement.

Clauses are made of keywords, identifiers, numbers
and text strings. Keywords and identifiers are mutually
exclusive, that is, if $k$ is a keyword,
it cannot be an identifier.
\comment{This rule is relaxed in most
\sql\ dialects -- and that makes a lot of sense,
because \sql\ has an extraordinary large
number of keywords which sometimes makes the choice
of meaningful identifiers a non-trivial task.
At the time of writing, the \nowdb\ parser
does not relax this rule, but it will do so
in the future.}
Keywords are defined by the language,
identifiers are chosen by the user
and refer to entities in the database,
such as tables, types, indices, \etc\

In this specification,
keywords are typeset in boldface
(\eg\ \keyword{table});
identifiers are typeset in italics
(like `mytable' in 
``\keyword{create table} \identifier{mytable}'').

\sql\ is a textual interface.
All statements that are passed to the database
have a textual form. The results produced
by the database, however, are not. They are
binary data which may or may not
contain textual parts.

In \nowdb\ \sql\ statements are strings
of \acronym{utf}-8 characters.
Keywords, identifiers and numbers, however,
must contain only characters
from the \acronym{ascii} subset.
Text, by contrast, may contain any
\acronym{utf}-8 character.
\comment{It is already possible to store
\acronym{utf}-8 in the database.
But comparison and sorting are not yet
\acronym{utf}-compliant. This is an
urgent to-do.}

Keywords and identifiers are case-insensitive.
There is no difference in
\term{SELECT}, \term{select} or \term{Select}
and so on.
Text, by contrast, is case-sensitive;
\term{'hello world'} and \term{hello World}
are not the same!

\sql\ is a \term{guest} language
that needs some kind of framework
to support it. One way to provide this
framework is the \nowdb\ client,
which provides two means to execute
\sql\ \term{statements} in the database,
\ie\ by means of the \tech{-Q} parameter
and by means of standard input.

Another way is a guest language
that provides means to pass \sql\ statements
to the database and means to receive
and interpret the results produced by such statements.

The protocol that defines how data are exchanged
between the database and the host system
is not part of this specification.
Currently, native client and server libraries
exist that implement this protocol
without exposing it to the user.
To support open standards in the future, such as
\acronym{odbc} and \acronym{jdbc},
parts of this protocol must be documented
and published.

\section{Types}
\subsection{Static Types}
The static types constitute the \nowdb\ \sql\ type system.
The static types can be used in \sql\ statements.
The declaration form is used in \acronym{ddl} statements
to define types, edges, procedure and functions.
In \acronym{dml}, \acronym{dll} and \acronym{dql} statements,
instances of the types are used, \ie\
types are not explicitly declared, but used implicitly
by means of their constructors, which are sufficient
to determine the type uniquely.

In the case of numeric types
(integers, unsigned integers and floats),
\nowdb\ silently corrects type mismatches where possible.
An unsigned integer inserted into a field where
a signed integer is expected, is implicitly converted
to an integer; correspondingly a signed integer
is converted to an unsigned integer if possible.
If the unsigned integer is out of range or
the signed integer is negative, the statement
is rejected with a type error.
Likewise, signed or unsigned integers are converted to floats
if necessary (and possible) and a float might be converted
to an integer (or unsigned integer) if it actually
represents an integer.

\begin{minipage}{\textwidth}
\textbf{Integer}\\
Declaration: $int$, $integer$ \\
Values: $-2^{63} \dots 2^{63}-1$ \\
Constructors: $\pm n$, where $n$ is an unsigned integer.\\
Null: $+0$ \\
Examples: $-1, +0, +1$
\end{minipage}

\begin{minipage}{\textwidth}
\textbf{Unsigned Integer} \\
Declaration: $uint$, $uinteger$ \\
Values: $0 \dots 2^{64}-1$  \\
Constructors: One digit from the range $0\dot 9$
or one digit from range $1\dot 9$ followed by
a sequence of digits ($0\dots9$). \\
Null: $0$ \\
Examples: $0, 1, 2, 1024$, but not: $01$.
\end{minipage}

\begin{minipage}{\textwidth}
\textbf{Float} \\
Declaration: $float$ \\
Represents a \term{binary64} \acronym{ieee}-754 floating point number.
For possible values, please refer to the standard or to the table in
\url{https://en.wikipedia.org/wiki/IEEE\_754#Basic\_and\_interchange\_formats}.\\
Constructors: any integer followed by a dot and a sequence of digits,
              optionally followed by $e$ followed by an integer.
              \comment{The exponential form is not yet available.} \\
Null: $0.0$ \\
Examples: $-1.0, 0.0, 1.0, 3.14159, 1.797693e308$ 
\end{minipage}

\begin{minipage}{\textwidth}
\textbf{Time} \\
Declaration: $time$, $date$ \\
Values:  \acronym{utc} 1677-09-21T00:12:44 --
         \acronym{utc} 2262-04-11T23:47:16 \\
Precision: nanosecond. \\
Note, however, that range and precision depend on server configuration.
With less precision, a higher range can be reached.
Please refer to the database configuration guide. \\
Timezone: \acronym{utc} \\
Constructor: any integer or any string following \acronym{iso}-8601 
or any string following a locally defined time format. \\
Null: $0$ \\
\comment{It should be possible to perform arithmetic
with time units and timestamps
in \sql, \eg: \\
\term{where timestamp = today + 7*day}.\\
That is, why a null type makes sense for time.
Currently, arithmetic with time is not possible in \sql.
(It is possible with the date and time types in host languages, though,
and therefore not urgent.)} \\
Examples:\\
1535284617906179393, \\
'1940-12-21', \\
'1904-06-16T11:43:10', \\
'2011-11-11T11:11:11.123456789'
\end{minipage}

\begin{minipage}{\textwidth}
\textbf{Bool} \\
Declaration: $bool$ \\
Values: $true$, $false$ \\
Null: $false$
\end{minipage}

\begin{minipage}{\textwidth}
\textbf{Text} \\
Declaration: $text$ \\
Values: \acronym{utf}-8 string with up to 255 bytes\\
Constructor: string enclosed by ' \\
Null: '' (the empty string),\\
Examples:\\
'hello world',\\
\begin{CJK}{UTF8}{gbsn}
'鎮州臨濟慧照禪師語錄序。' 
\end{CJK} \\
\comment{An important detail is not yet handled:
text that \emph{contains} the character '.
This is important for recursive \sql, \eg\ \\
\term{exec metaquery('select * from myedge 
       where a = $\backslash$'some text$\backslash$'')}}
\end{minipage}

\begin{minipage}{\textwidth}
\textbf{Longtext} \\
Declaration: text \\
Values: \acronym{utf}-8 string with up to 4096 bytes\\
\comment{Longtext is not yet available.}
\end{minipage}

\begin{minipage}{\textwidth}
\textbf{Blob} \\
\comment{Blob is nice to have, but there are currently
no concrete plans to add such a datatype.}
\end{minipage}

\subsection{Dynamic Types}
Dynamic types are not used in \sql\ statements,
but rather describe the return values of \sql\
statements. As such, they live in the context
of a host language (C, Python, Lua, \etc).
There concrete implementation, therefore,
depends on that language and any language
functioning as either client or server-side language
needs to implement these types.
For more information on concrete implementation
of dynamic types, please refer to the
host language \acronym{api} specifications.

\begin{minipage}{\textwidth}
\textbf{Status}\\
Values: \acronym{ok}, \acronym{nok}\\
In the case of error (\acronym{nok}), \nowdb\ provides:
\begin{itemize}
\item an error code
\item a detailed error message
\end{itemize}
Error codes together with a brief description
of their meaning can be found in \ref{chpt_errors}.
\end{minipage}

\begin{minipage}{\textwidth}
\textbf{Report}\\
Reports consist of up to three values:
\begin{itemize}
\item number of affected rows
(returned by all \acronym{dml} and \acronym{dll} statements)
\item number of errors
(returned only by \acronym{dll} statements)
\item running time
(returned by most \acronym{dml} and \acronym{dll} statements)
\end{itemize}
\end{minipage}

\begin{minipage}{\textwidth}
\textbf{Row}\\
A row is the result of a projection;
it consists of an array of values
with type information called \term{fields}.
In the host language, one would access a field
typically by an expression of the form:
$row.field(i);$
which would return a tuple $(value,type)$.

A row result consists of one or many rows.
The host language shall provide means
to iterate over rows.
\end{minipage}

\begin{minipage}{\textwidth}
\textbf{Cursor}\\
A cursor is an iterable collection of rows.
The iteration directive is \term{fetch}.
Each fetch may return one or many rows.
A cursor is a server-side resource
and shall be closed using the directive \term{close}.
\end{minipage}

\section{Data Definition}
\comment{Due to some time pressure in producing this document,
there are no syntax diagrams, which would make things much
clearer. Instead, examples are given. Examples are easier
to understand quickly, but they are much less precise.}

\comment{For almost all \acronym{ddl} subtypes,
\term{alter} is not yet implemented.}

\subsection{Schema}
The keywords
\term{schema}, \term{database} and \term{scope}
are interchangeable.

\subsubsection{CREATE}
The \term{create schema} statement
creates an empty database physically on disk.
It has the following form:

\keyword{create schema} \identifier{mydb}

This would create the directory `mydb'
immediately below the base path (with which
the \nowdb\ daemon was started) and
within that directory all objects necessary
to manage that database.

The following forms are equivalent:

\keyword{create database} \identifier{mydb}\\
\keyword{create scope} \identifier{mydb}

All \keyword{create} clauses can be combined
with the clause \keyword{if not exists}, \eg:

\keyword{create schema} \identifier{mydb} \keyword{if not exists}

The \keyword{if not exists}-clause
suppresses the `duplicate key' error
in case the schema
already exists.
It is a convenient way to avoid
that a \sql\ script is abandoned
in such a situation.

\subsubsection{DROP}
The \term{drop schema} statement
removes a database physically from disk.
It has one of the following forms,
which are all equivalent:

\keyword{drop schema} \identifier{mydb}\\
\keyword{drop database} \identifier{mydb}\\
\keyword{drop scope} \identifier{mydb}

The statement removes all objects and data
belonging to the database `mydb' and
the directory `mydb' from the disk.

All \keyword{drop} clauses can be combined
with the clause \keyword{if exists}, \eg:

\keyword{drop schema} \identifier{mydb} \keyword{if exists}

The \keyword{if exists}-clause
suppresses the `key not found' error
in case the schema
does not exist.
It is a convenient way to avoid
that a \sql\ script is abandoned
in such a situation.

\subsubsection{ALTER}

\subsection{Table}
\subsubsection{CREATE}
The \term{create table} statement
creates a new storage entity for edges
physically on disk.

The simplest form is:

\keyword{create table} \identifier{mytable}

The keyword table may be decorated
with a sizing option:

\keyword{create big table} \identifier{mytable}

Valid sizing keywords are:
\keyword{tiny, small, medium, big, large, huge}.

Sizing keywords affect the allocation units
of disk space. The concrete meaning is not
part of this specification and may change
in the future.

It is also possible to add options to
to a \term{create table} statement.
Options have the general form

\term{\keyword{set} option = value, option = value}.

Valid options and values are listed in the following table:

\bgroup
\renewcommand{\arraystretch}{1.3}
\begin{center}
\begin{tabular}{||c||c||c||c||}\hline
Option & Values & Meaning & Default \\\hline\hline
\keyword{stress} & \keyword{moderate} & Low ingestion volume with occasional peaks & X \\\cline{2-4}
                 & \keyword{constant} & Constant ingestion of high volume          &   \\\cline{2-4}
                 & \keyword{insane} & Constant ingestion of very high volume       &   \\\hline\hline
\keyword{disk} & \keyword{hdd}  & Disk space is allocated in large chunks          & X \\\cline{2-4}
               & \keyword{ssd}  & Disk space is allocated in small chunks          &   \\\cline{2-4}
               & \keyword{raid} & Currently, no effect                             &   \\\hline\hline
\keyword{compression} & 'zstd'  & zstd is used for compression                     & X \\\cline{2-4}
                      & 'lz4'   & lz4 is used for compression (\comment{not available}) &   \\\cline{2-4}
                      & ''      & Data in this table are not compressed at all     &   \\\cline{1-4}
\end{tabular}
\end{center}
\egroup

The option \keyword{stress} affects the number of threads
allocated to perform ingestion tasks
like compression, sorting and indexing.
How many threads are allocated
is not part of this specification
and may vary between platforms.

The user may decide on the compression algorithm to use.
The standard compression algorithm is \term{zstd},
which is fast, but has also very good compression ratio.
It is recommended to use \term{zstd} in most cases.
\term{lz4} (\comment{not yet available}) is faster than \term{zstd},
in particular on decompression,
but has a weaker compression ratio.
Finally, no compression at all (empty string)
makes sense on small tables
that are known never to grow beyond some megabyte in size
(or beyond some million edges).

An example of a \term{create table} statement with options is

\keyword{create table} \identifier{mytable}
\keyword{set} \keyword{stress} = \keyword{constant},
              \keyword{compression} = 'zstd'

\subsubsection{DROP}
The \term{drop table} statement removes
an existing storage entity for edges
physically from disk.
It has the form:

\keyword{drop table} \identifier{mytable}

\subsubsection{ALTER}

\subsection{Type}


\subsection{Edge}
\subsection{Index}
\subsection{Procedure}
\subsection{Function}
\subsection{Period}

\section{Data Manipulation}
\subsection{Insert}
\subsection{Update}
\subsection{Delete}

\section{Data Loading}

\section{Data Query}

\section{Miscellaneous}
\subsection{Use}
\subsection{Exec}


\chapter{Data Modelling}\label{chpt_model}
\comment{
stress differences between graph and relational\\
what is good design / best practice for both:
timeseries and graph \\
examples:
wmo (pure timeseries),
retail (timeseries + graph)
}

\chapter{The NoWDB daemon}\label{chpt_nowdbd}
\comment{tbc}

\chapter{The Client Tool}\label{chpt_clienttool}
\comment{tbc}

\chapter{Higher Level C and C++ Client}\label{chpt_ccpp}
\comment{tbd}

\chapter{Python Client}\label{chpt_pythonclient}
\section{Outline}
The python client consists of a bunch
of modules all starting with prefix \term{now}.
The main module which defines
the \term{Connection} and the \term{Result} class
is actually called just \tech{now.py}.
Support functions (which are also available)
for server-side python are defined in
\tech{nowutil.py}.

The modules use the package \term{dateutil}
which must be installed on the system in order
to use the \term{now} modules.
(Please refer to chapter
\ref{chpt_install} for details.)

A client program needs to import at least
the \term{now} module. The import may have
any form (\ie\ \term{import now} or
\term{from now import}).

\section{Connections}
The \term{Connection} class represents
a \acronym{tcp/ip} connection to the
database. It provides the following 
methods:
\subsubsection{\_\_init\_\_}
The constructor takes four arguments,
which are all strings:
\begin{itemize}
\item Address:
used to determine the address of the server.
The parameter is passed to the system service
\tech{getaddrinfo} and allows:
an \acronym{ip}v4 address,
an \acronym{ip}v6 address or
a hostname.
Examples: ``127.0.0.1'', ``localhost'',
``myserver.mydomain.org''.

\item Port:
used to determine the port of the server.
The parameter is passed to the system service
\tech{getaddrinfo} and allows:
a port number or a known service name.

\item User:
\comment{currently not used}
\item Password:
\comment{currently not used}
\end{itemize}

\subsubsection{execute(statement)}
The method sends the string \term{statement} to the database
and waits for the database response.
The method retuns an instance of the \term{Result} class
or, on internal errors, raises one of the exceptions
\term{ClientError} or \term{DBError}.

\subsubsection{close()}
The method closes the connection and
releases the C objects allocated by
the constructor. If the connection
cannot be closed for any reason,
the exception \term{ClientError} is raised.

\subsubsection{Resource Manager}
\term{Connection} is a \term{resource manager}.
The method \term{close} is therefore
rarely necessary. Instead, \term{Connection} can
be used with the \term{with} statement, \ie:

\begin{python}
\begin{lstlisting}
with Connection("localhost", "55505", None, None) as conn:
    # here goes your code
    # refer to the connect as 'conn'
\end{lstlisting}
\end{python}

\section{Results}
\subsection{Status}
\subsection{Cursors}
\subsection{Rows}
\subsection{Reports}
\section{Support Functions}
\section{Exceptions}
\subsubsection{ClientError}
\subsubsection{DBError}
\subsubsection{WrongType}


\chapter{Go Client}\label{chpt_goclient}
\comment{tbd}

\chapter{Lua Client}\label{chpt_luaclient}
\comment{tbd}

\chapter{The Low-Level C Client}\label{chpt_llc}
\section{Outline}
The low-level C \acronym{api} is not intended
for application development. For this purpose
high-level C and a \CC\ \acronym{api}s exist
(please refer to \comment{document xxx}).
This \acronym{api} aims instead to ease
the development of client language bindings.
The Python, Lua and Go client \acronym{api}s
are built on it.

The \acronym{api} consists of the \tech{libnowdbclient}
library, which is installed in a canonical library directory
such as \tech{/usr/local/lib},
and the \tech{nowclient.h} interface,
which is intalled in a canonical header directory
such as \tech{/usr/local/include}.

The \acronym{api} provides services for
time conversion, connection management,
result handling and error handling.

\section{Time}
The \acronym{api} defines the \nowdb\ time type,
\tech{nowdb\_time\_t} as \tech{uint64\_t}.
Values of this type represent a \acronym{unix}
timestamp with nanosecond precision.
The smallest and greatest possible values
are defined as 
\begin{itemize}
\item \acronym{nowdb\_time\_dawn}
(`1677-09-21T00:12:44')
and
\item \acronym{nowdb\_time\_dusk}
(`2262-04-11T23:47:16')
respectively.
\end{itemize}

The standard formats are defined as
\begin{itemize}
\item \acronym{nowdb\_time\_format}
\tech{"\%Y-\%m-\%dT\%H:\%M:\%S"} and
\item \acronym{nowdb\_date\_format}
\tech{"\%Y-\%m-\%d"}. 
\end{itemize}

The following time conversion functions are available:

\subsubsection{nowdb\_time\_fromUnix}
The function receives a \acronym{posix} \tech{struct timespec}
and returns a \tech{nowdb\_time\_t}. The conversion never fails.

\subsubsection{nowdb\_time\_toUnix}
The function receives a \tech{nowdb\_time\_t} 
and a pointer to a \acronym{posix} \tech{struct timespec},
which must not be \acronym{null}.
The function returns an \tech{int} representing
an error code (see chapter \ref{chpt_errors}).
The conversion fails if the pointer to the \term{timespec}
structure is \acronym{null} or when the
\tech{nowdb\_time\_t} value is out of range
(which can happen only with custom time configurations).

\subsubsection{nowdb\_time\_parse}
The function receives
\begin{itemize}
\item a time string (which must not be \acronym{null}),
\item a format string (which must not be \acronym{null}) and
\item a pointer to a \tech{nowdb\_time\_t}
(which must not be \acronym{null}).
\end{itemize}

The function returns an \tech{int} representing
an error code (see chapter \ref{chpt_errors}).

The function attempts to parse the time string
according to the format string and, if successful,
stores the result at the address passed in as third parameter.

\subsubsection{nowdb\_time\_show}
The function receives
\begin{itemize}
\item a \tech{nowdb\_time\_t}
\item a format string (which must not be \acronym{null}),
\item a \tech{char} buffer (which must not be \acronym{null}) and
\item a \tech{size\_t} indicating the size of the buffer.
(which must not be \acronym{null}).
\end{itemize}

The function returns an \tech{int} representing
an error code (see chapter \ref{chpt_errors}).

The function writes a string representation
of the \tech{nowdb\_time\_t} value
according to the format string
into the \tech{char} buffer.
The function fails (among other reasons),
if the buffer is not big enough to hold the result.

\subsubsection{nowdb\_time\_get}
The function expects no argument and returns a \tech{nowdb\_time\_t}
representing the current time. On failure,
the function returns \acronym{nowdb\_time\_dawn}
(which is certainly not \emph{now}).

\section{Connection}
The type \tech{nowdb\_con\_t} represents a
\acronym{tcp/ip} connection to the database.
The type is defined as pointer to an
\term{anonymous struct} and cannot be
allocated by the user.

Note that connections are not \term{threadsafe}.
Threads must not access connection objects
concurrently.

For connection management the following services
are available:

\subsubsection{nowdb\_connect}
The function receives six arguments:
\begin{itemize}
\item a pointer to a \tech{nowdb\_con\_t}
(which must not be \acronym{null}),
\item a string
(which must not be \acronym{null})
defining the \term{node}
on which the database is running and which
may be an \acronym{ip} \tech{v4} or \tech{v6} address or
a hostname.
\item a string
(which must not be \acronym{null})
defining the service which
is usually the port to which the database is listening,
\item a string
(which may be \acronym{null})
defining the user (\comment{currently not used}),
\item a string
(which may be \acronym{null})
defining the password of that user (\comment{currently not used}) and
\item an \tech{int} representing connection flags.
\end{itemize}

The function returns an \tech{int} representing
a client error code (see section \ref{sec_clnterrors}).

The function attempts to connect to the database server.
On success (when the return value is \acronym{ok}),
the \tech{nowdb\_con\_t} pointer is guaranteed
to point to a valid connection object.
Otherwise, no memory is allocated.

The flags represent connection options.
Currently, the following options are available:
\begin{itemize}
\item \acronym{nowdb\_flags\_text}:
results for this session will be sent
in a textual (\acronym{csv}) format.
\comment{Not available}

\item \acronym{nowdb\_flags\_le}:
results for this session will be sent
in binary \term{little endian} format.

\item \acronym{nowdb\_flags\_be}:
results for this session will be sent
in binary \term{big endian} format.
\comment{Not available}
\end{itemize}

\subsubsection{nowdb\_connection\_close}
The function receives a \tech{nowdb\_con\_t}
and returns an \tech{int} representing
a client error code (see section \ref{sec_clnterrors}).

The function terminates the connection.
On success, it is guaranteed that all
resources have been freed.
Otherwise, when an error code is returned,
resources are still allocated and must be
freed using \tech{nowdb\_connection\_destroy}.

\subsubsection{nowdb\_connection\_destroy}
The function receives a \tech{nowdb\_con\_t}.
The function frees all resources allocated to this connection.
The function never fails and
is declared as \tech{void}.

\section{Result}
The \acronym{api} implements the four
\nowdb\ dynamic types:

\begin{itemize}
\item \acronym{nowdb\_result\_status}
\item \acronym{nowdb\_result\_report}
\item \acronym{nowdb\_result\_cursor}
\item \acronym{nowdb\_result\_row}
\end{itemize}

Three anonymous structs are defined to
represent these types:
\begin{itemize}
\item \tech{nowdb\_result\_status} represents
\begin{itemize}
\item\acronym{nowdb\_result\_status} and
\item\acronym{nowdb\_result\_report},
\end{itemize}
\ie\ the services defined for \term{status}
also accept an object of type \term{report}
and \textit{vice versa};
\item \tech{nowdb\_result\_cursor} 
represents \acronym{nowdb\_result\_cursor};
\item \tech{nowdb\_result\_row} 
represents \acronym{nowdb\_result\_row}
\end{itemize}

\subsection{Execution}
\subsubsection{nowdb\_exec\_statement}
The function receives
\begin{itemize}
\item a connection
\item a string representing an \sql\ statement
\item a pointer to a result object
\end{itemize}

None of those must be \acronym{null}.

The function returns an \tech{int} representing
a client error code.

The function sends the \sql\ statement
to the database and waits for the result.
On success,
the \tech{nowdb\_result\_t} pointer is guaranteed
to point to a valid result object.
Otherwise, no additional memory is allocated.

\subsubsection{nowdb\_exec\_statementZC}
This function is a \term{zerocopy} variant
of \tech{nowdb\_exec\_statement}.
\tech{nowdb\-\_\-exec\-\_\-statement} copies all
data received from the database to a private
buffer that belongs to the result object.
This allows user programs to process
interleaving queries, \ie\
statements can be executed,
while others still have pending results.

In some cases, this is not necessary,
in particular when the result is just
a status that is checked once immediately
after \tech{exec} has returned.
In such cases
\tech{nowdb\-\_exec\-\_statementZC}
might be more efficient, because
it does not copy the result data into a private buffer,
but leaves them in the connection object.
The next query, hence, will overwrite
those data.

Note that \tech{nowdb\_exec\_statementZC}
is not allowed when the result
may be a cursor.

\subsection{Status and Report}
\subsubsection{nowdb\_result\_errcode}
The function receives a \tech{nowdb\_result\_t}
and returns its error code which is a server-side
error code (or 0 for success).

\subsubsection{nowdb\_result\_eof}
The function receives a \tech{nowdb\_result\_t}
and returns an \tech{int} which is $\neq 0$
if the error code is \term{end-of-file}
and 0 otherwise.

\subsubsection{nowdb\_result\_details}
The function receives a \tech{nowdb\_result\_t}
and returns a constant string providing details
on the error represented by the status.
If no error has occurred, the result string
is \tech{"OK"}.

\subsubsection{nowdb\_result\_report}
The function is declared as \tech{void}.
The function receives a \tech{nowdb\_result\_t}
and three more parameters:
\begin{itemize}
\item a pointer to a \tech{uint64\_t}
which must not be \acronym{null}
and to which the number of affected rows
is written; 
\item a pointer to a \tech{uint64\_t}
which must not be \acronym{null}
and to which the number of errors
is written; 
\item a pointer to a \tech{uint64\_t}
which must not be \acronym{null}
and to which the running time in microseconds
is written.
\end{itemize}

\subsubsection{nowdb\_result\_destroy}
The function is declared as \tech{void}.
The function receives a \tech{nowdb\_result\_t}
and frees all resources assigned to it.

\subsection{Cursor}
\subsubsection{nowdb\_cursor\_open}
The function receives a \tech{nowdb\_result\_t}
and a pointer to a \tech{nowdb\_cursor\_t}
(which must not be \acronym{null}).
It returns an \tech{int} representing
a client error code.
The function \emph{casts} the result
passed in to the pointer address.
The function fails if the result
does not represent a valid result
or if that result was created
with \term{zerocopy} option.
Note that \tech{nowdb\_exec\_statementZC}
is not allowed when the result
may be a cursor.

\subsubsection{nowdb\_cursor\_errcode}
The function receives a \tech{nowdb\_cursor\_t}
and returns its error code.

\subsubsection{nowdb\_cursor\_details}
The function receives a \tech{nowdb\_cursor\_t}
and returns a constant string providing error details.
If no error has occurred, the result string
is \tech{"OK"}.

\subsubsection{nowdb\_cursor\_eof}
The function receives a \tech{nowdb\_cursor\_t}
and returns an \tech{int} which is $\neq 0$
if the error code is \term{end-of-file}
and 0 otherwise.

\subsubsection{nowdb\_cursor\_ok}
The function receives a \tech{nowdb\_cursor\_t}
and returns an \tech{int} which is $\neq 0$
if the error code is \acronym{ok}
and 0 otherwise.

\subsubsection{nowdb\_cursor\_id}
The function receives a \tech{nowdb\_cursor\_t}
and returns a \tech{uint64\_t} which
represents the identifier under which this
cursor is known in the server.

\subsubsection{nowdb\_cursor\_fetch}
The function receives a \tech{nowdb\_cursor\_t}
and returns an \tech{int} which
represents a client error code.
On success, the cursor fetches the next
bulk of rows from the server.
If there are no more rows in the server,
the error code of the cursor passes to \term{end-of-file}.

\subsubsection{nowdb\_cursor\_row}
The function receives a \tech{nowdb\_cursor\_t}
and returns a \tech{nowdb\_row\_t}.
Note that the result is not a copy
of the rows in the cursor, but a reference
to those.
It is therefore not necessary to destroy
the returned rows.
On the other hand, the rows are lost
when either \tech{fetch} or \tech{close}
is called on the cursor. 

\subsubsection{nowdb\_cursor\_close}
The function receives a \tech{nowdb\_cursor\_t}
and returns an \tech{int} which represents an
error code.
It sends a close request for this cursor
to the database and, on success,
frees all resources assigned to this cursor.
Otherwise, if the close request fails,
the cursor must be destroyed using 
\tech{nowdb\_result\_destroy} and
casting the cursor to a \tech{nowdb\_result\_t}

\subsection{Row}
\subsubsection{nowdb\_row\_next}
The function receives a \tech{nowdb\_row\_t}
and returns an \tech{int} that represents
a client error code.
It advances to the next row.
If the current row was already the last one,
\term{end-of-file} is returned.

\subsubsection{nowdb\_row\_rewind}
The function is declared as \tech{void}.
The function receives a \tech{nowdb\_row\_t}.
It resets the row struct to the first row.

\subsubsection{nowdb\_row\_field}
The function receives a \tech{nowdb\_row\_t}
and two more parameters, namely
\begin{itemize}
\item an \tech{int}, $n$, indicating
that we want to obtain the $n$th field
starting to count from 0
for the first field in the row;
\item a pointer to an \tech{int}
which must not be \acronym{null}
and is set to the type of the field.
\end{itemize}
The function returns the address
of the first byte of the $n$th field
or \acronym{null} on error.

Valid types are
\begin{itemize}
\item \acronym{nowdb\_typ\_uint}
\item \acronym{nowdb\_typ\_int}
\item \acronym{nowdb\_typ\_float}
\item \acronym{nowdb\_typ\_time}
\item \acronym{nowdb\_typ\_text}
\item \acronym{nowdb\_typ\_bool}
\end{itemize}

\subsubsection{nowdb\_row\_copy}
The function receives and returns
a \tech{nowdb\_row\_t} and
copies the row passed in allocating
new memory. Rows created with copy
must be destroyed using \tech{nowdb\_result\_destroy}
casting the row to the generic result type.

\subsubsection{nowdb\_row\_write}
The function receives a \acronym{file} pointer and
a \tech{nowdb\_row\_t}.
It returns an \tech{int} representing
a client error code.
The function writes a textual representation
of the row(s) passed in to the file.

\section{Error Handling}
\subsubsection{nowdb\_err\_explain}
The function receives a (client or server) error code
and returns a constant string explaining the error.

\begin{minipage}{\textwidth}
\section{Error Codes}\label{sec_clnterrors}
\bgroup
\renewcommand{\arraystretch}{1.3}
\begin{center}
\begin{longtable}{||l||c||l||}\hline
\textbf{Error Name} & \textbf{Numerical Code} & \textbf{Meaning} \\\hline\endhead\hline
out of memory                             &     -1 & \\\hline\hline
no connection                             &     -2 & \\\hline\hline
socket error                              &     -3 & \\\hline\hline
error on address                          &     -4 & \\\hline\hline
cannot create result                      &     -5 & \\\hline\hline
invalid parameter                         &     -6 & \\\hline\hline
error on read operation                   &   -101 & \\\hline\hline
error on write operation                  &   -102 & \\\hline\hline
error on open  operation                  &   -103 & \\\hline\hline
error on close operation                  &   -104 & \\\hline\hline
use statement failed                      &   -105 & \\\hline\hline
protocol error                            &   -106 & \\\hline\hline
statement or requested resource too big   &   -107 & \\\hline\hline
operating system error (check errno)      &   -108 & \\\hline\hline
time or date format error                 &   -109 & \\\hline\hline
cursor with zerocopy requested            &   -110 & \\\hline\hline
cannot close cursor                       &   -111 & \\\hline
\end{longtable}
\end{center}
\egroup
\end{minipage}


\chapter{Embedded Python}\label{chpt_pythonemb}
\section{Outline}
The embedded python library consists mainly
of the \term{nowdb} module, which defines
the \term{execute} function and the \term{Result} class.
Support functions (which are also available
for client-side python) are defined in
\tech{nowutil.py}.

The module uses the package \term{dateutil}
which must be installed on the system in order
to use the \term{nowdb} module.
(Please refer to chapter
\ref{chpt_install} for details.)

An embedded Python program needs to import at least
the \term{nowdb} module. The import must have
the form \term{import nowdb}, because the database
interacts with this module through the user-defined
module.

The user may define a module-level function
\term{cleanup()} in his or her module.
This function is called by the database
before the module is unloaded at the end of
the session. A meaningful use case for this
function is provided below in section \ref{sec_func}.

The embedded \acronym{api} is very similar
to that provided by the client library,
but there are also some differences.
The most obvious difference is that
there is no \term{Connection} class.
Instead, there is a module-level function
to execute \sql\ statements.

There are also model-level functions
to create results. In the client library
this is not necessary, because clients
do not need to return results.
Likewise, the \term{Result} class
has some more methods to deal with
the specifics of creating results
and returning them to the database.

\comment{
Currently, we only discuss procedures --
in the future, there will also be functions
(which can be called in \sql\ context)
and those will be discussed here as well.
}

\section{Module-level Functions}

\subsubsection{execute(statement)}
The function calls the parser against the statement and, on success,
executes the parsing result in the database.
The function returns an instance of the \term{Result} class
or, on internal errors, raises the exception \term{DBError}.

\subsubsection{success()}
The function creates a \term{Status} result
that represents \acronym{ok}.

\subsubsection{makeError(code, msg)}
The function creates a \term{Status} result
that represents an error event with error code
\term{code} and detailed information \term{msg}.

\subsubsection{makeRow()}
The function creates a \term{Row} result.
The row is initially empty.

\subsubsection{convert(typ, value)}
The function converts \term{value} according
to the \nowdb\ \sql\ type \term{typ}.
Available types are
\acronym{text}, \acronym{time},
\acronym{float}, \acronym{int}, \acronym{uint} and
\acronym{bool},
which are defined as global constants.

\section{Results}
The \term{Result} class represents
the dynamic types described in chapter \ref{chpt_sql}.
Instances of \term{Result} are
created and returned by
the \term{execute()} function
and by the \term{Result} creators discussed above.

The \term{Result} class has the following methods:

\subsubsection{rType()}
The method returns the result type,
either \acronym{status},
\acronym{report}, \acronym{row} or \acronym{cursor}.

\subsubsection{ok()}
The method returns \term{True}
if the instance does not represent
an error and \term{False} otherwise.

\subsubsection{toDB()}
The method returns a \term{void} pointer
to the underlying C structure.
The database expects such a pointer
as result from a procedure call and
wouldn't know what to do
with a Python result instance.
Any result returned to the database
must therefore call \term{toDB()} 
and return the result of that method.
The following example would return
the result of a statement as is:

\begin{python}
\begin{lstlisting}
with execute("select count(*) from sales") as cur:
    return cur.toDB()
\end{lstlisting}
\end{python}

\subsubsection{release()}
The method releases the C objects
allocated with the \term{Result} object.
Since \term{Result} is a resource manager,
it is rarely necessary to use it explicitly.

\subsubsection{Resource Manager}
Result is a resource manager.
\term{Result} can therefore
be used with the \term{with} statement, \eg:

\begin{python}
\begin{lstlisting}
with execute("select count(*) from sales") as cur:
    # here goes your code
    # 'cur' is the result and
    # if no error has occurred, cur is a cursor
\end{lstlisting}
\end{python}

\subsection{Status}
If \term{Result} is a \term{Status},
two more methods are available:

\subsubsection{code()}
The method returns the \nowdb\ error code.
The error code may be 0,
which means that the call was successful,
no error has occurred;
or it may be one of the error codes listed in
chapter \ref{chpt_errors}.

\subsubsection{details()}
The method returns
detailed information about the error
(or \term{None} if no error has occurred).

\subsection{Cursors}
If the result is a cursor,
four more methods are available:

\subsubsection{fetch()}
The method fetches the next bulk
of rows from the database.
After successful completion,
the cursor contains this bulk
of rows, which can be obtained by means of
the method \term{row()}.
Note that the first bulk of rows
is available immediately after
Cursor creation.

\subsubsection{row()}
The method obtains the current
bulk of rows from the cursor.
It returns a \term{Row} result.

\subsubsection{close()}
The method closes the cursor.
Cursors need to be closed
explicitly; the method, however,
is usually not called directly,
but implicitly, when the result
is created by means of a \term{with} statement.

\subsubsection{eof()}
The method returns \term{True}
if the error state of the cursor
is \term{end-of-file} and \term{False}
otherwise.

\subsubsection{Iterator}
\term{Cursor} is an iterator.
Usually, \term{fetch()}, \term{row()}
and \term{eof()} do not need to be called
explicitly.
Instead a \term{for}-loop  can be used, \eg:

\begin{python}
\begin{lstlisting}
with execute("select * from sales") as cur:
    if not cur.ok():
        print "ERROR: %s " % cur.details()
        return
    for row in cur:
        # process cursor
        # row holds a single row
\end{lstlisting}
\end{python}

\subsection{Rows}
If the result is a row,
four more methods are available:

\subsubsection{field(n)}
The method returns the value
of the $n$th field (starting to count
from 0 for the first field).
The value is an \sql\ base type
converted to Python.
Conversion takes place in the obvious way
(integer and unsigned integer to int,
 float to float, text to string, \etc).
An exception are \term{time} fields.
One could expect \term{field} to return
a \term{datetime}, but that is not the case.
Time is returned as a (signed) integer
representing a \acronym{unix} timestamp
with nanosecond precision.
It can be converted with \term{now2dt}
(please refer to chapter \ref{chpt_pythonclient}
for details).

\subsubsection{nextRow()}
The method advances to the next row.
If there was one more row, the method
returns \term{True} and the next call
to \term{field(n)} will return the
$n$th field of that row.
Otherwise, if no more rows are available,
the method returns \term{False}.

Note that, working with a cursor
as iterator, it is not necessary
to use this method. The iterator
produces single rows.
The main use case of \term{nextRow()}
is for user-defined procedures that
return bulks of rows. An example
for this usage is:

\begin{python}
\begin{lstlisting}
with execute("exec myproc()") as row:
    if not row.ok():
        print "ERROR: %s " % row.details()
        return
    while True:
        # use row.field(n) to access fields
        if not row.nextRow():
            break
\end{lstlisting}
\end{python}

\subsubsection{add2Row(typ, value)}
The method adds \term{value}
as \nowdb\ \sql\ type \term{typ} to
the row. This method is intended
for rows that are created in user code.
A usage example is:

\begin{python}
\begin{lstlisting}
with execute("select origin from edge where edge = 'buys'") as cur:
    our = None
    for row in cur:
       if our is None:
          our = makeRow() # create a row result
       else:
          our = closeRow() # terminate row

       with execute("select client_name from client\
                      where client_key = " +
                      str(row.field(0))) as cur2:
            for cow in cur2:
                our.add2Row(UINT, cur.field(0))
                our.add2Row(TEXT, cow.field(0))

    return our.toDB() 
\end{lstlisting}
\end{python}

This is a na\"ive join implementation:
For each edge, the function executes a \term{select}
on the \term{client} with the key found in \term{origin}.
The function creates a row ``our'' which is initialised
to \term{None}. Once it was created, it is closed,
so that each iteration will write to a new row.
Each of the rows contains two fields: the \term{origin}
from edge and the \term{client\_key} from \term{client}.
Finally, the row is returned (calling $toDB()$).

\subsubsection{closeRow()}
The method inserts an \term{end-of-row} marker into the row.
It is intended to be used with bulks of rows created by
user-defined procedures. Note that it is not necessary
to close a single row result. The marker separates
rows and is therefore needed only for bulks.

\subsection{Reports}
If the result is a report,
three more methods are available:

\subsubsection{affected()}
The method returns the number of affected rows.

\subsubsection{errors()}
The method returns the number of errors.
Note that \acronym{dml} statements will
return a \term{Status} result if an error occurred.
In practice, only \acronym{dll} statements provide
this information.

\subsubsection{runTime()}
The method returns the running time of 
the \acronym{dml} or \acronym{dll} statement.

A usage example is

\begin{python}
\begin{lstlisting}
with execute("load '/opt/import/client.csv into client") as rep:
    if not rep.ok():
        print "ERROR: %s " % rep.details()
        return
    
    print "affected: %d, errors: %d, running time: %dus" % 
             (rep.affected(), rep.errors(), rep.runTime())
\end{lstlisting}
\end{python}

\section{Exceptions and Errors}
\subsubsection{DBError}
Is raised when an error in the database occurred.

\subsubsection{WrongType}
Is raised in case of type mismatch, \ie\
trying to call a method not available for
this specific return type.

\subsubsection{explain(err)}
The function \term{explain} returns
a description of the error code passed in.

\section{Functional Cursor Library}\label{sec_func}
\comment{tbc}


\chapter{Embedded Lua}\label{chpt_luaemb}
\comment{tbd}

\chapter{Detailed Installation Guide}\label{chpt_install}
\comment{tbd}

\chapter{Defining Data Loaders}\label{chpt_loader}
\comment{tbd}

\chapter{Pub/Sub and Filters}\label{chpt_pubsub}
\comment{tbd}

\chapter{Optimising Queries}\label{chpt_opt}
\section{Terminology}

\section{Fullscan}

\section{Searching}
Index vs. no index,
role of periods 

\section{Grouping and Ordering}
Explain: FRANGE, KRANGE and CRANGE,
grouping and ordering without indices

\section{Joining}


\chapter{Table and Index Sizing}\label{chpt_sizing}
\comment{tbc}

\chapter{Error Codes}\label{chpt_errors}
\bgroup
\renewcommand{\arraystretch}{1.3}
\begin{center}
\begin{longtable}{||l||c||l||}\hline
\textbf{Error Name} & \textbf{Numerical Code} & \textbf{Meaning} \\\hline\endhead\hline

out of memory                   &      1 & \\\hline\hline
invalid parameter or value      &      2 & \\\hline\hline
no resource available           &      3 & \\\hline\hline
resource is busy                &      4 & \\\hline\hline
requested resource too big      &      5 & \\\hline\hline
locking error                   &      6 & \\\hline\hline
unlock error                    &      7 & \\\hline\hline
end of file                     &      8 & \\\hline\hline
feature not supported           &      9 & \\\hline\hline
bad path                        &     10 & \\\hline\hline
bad name                        &     11 & \\\hline\hline
error in map operation          &     12 & \\\hline\hline
error in unmap operation        &     13 & \\\hline\hline
error in read operation         &     14 & \\\hline\hline
error in write operation        &     15 & \\\hline\hline
error in open operation         &     16 & \\\hline\hline
error in close operation        &     17 & \\\hline\hline
error in remove operation       &     18 & \\\hline\hline
error in seek operation         &     19 & \\\hline\hline
internal error (panic)          &     20 & \\\hline\hline
error reading data catalog      &     21 & \\\hline\hline
error in time operation         &     22 & \\\hline\hline
key not found                   &     26 & \\\hline\hline
duplicated key                  &     27 & \\\hline\hline
duplicated name                 &     28 & \\\hline\hline
error in sync operation         &     30 & \\\hline\hline
error in pthread operation      &     31 & \\\hline\hline
error in sleep operation        &     32 & \\\hline\hline
error in dequeue operation      &     33 & \\\hline\hline
error in enqueue operation      &     34 & \\\hline\hline
worker thread failed            &     35 & \\\hline\hline
timeout                         &     36 & \\\hline\hline
bad block                       &     38 & \\\hline\hline
bad filesize                    &     39 & \\\hline\hline
cannot set max files            &     40 & \\\hline\hline
error in move operation         &     41 & \\\hline\hline
index-related error             &     42 & \\\hline\hline
wrong or unknown version        &     43 & \\\hline\hline
error in compression            &     44 & \\\hline\hline
error in decompression          &     45 & \\\hline\hline
error in compression dictonary  &     46 & \\\hline\hline
error in data store             &     47 & \\\hline\hline
error on table level            &     48 & \\\hline\hline
error on database level         &     49 & \\\hline\hline
error in stat operation         &     50 & \\\hline\hline
error in create operation       &     51 & \\\hline\hline
error in drop operation         &     52 & \\\hline\hline
wrong or unknown magic number   &     53 & \\\hline\hline
error in loader                 &     54 & \\\hline\hline
error in trunc operation        &     55 & \\\hline\hline
error in flush operation        &     56 & \\\hline\hline
error in beet library operation &     57 & \\\hline\hline
error in aggregate function     &     58 & \\\hline\hline
resource not found              &     59 & \\\hline\hline
parser error                    &     60 & \\\hline\hline
error waiting on signal         &     61 & \\\hline\hline
error in signal operation       &     62 & \\\hline\hline
error setting signal set        &     63 & \\\hline\hline
protocol error                  &     64 & \\\hline\hline
cannot create socket            &     65 & \\\hline\hline
error in bind operation         &     66 & \\\hline\hline
error in listen operation       &     67 & \\\hline\hline
cannot accept                   &     68 & \\\hline\hline
server error                    &     69 & \\\hline\hline
cannot find or use address      &     70 & \\\hline\hline
python interpreter error        &     71 & \\\hline\hline
unknown symbol                  &     72 & \\\hline\hline
user error                      &     73 & \\\hline\hline
unknown error                   &   9999 & \\\hline\hline
\end{longtable}
\end{center}
\egroup


\end{document}
