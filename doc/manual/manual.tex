%% =======================================================
%% (c) 2018 Tobias Schoofs
%% =======================================================
%% NowDB
%% =======================================================

% Plain Style
\documentclass{scrreprt}

%% =======================================================
%% (c) Tobias Schoofs
%% =======================================================
%% Commands 4 Programmers
%% =======================================================

%include lhs2TeX.fmt
%include lhs2TeX.sty

%\usepackage[pdftex]{graphicx}
%\usepackage{ucs}
%\usepackage[utf8x]{inputenc} 
\usepackage{tabto}
\usepackage[russian,portuguese,german,english]{babel}
\usepackage{CJK}
\usepackage{amsfonts}
\usepackage{amsfonts}

\usepackage{amsmath}
\usepackage{amssymb}
\usepackage{amsthm}
\usepackage{amscd}

\usepackage{siunitx}

\usepackage{listings}
\usepackage{longtable}

\usepackage{tikz}
\usepackage{pgfplots}

\usepackage{relsize}
\usepackage{xcolor}

\usepackage{soul}

\long\def\ignore#1{}

\newcommand{\acronym}[1]{\textsc{#1}}

\newcommand{\term}[1]{\textit{#1}}
\newcommand{\tech}[1]{{\ttfamily #1}}
\newcommand{\latin}[1]{\textit{#1}}
\newcommand{\speech}[1]{\textit{#1}}

\newcommand{\ie}{\textit{i.e.}}
\newcommand{\eg}{\textit{e.g.}}
\newcommand{\etc}{\textit{etc.}}
\newcommand{\viz}{\textit{viz.}}
\newcommand{\vs}{\textit{vs.}}

\newcommand{\sql}{\acronym{sql}}

\newcommand{\code}[1]{{\ttfamily #1}}
\newcommand{\cmdline}[1]{{\ttfamily #1}}

\newenvironment{sqlcode}{
\small
\begin{minipage}{\textwidth}
\lstset{language=sql,
        keepspaces=true,
        showspaces=false,
        showstringspaces=false}
}{
\end{minipage}
}

\newenvironment{python}{
\small
\begin{minipage}{\textwidth}
\lstset{language=python,
        keepspaces=true,
        showspaces=false,
        showstringspaces=false}
}{
\end{minipage}
}

\newenvironment{ccode}{
\small
\begin{minipage}{\textwidth}
\lstset{language=C,
        keepspaces=true,
        showspaces=false,
        showstringspaces=false}
}{
\end{minipage}
}

\newcommand{\keyword}[1]{\textbf{#1}}
\newcommand{\identifier}[1]{\textit{#1}}

\newcommand{\Rom}[1]{\uppercase\expandafter{\romannumeral #1\relax}}

\newcommand{\CC}{C\nolinebreak[4]\hspace{-.05em}\raisebox{.3ex}{\relsize{-2}{\textbf{++}}}}
\newcommand\csharp{C\nolinebreak[4]\hspace{-.02em}\raisebox{.3ex}{\relsize{-1}{\#}}}

\newcommand{\comment}[1]{\textcolor{red}{#1}}

\newcommand{\nowdb}{\textsc{n}o\textsc{wdb}}

\newcommand{\connect}[2]{
  \draw [-,color=black] (#1) to (#2)
}

% tikz
% \newcommand{\drawDataPoint}{\draw circle (0.2)}
% \drawDataGroup

% \renewcommand{\gcd}{\textsc{gcd}}


\usepackage{authblk}
\usepackage[toc,page]{appendix}
\usepackage{url}
\usepackage{hyperref}
\usepackage{algorithmic}
\usepackage{nicefrac}

\begin{document}
\setlength{\parindent}{0pt}
\setlength{\parskip}{8pt}

\title {NoWDB}
\author {tobias.schoofs@gmx.net}
\date{\today}
%\includegraphics[scale=0.75]{sonic.jpg}\\[24pt]\today}
\maketitle
\tableofcontents

\chapter{Introduction}\label{chpt_intro} 
\nowdb\ is a kind of database.
It merges the concepts
of \term{graph} and \term{timeseries} database.
Timeseries databases typically have simple
data models centred around timelines
consisting of pairs of the form
$(timestamp,value)$
with additional $tags$ to
distinguish thematically different timelines.
An example may be a weather forecast
application with timelines describing
temperature, humidity and air pressure
at certain locations. The values would
reflect these measurements and timestamps
would refer to the points in time when
the measurements were taken. Tags would
be used to distinguish timelines
(temperature, humidity, pressure) and
to identify the location from where the
respective measurement comes.

Graph databases replace the set-theoretic
fundaments of relational databases by
graph theory. Graph databases do not deal
with relations over sets, but with
sets of vertices that are connected
by edges. In Twitter-like applications,
vertices may represent users.
The connections between users such as
\term{following} could then be expressed
in terms of edges between vertices.
Applications built on top of graph databases
typically focus on finding relations
between vertices; a goal may be to decide
whether a user $A$ belongs to the network
of a user $B$ where a follower of a
follower of $B$ is considered part of
$B$'s network. Another challenge may be
to compute how many users have seen
a certain tweet or how many users
see tweets of user $A$ in general.

While timeseries databases stress
fast processing of large volumes of
data with similar and, typically, simple structure,
graph databases focus on efficiently handling
data with growing complexity.

\nowdb\ aims to provide the
performance advantages of timeseries databases
using concepts from graph databases
to allow more complex data models
than usually seen with timeseries databases.
\nowdb\ can thus be applied to a wider
range of applications than pure
timeseries databases without losing
their performance advantage.

\nowdb\ is in particular strong with
data that can be organised in some variant of the
\term{star schema}. Relations between
fact tables and dimensional tables
are expressed in terms of
weighted and timestamped edges
between dimensional data that
are represented as complex vertices.
In contrast to traditional
\term{star schema} applications,
\nowdb\ is not limited to data analysis.
Many features stress real-time data
processing providing \term{publish and subscribe},
online data filtering and integration with big-data
infrastructure.

\newpage
This manual documents the main features
of the database and discusses important
use cases. The next chapter provides
a \term{Quick Start} tutorial that helps
understanding the concepts behind \nowdb\
and introduces the most important tools.
The chapter will close
with an overview of the remaining
chapters of this document.

\comment{
Throughout the document, the reader will encounter
red comments like this one.
These comments aim to clarify the current state
of the prototype. They, in particular, draw
attention to features that are not yet available
or to shortcomings of their current implementation.
This way, the manual also serves as an agenda
for the months to come. The goal is indeed
to get rid of all those comments. When the last
red line has gone, \nowdb\ is ready for the first
release.
}

\chapter{Quick Start}\label{chpt_quickst}
\section{Getting Started}

The easiest way to get started
is to use the \nowdb\ docker containing
the database server and clients.

\comment{more instructions of how to get it $\dots$}

The docker does not start the database by itself.
You need to start it explicitly. There is a script
called \tech{nowstart.sh} in the root of the docker
that does that.
(You may want to adapt this script to your specific needs!)

Here is one way to start the docker:

\cmdline{
docker run --rm -p 55505:55505 $\backslash$\\
\hspace*{2cm} -v /opt/dbs:/dbs -v /var/log:/log $\backslash$\\
\hspace*{2cm} -d nowdbdocker /bin/bash -c "/nowstart.sh"
}

This command creates the docker container and starts it.
The parameters are
\begin{itemize}
\item \tech{--rm}
instructs the docker daemon to remove
the container immediately after it will have stopped.

\item \tech{-p 55505:55505} binds the host port \term{55505}
to the same docker port.

\item \tech{-v} maps the host path
\tech{/opt/dbs} to the docker path \tech{/dbs} and
the host path \tech{/var/log} to the docker path \tech{/log}.

\item \tech{-d} means the docker runs in the background
(\term{detached}).

\item \tech{nowdbdocker} is the name of the docker image.

\item \tech{/bin/bash} is the command to be executed
within the docker; \tech{-c} passes a command
to be executed to \tech{bash},
namely \tech{/nowstart.sh}.
\end{itemize}

The script \tech{nowstart.sh}, contains the
instructions to start the \nowdb\ daemon.

Looking into the script,
we see it first sets some environment variables:

\cmdline{
export LD\_LIBRARY\_PATH=/lib:/usr/lib:/usr/local/lib
}

This sets the search path for shared libraries.

\cmdline{
export PYTHONPATH=/pynow:\$PYTHONPATH
}

This sets the search path for Python modules
(we will discuss that later).

Then the script starts the daemon iself:

\cmdline{
nowdbd -b /dbs -y 2>/log/nowdbd.log
}

The script passes two options to the daemon:
the base directory where all databases
managed by this particular daemon live (\tech{-b /dbs})
and the \tech{-y} switch, which activates
server-side Python support.

Now the daemon is listening to port 55505
and is ready to respond to database requests.
The daemon starts by printing a welcome banner
to standard output:

\begingroup
\small
\begin{minipage}{\textwidth}
\begin{verbatim}
+---------------------------------------------------------------+ 
 
  UTC 2018-10-09T12:03:21.631000000
 
  The server is ready
 
    - with python support enabled
 
+---------------------------------------------------------------+ 
  nnnn   nnnn          nnnn    nnnnnn       nnnnnn       nnnnnn  
    wi  i   wi       i      i    iw           iw           er   
    wi i     wi     n        n    iw         e  wi        e    
    wii      wi    wi        iw    iw       e    wi      e        
    wi       wi    wi        iw     iw     e      wi    e        
    wi       wi     n        n       iw   e        wi  e        
    wi       wi      i      i         iw e          wie          
   nnnn     nnnn       nnnn             n            n            
+---------------------------------------------------------------+ 

connections: 128
port       : 55505
domain     : any
base path  : /dbs
\end{verbatim}
\end{minipage}
\endgroup

\begin{minipage}{\textwidth}
Here are some more options provided by
the \nowdb\ daemon:

\begin{itemize}
\item \tech{-b} The base directory,
where databases are stored.
(default is the current working directory).

\item \tech{-s} the binding domain, default:
any. If set to a host or a domain, the server
will accept only connections from that host
or domain. Example: \tech{-s localhost} does
only accept connections from the server.
The host or domain can be given as name (\term{localhost})
or as \acronym{ip} address (\tech{127.0.0.1});

\item \tech{-p} the port to which
the server will listen; default is 55505,
but any other (free) port may be used.
 
\item \tech{-c} size of the connection pool.
If the argument is 0, the connection pool grows
indefinitely; otherwise, for \tech{-c n},
$n$ being a positive integer,
the server will create a connection pool
up to $n$ sessions and, all sessions in
the pool are used, refuse to accept
more connections. The default pool size is 128;
\comment{Notice that due to a memory leak
in the Python interpreter, there is no way
to instruct the server to accept indefinitely
many connections, but to only maintain $n$
in the pool.}

\item \tech{-q} runs in quiet mode
(\ie\ no debug messages are printed to standard error);
\item \tech{-n} does not print the starting banner;
\item \tech{-y} activate server-side Python support;
\item \tech{-l} activates server-side Lua support;
\item \tech{-V} prints version information to standard output;
\item \tech{-?} or \tech{-h} prints a help message to standard error.
\end{itemize}
\end{minipage}

\section{First Steps}
Once \term{nowdbd} is running, we can connect to pass
queries to the server. There is a tool to do this from the command line
called \term{nowclient}. Here is a usage example:

\cmdline{
nowclient -d retail -Q "select count(*) from sales where customer=12345"
}

In this form, the client will try to connect to a server
running on the same host and listening to port 55505.
Furthermore, it will request to use the database \term{retail}
and send the query indicated by the \tech{-Q} parameter.

\begin{minipage}{\textwidth}
If successful, the client will print some
processing information to standard error and the query result
to standard output, \eg:

\cmdline{
executing "use retail" \\
OK \\
executing "select count(*) from sales where customer=12345" \\
59
}
\end{minipage}

\begin{minipage}{\textwidth}
With option \tech{-q} we suppress the processing information.
We would then only see the result:

\cmdline{
59
}
\end{minipage}

\begin{minipage}{\textwidth}
Here are more options supported by the client tool:
\begin{itemize}
\item \tech{-s} 
The server address or name, \eg\ \tech{myserver.mydomain.org} or
\tech{127.0.0.1}. Default is \tech{127.0.0.1};

\item \tech{-p}
the port to which the database is listening. Default: 55505;

\item \tech{-d}
the database to which we want to connect. 
Default: no database at all, which means
that we cannot send queries without naming a database.
Below we will look at alternatives to using this parameter;

\item \tech{-Q}
the query we want the database to process;

\item \tech{-t}
print some (server-side) timing information to standard error;

\item \tech{-q}
quiet mode: don't print processing information to standard error.

\item \tech{-V} prints version information to standard output;
\item \tech{-?} or \tech{-h} prints a help message to standard error.
\end{itemize}
\end{minipage}

The client tool is able to read from standard input;
this way, more than one query can be processed by one call
to \term{nowclient}.
The following command processes the same query as the one above,
but uses standard input instead of the options \tech{-d} and \tech{-Q}:

\cmdline{
echo "use retail;select count(*) from sales where customer=12345;" $| \backslash$ \\
\hspace*{2cm} nowclient
}

Notice that, using standard input,
we need to terminate single \sql\ statements
by a semicolon. This is even true for the last statement.
Leaving the semicolon out would lead to an error.

Of course, we can do much more useful things than just
getting rid of the options.
The main point of reading from standard input is
that we can put \sql\ statements into a file and
$cat$ it to nowclient. A useful example may be:

\begin{sqlcode}
\begin{lstlisting}
drop schema retail if exists;
create schema retail; use retail;

create large table sales set stress=constant;
create table statistics;

create medium index idx_sales_eds on sales (edge, destin);
create index idx_sales_eor on sales (edge, origin);

create type product (
        prod_key uint primary key,
        prod_desc text,
        prod_price float
);
create type client (
        client_key uint primary key,
        client_name text
);
create edge buys (
        origin client,
        dest product,
        weight uint,
        weight2 float
);
load '/opt/data/products.csv' into product use header;
load '/opt/data/clients.csv' into client use header;
load '/opt/data/sales.csv' into sales;
\end{lstlisting}
\end{sqlcode}

Let's assume we had this code in a file called
\tech{create\_retail.sql}; then we could
send it to \term{nowclient}:

\cmdline{
cat create\_retail.sql | nowclient
}

which would create the retail database.

The script shows some of the peculiarities of \nowdb.
The beginning is quite regular \sql:

\begin{sqlcode}
\begin{lstlisting}
drop schema retail if exists;
create schema retail; use retail;
\end{lstlisting}
\end{sqlcode}

The first line drops the database retail,
\ie\ it removes all its data physically
from disk. The \term{if exists} part
is included to avoid an error 
(and hence the termination of the script)
in the case
the database does not yet exist.

In the second line the schema `retail'
is created.
The third statement (still in the second line)
instructs \nowdb\ to use the newly created
schema `retail' in all following statements.

The next line is a bit uncommon:

\begin{sqlcode}
\begin{lstlisting}
create large table sales set stress=constant;
\end{lstlisting}
\end{sqlcode}

The statement creates a table called `sales';
\nowdb, however, is not a relational database.
There are no  
`tables' with the meaning of that term
in the relational world.
In fact, tables are just units of storage 
for edges. \comment{A better naming convention
would probably be `tablespace'. The first name,
by the way, was \term{context} $\dots$}

The statement explicitly says that we
want a \emph{large} table and that there
will be \emph{constant} stress (\ie\ ingestion load)
on that table. The \term{create table} statement,
hence, is much more oriented to storage
and processing details than to the logical structure
of the table (as it would be in a relational database).

A similar is true for the index creation:

\begin{sqlcode}
\begin{lstlisting}
create medium index idx_sales_eds on sales (edge, destin);
\end{lstlisting}
\end{sqlcode}

This statement creates a \emph{medium} index on table \term{sales}
with the fields `edge' and `destination' as index keys.
Index sizing is a difficult topic and will be discussed
in chapter \ref{chpt_sizing}.
As a rule of thumb, it is almost always better to assume
small sizing. Large indices have large storage nodes
and those nodes are very often read from and written
to disk. Only when we know that our index will have
many data points per key, it is advisable
to use larger index sizing. The default sizing (used in the next
line) indicatively is \term{small}.

The next two blocks of code create the types
`product' and `client'.
Types describe vertices.
Each database has a set
of vertices that can be connected
by means of edges to form graphs.
The graphs structure all data
in our specific application.
The vertex types thus describe the universe of discourse.
Technically, vertices are stored in
\term{column-oriented} tables (which are
invisible to the user).
\comment{Well, not entirely true at the moment,
but it is the direction to go -- sooner or later
the statement will be true $\dots$}

The attribute types used in the script
are \term{uint}, \term{float} and \term{text}.
The first is a 64bit unsigned integer;
the second is a 64bit floating point number
(a.k.a. \term{double} in languages like C);
\term{text} is a string of up to 255 bytes,
which represent \acronym{utf}-8 characters.
For more information on \sql\ types,
please refer to chapter
\ref{chpt_sql}.

Every vertex type needs a \term{primary key} and
that primary key must consist of only one attribute.
There are no composed keys like in relational
databases.

The next block defines an edge.
Edges are links between vertices.
Contrary to vertices, edges have a fixed structure
with up to seven fields:
\term{edge}, \term{origin}, \term{destination},
\term{label}, \term{timestamp},
\term{weight} and \term{weight2}.

The first field \term{edge} identifies the edge type.
This appears somewhat redundant (from the relational perspective),
but it is indeed necessary, because
edges of different types can be stored in the same table.
The type of the \term{edge} field is \term{text}
and cannot be changed. It is therefore not mentioned in
the script. However, it is always present.

The next two fields, \term{origin} and \term{destination}
(usually shortened to \term{destin} or even \term{dest})
refer to the vertices that are linked by this edge.
The types of \term{origin} and \term{destin} are defined
by the user. The types are restricted to vertex types.
In the script, the types are
\term{client} and \term{product},
which we defined before.

The next field is \term{label}.
Labels can be used
to identify inherent relations between edges.
The type of the label field is defined by the user.
It may be either \term{uint} or \term{text}.
Since we have not defined the label field in the script above,
the label would not be available in this specific edge type.
A useful example using labels is given further below.

With the next field, the timeseries aspect of \nowdb\
comes in. Indeed, edges have a \term{timestamp}.
In our retail application,
edges of type \term{buys}, indicate that a client
bought a certain product \emph{at a given time}.

The timestamp is always present in all edges and
its type cannot be changed.
It always has the type \term{time},
which (usually) is a positive or negative
offset from the \acronym{unix} \term{epoch}
with nanosecond precision.

Finally, edges are weighted. Each edge can be weighted
in two dimensions (\term{weight} and \term{weight2}).
The number $2$ is somewhat arbitrary.
It is in fact a compromise between space and convenience.
One should not interpret too much into it.

The \term{weight} fields can have any basic type
(\term{uint}, \term{int}, \term{float}, \term{text},
\term{time} or \term{bool}).

The fields whose type can be defined by the user,
can also be renamed. We could say, for instance,
\tech{weight uint as quantity}.
The field \term{weight} would then be accessible as
\term{quantity} in \sql\ statements. 
\comment{Renaming is not yet possible.}

In the script above we define four of the fields,
but six will be accessible by the application,
because \term{edge} and \term{timestamp} are always available.
But \term{label} will remain invisible as if it did not exist.

In the final section,
the script loads data from three different \acronym{csv}s
into the database. \nowdb\ provides loaders
for different formats. \acronym{csv} is just one example.
There are many more and the loader even allows 
users to define
their own formats using Apache Avro.
With Avro it is possible to define binary formats, which
can be much faster than textual formats such as \acronym{csv}.
Using a serialisation system like Avro also eases
interoperability of the database with
external systems and applications and it
significantly eases version management should data formats
change over time (what they always do). 
For more details on data loaders, please refer to chapters
\ref{chpt_sql} and \ref{chpt_loader}.
\comment{Avro is not yet available.}

Loaders, in general, are usually much more efficient than
the \sql\ \term{insert} statement.
The drawback of \term{insert} is that each statement
needs the whole cycle of \sql\ parsing and execution,
while loaders only need one cycle. Since a data source
can contain millions or even billions of rows,
loading is way more efficient than inserting in most cases.

The first two \acronym{csv}s in the script contain vertices.
As such they need to have a header and we need to instruct
the loader of how to interpret the data.
Since edges always have the same format, they don't need a header
and we do not give any instructions of how to interpret the data.

\section{First Queries}
The alternative to loading data is, of course, the conventional
\term{insert} statement:

\begin{sqlcode}
\begin{lstlisting}
insert into client(9000001, 'Popeye the Sailor');
insert into product(100001, 'Spinach, 450g net', 1.99);
\end{lstlisting}
\end{sqlcode}

These two statements insert a client and a product respectively.
We can also name the attributes explicitly, like:

\begin{sqlcode}
\begin{lstlisting}
insert into product(prod_key, prod_desc, prod_price)
              (100002, 'Candy Cigarettes, 20', 2.49);
\end{lstlisting}
\end{sqlcode}

Now we insert a bunch of edges:

\begin{sqlcode}
\begin{lstlisting}
insert into sales (edge, origin, destin, timestamp, weight, weight2)
           ('buys', 9000001, 100001, '1929-01-17T09:35:12', 1, 1.99)
insert into sales (edge, origin, destin, timestamp, weight, weight2)
           ('buys', 9000001, 100001, '1929-01-19T10:15:01', 2, 3.98)
insert into sales (edge, origin, destin, timestamp, weight, weight2)
           ('buys', 9000001, 100001, '1929-01-20T17:12:55', 3, 5.97)
insert into sales (edge, origin, destin, timestamp, weight, weight2)
           ('buys', 9000001, 100001, '1929-01-22T08:27:32', 1, 1.99)
insert into sales (edge, origin, destin, timestamp, weight, weight2)
           ('buys', 9000001, 100001, '1929-01-25T12:09:59', 1, 1.99)
insert into sales (edge, origin, destin, timestamp, weight, weight2)
           ('buys', 9000001, 100001, '1929-01-26T21:19:44', 2, 3.98)
insert into sales (edge, origin, destin, timestamp, weight, weight2)
           ('buys', 9000001, 100002, '1929-01-22T08:27:51', 1, 2.49)
\end{lstlisting}
\end{sqlcode}

\begin{minipage}{\textwidth}
Worth noticing here is the time format,
which follows \acronym{iso}-8601.

The format is
\begin{itemize}
\item 4 digits for the year and hyphen
\item 2 digits for the month and hyphen
\item 2 digits for the day of month
\item `T' to mark the beginning of the time section
\item 2 digits for the hour and colon
\item 2 digits for the minute and colon
\item 2 digits for the second and,
\item if finer grain is necessary,
a dot followed by up to 9 digits
for the nanoseconds.
\end{itemize}
\end{minipage}

This format can be used anywhere in \nowdb\ \sql.
However, it is also possible to define custom
date and time formats. How to do this is discussed
in chapter \ref{chpt_sql}. \comment{That's not yet possible.}

Worth noticing is also the first date.
It was on Jan, 17, 1929 that Popeye had his first
appearance in a newspaper of the \term{King Features}
Syndicate.

Now that we have inserted some data
into our database, we are able to perform $selects$, \eg:

\cmdline{
nowclient -d retail -Q
"select count(*) from sales $\backslash$ \\
\hspace*{4.7cm} where edge='buys' $\backslash$ \\
\hspace*{4.7cm} and origin=9000001"
}

which would give us 7 and would count Popeye's visits to the supermarket.
We can also count how often Spinach was bought:

\begin{sqlcode}
\begin{lstlisting}
select count(*) from sales
 where edge='buys'
   and destin=100001
\end{lstlisting}
\end{sqlcode}

which shows us 6.
Or we can ask how much Popeye bought and paid per type of product:

\begin{sqlcode}
\begin{lstlisting}
select edge, destin, count(*), sum(weight), sum(weight2)
  from sales 
 where edge='buys' 
   and origin=9000001 
 group by edge, destin
\end{lstlisting}
\end{sqlcode}

\begin{minipage}{\textwidth}
which would give us:
\begin{verbatim}
buys;100001;6;10;19.9000
buys;100002;1;1;2.4900
\end{verbatim}
\end{minipage}

Notice that the output
of the client tool does not resemble the classical
pretty-printed output produced by most database
client tools today. The advantage of such output is
that it is easier for humans to read.
The \acronym{csv}-like output shown above, however,
is better for interoperability, for instance,
when we want to combine it through pipes
with other programs, like this:

\cmdline{
nowclient -d retail -Q "select * from sales" | cut -d";" -f2 | $\dots$
}

On the other hand, there are tools that
produce more readable output from \acronym{csv} input,
such as \tech{csvlook} from the \tech{csvkit} package.\footnote{Have
a look at
\url{https://github.com/jeroenjanssens/data-science-at-the-command-line}}
To obtain a traditional pretty-printed output we could do the following
(assuming that \tech{query.sql} contains the query we used above):

\cmdline{
cat query.sql | nowclient | csvformat -d";" | $\backslash$ \\
    header -a 'edge,product,count,quantity,price' | csvlook
}

and would obtain for the grouping query used above:

\begin{minipage}{\textwidth}
\begin{verbatim}
|----------+---------+-------+----------+---------|
|  edge    | product | count | quantity | price   |
|----------+---------+-------+----------+---------|
|  buys    | 100001  | 6     | 10       | 19.9000 |
|  buys    | 100002  | 1     | 1        | 2.4900  |
|----------+---------+-------+----------+---------|
\end{verbatim}
\end{minipage}

Here is a more typical time series query illustrating
the advantage of the pretty printer:

\begin{sqlcode}
\begin{lstlisting}
select destin, timestamp, weight, weight2
  from sales 
 where edge='buys' 
   and origin=9000001 
 order by timestamp
\end{lstlisting}
\end{sqlcode}

\begin{minipage}{\textwidth}
which, with the same technique as above, shows:
\begin{verbatim}
|----------+---------------------+----------+---------|
|  product | timestamp           | quantity | price   |
|----------+---------------------+----------+---------|
|  100001  | 1929-01-17T09:35:12 | 1        | 1.9900  |
|  100001  | 1929-01-19T10:15:01 | 2        | 3.9800  |
|  100001  | 1929-01-20T17:12:55 | 3        | 5.9700  |
|  100001  | 1929-01-22T08:27:32 | 1        | 1.9900  |
|  100002  | 1929-01-22T08:27:51 | 1        | 2.4900  |
|  100001  | 1929-01-25T12:09:59 | 1        | 1.9900  |
|  100001  | 1929-01-26T21:19:44 | 2        | 3.9800  |
|----------+---------------------+----------+---------|
\end{verbatim}
\end{minipage}

We can also select from vertices,
but instead of a table, we use the type:

\begin{sqlcode}
\begin{lstlisting}
select prod_price from product
 where prod_key = 100001;
\end{lstlisting}
\end{sqlcode}

which shows
\begin{verbatim}
1.99
\end{verbatim}

and

\begin{sqlcode}
\begin{lstlisting}
select client_name from client
 where client_key = 9000001;
\end{lstlisting}
\end{sqlcode}

which gives
\begin{verbatim}
Popeye the Sailor
\end{verbatim}

Much more typical for \nowdb, however,
is to use vertices together with edges. Edges connect vertices
and can therefore be seen as the `relations' in \nowdb.
What we typically want is either find the vertex
at the other end of the edge (\eg\ find the destination
for a given origin) or to look at edges with the attributes
of the vertices added to them.

Both patterns are, in \sql, instances of \term{joins}.
An instance of the first pattern would be:

\begin{sqlcode}
\begin{lstlisting}
select timestamp, prod_desc, prod_price
  from sales join product on destin
 where edge = 'buys'
   and origin = 9000001;
\end{lstlisting}
\end{sqlcode}

\begin{minipage}{\textwidth}
\begin{verbatim}
|---------------------+----------------------+---------|
| timestamp           | product              | price   |
|---------------------+----------------------+---------|
| 1929-01-17T09:35:12 | Spinach, 450g net    | 1.9900  |
| 1929-01-19T10:15:01 | Spinach, 450g net    | 1.9900  |
| 1929-01-20T17:12:55 | Spinach, 450g net    | 1.9900  |
| 1929-01-22T08:27:32 | Spinach, 450g net    | 1.9900  |
| 1929-01-25T12:09:59 | Spinach, 450g net    | 1.9900  |
| 1929-01-26T21:19:44 | Spinach, 450g net    | 1.9900  |
| 1929-01-22T08:27:51 | Candy Cigarettes, 20 | 2.4900  |
|---------------------+----------------------+---------|
\end{verbatim}
\end{minipage}

Note the difference in the price column.
In the previous query we used \term{weight2} of sales,
which (as you may have realised)
is the multiplication of the product price and the value
in \term{weight}, which, in its turn,
represents the number of items.
Here, however, we use the value in \term{prod\_price}
which is the base price of one unit of that product.

We can, of course, combine joins with grouping:

\begin{sqlcode}
\begin{lstlisting}
select edge, destin, count(*), sum(prod_price)
  from sales join product on destin
 where edge = 'buys'
   and origin = 9000001
 group by edge, destin;
\end{lstlisting}
\end{sqlcode}

\begin{minipage}{\textwidth}
\begin{verbatim}
|----------+---------+-------+---------|
|  edge    | product | count | price   |
|----------+---------+-------+---------|
|  buys    | 100001  | 6     | 11.9400 |
|  buys    | 100002  | 1     | 2.4900  |
|----------+---------+-------+---------|
\end{verbatim}
\end{minipage}

Note again the sum of the price which, here,
is just the sum of the base price per unit of the product.

The point about the first joining pattern
is that it joins only one of the two vertices
with the edge. The second pattern is a bit more complex:

\begin{sqlcode}
\begin{lstlisting}
select timestamp, prod_desc, client_name
  from sales
  join product on destin
  join client on origin
 where edge = 'buys'
   and origin = 9000001;
\end{lstlisting}
\end{sqlcode}

The result of this query would be:

\begin{minipage}{\textwidth}
\begin{verbatim}
|---------------------+----------------------+-------------------|
| timestamp           | product              | client            |
|---------------------+----------------------+-------------------|
| 1929-01-17T09:35:12 | Spinach, 450g net    | Popeye the Sailor |
| 1929-01-19T10:15:01 | Spinach, 450g net    | Popeye the Sailor |
| 1929-01-20T17:12:55 | Spinach, 450g net    | Popeye the Sailor |
| 1929-01-22T08:27:32 | Spinach, 450g net    | Popeye the Sailor |
| 1929-01-25T12:09:59 | Spinach, 450g net    | Popeye the Sailor |
| 1929-01-26T21:19:44 | Spinach, 450g net    | Popeye the Sailor |
| 1929-01-22T08:27:51 | Candy Cigarettes, 20 | Popeye the Sailor |
|---------------------+----------------------+-------------------|
\end{verbatim}
\end{minipage}

\comment{
Unfortunately, joins are not yet available :-(
}

An edge field that we have not yet used is the \term{label}.
The \term{label} is intended to create a connection
between edges that are inherently related. In our example, we see
that at one day Popeye bought two different things:
spinach and candy cigarettes.
That was on Jan, 22.

\begin{minipage}{\textwidth}
These two edges, hence, relate to the same visit at the supermarket.
We could link these edges using a label. The edge would then be created
as, for instance:
\begin{sqlcode}
\begin{lstlisting}
create edge buys (
        origin client,
        dest product,
        label text,
        weight uint,
        weight2 float
);
\end{lstlisting}
\end{sqlcode}
\end{minipage}

The label text would correspond to a token generated by
the cashpoint when a customer starts the check-out
and all products scanned during the procedure
would have the same token.
We could then insert edges using that token as label:

\begin{sqlcode}
\begin{lstlisting}
insert into sales (edge, origin, destin, timestamp,
                            label, weight, weight2)
   ('buys', 9000001, 100001, '1929-01-22T08:27:32',
                      'tx-19290122082731', 1, 1.99)
insert into sales (edge, origin, destin, timestamp,
                            label, weight, weight2)
   ('buys', 9000001, 100002, '1929-01-22T08:27:51', 
                      'tx-19290122082731', 1, 2.49)
\end{lstlisting}
\end{sqlcode}

\begin{minipage}{\textwidth}
We now can select edges according to the label:

\begin{sqlcode}
\begin{lstlisting}
select destin, timestamp, weight, weight2
  from sales 
 where label = 'tx-19290122082731'
 order by timestamp
\end{lstlisting}
\end{sqlcode}
\end{minipage}

\begin{minipage}{\textwidth}
This query would result in:

\begin{verbatim}
|----------+---------------------+----------+---------|
|  product | timestamp           | quantity | price   |
|----------+---------------------+----------+---------|
|  100001  | 1929-01-22T08:27:32 | 1        | 1.9900  |
|  100002  | 1929-01-22T08:27:51 | 1        | 2.4900  |
|----------+---------------------+----------+---------|
\end{verbatim}
\end{minipage}

\section{The Python Client}
Until here we always used the client \emph{tool}
to perform queries.
That is certainly an important use case.
Much more typical, however, is to develop application code
that needs a client \acronym{api} to connect to the database.

\nowdb\ comes with a native client \acronym{api}
that is available in different languages, among others
C, \CC, Go, Python and Lua.
\comment{Only Python is currently available.
There is a low-level \acronym{api} for C,
but that is not for developing applications,
but for developing \acronym{api}s.}
Here, we will have a quick look at the Python \acronym{api}.

The module implementing the Python \nowdb\ \acronym{api}
is called \term{now.py} and must be imported into the client program.
For the Python interpreter to find this module,
it must be in a directory in the \acronym{pythonpath}.
There may be different ideas on how to install python modules.
The \nowdb\ installation will copy all \nowdb-related modules
to one specific folder and add this folder to the
\acronym{pythonpath}. But you also may install
the \nowdb\ Python \acronym{api} using \tech{pip}.
Then, everything is handled by the Python environment
and you don't need to care about these things.
For more details on installation, please refer
to chapter \ref{chpt_install}.

Anyway, here is a simple Python program:

\begin{python}
\begin{lstlisting}
import now

with now.Connection("localhost", "55505", None, None) as c:
   with c.execute("use retail") as r:
       if not r.ok():
          print "cannot use retail: %s" % r.details()
          exit(1)

   with c.execute("select count(*), sum(weight), avg(weight) \
                     from sales where edge='buys'") as cur:
       if not cur.ok():
          print "ERROR: %s" % cur.details()
          exit(1)
       for row in cur:
           print "count: %d, sum: %d, avg: %.2f" %
                 (row.field(0), row.field(1), row.field(2))
\end{lstlisting}
\end{python}

The program first creates a \term{Connection}
to a database listening on port 55505 on `localhost'.
It then executes a \term{use} statement on this connection
to indicate the database towards which
the following statements are directed.

The result of a \term{use} statement is a \term{status}.
A status is either \acronym{ok} or an error
whose details can be obtained by means of the method
\term{details()}, which returns a string.
Results also support the method \term{code()},
which would return a numeric error code.
It is often useful to know the error code to decide
what to do programmatically (abandon the program,
retry, try something else, \etc)

Results are \term{resource managers}. That means
that they can be used inside a \term{with} statement.
\term{with} assures that all resources (in this case
the result) are freed before the control leaves
the \term{with}-block even if an exception is raised.

The program then executes a query.
This time the execution returns a \term{cursor} (`cur').
The program checks whether the cursor is
in a good state. The statement may have
failed on the server side. The cursor would
then be in a state that is not \acronym{ok}.
In that case, the program prints the error details
and exits with return code 1.

Otherwise, if the result was fine,
it iterates over the cursor
printing for each row the fields 1-3.
Cursors, indeed, are iterators
that allow simple iteration using \term{for}.

\begin{minipage}{\textwidth}
Here is an example of an \term{insert} statement
(we assume that the connection, `c', was already established):

\begin{python}
\begin{lstlisting}
with c.execute("insert into edge (edge, origin, product, timestamp) \
                                 ('complains', 9000001, 100002, \
                                  '1929-01-23T08:45:00')") as rep:
       if not rep.ok():
          print "ERROR: %s" % cur.details()
          exit(1)
       else:
          print "rows affected: %d" % rep.affected()
          print "running time : %d" % rep.runTime()
\end{lstlisting}
\end{python}
\end{minipage}

In the case of data manipulation (\acronym{dml})
or data loading (\acronym{dll}) statements,
the result (if there was not error) is a report
A report has the methods \term{affected()},
which indicates the number of rows affected by this statement,
\term{runTime()},
which indicates the running time of the statement in microseconds,
and \term{errors()},
which, in the case of \term{load},
indicates the number of rows that resulted in an error.

For more details on the Python client,
please refer to chapter \ref{chpt_pythonclient}.
The other client \acronym{api}s are described
in chapters \ref{chpt_ccpp}, \ref{chpt_goclient} and
\ref{chpt_luaclient}.

\section{Python in the Database}
Like many other databases,
\nowdb\ supports
stored procedures and stored functions,
\ie\ code that is executed within the database.
This code can be written in several languages.
Currently Python and Lua are supported.
\comment{Lua is not yet available.}

The major advantage of Python is the huge
number of libraries available,
in particular for mathematics,
data science and machine learning.
But the poor design of the Python interpreter
imposes some limitations in concurrency.
(Please refer to chapter \ref{chpt_pythonemb}
for details.)
It is therefore recommended to use Python
in the database where it is strongest,
\ie\ for higher math and data science.
For `normal' \acronym{dba}
jobs or simple computations that do not require
sophisticated math libraries, the use one of
the light-weight languages,
such as Lua, is recommended.

The difference between stored procedures
and stored functions is that stored functions
run inside an \sql\ context. They
can be used in a \term{select} clause,
for instance. They are not allowed, however,
to execute whatever they want.
In a \term{select} clause,
they are not allowed to
execute \term{updates} or \term{inserts} for instance.
It would indeed be very strange when
a \term{select} would change the database.

Stored procedures, on the other hand,
cannot run in \sql\ context. They are
executed explicitly by the \term{exec}
statement. In exchange for this limitation,
they get a lot of power:
stored procedures are allowed to run
any \sql\ code they (or their programmers)
like to including \acronym{dml}, \acronym{dll}
and even \acronym{ddl}.

Stored procedures are composed of two
elements: there interface (or \term{signature})
and their implementation.
The interface defines how the procedure is
to be called; the implementation defines
what it does.

The interface is create by the \sql\ statement
\term{create procedure}, \eg:

\begin{sqlcode}
\begin{lstlisting}
create procedure sales.revenue(pClient uint, pDay time) language python
\end{lstlisting}
\end{sqlcode}

This statement defines the interface of a procedure
called `revenue' that takes two arguments:
\begin{itemize}
\item $pClient$ which is an unsigned integer and
\item $pDay$ which is a timestamp.
\end{itemize}
Furthermore, the procedure is written in Python and
it lives in the module `sales'.
The statement does not define a return value for the procedure.
But all procedures in \nowdb\ return a \term{result type},
that is polymorphic type that
can be either a status, a report, a row or a cursor.

The module `sales' must be located in a directory that is
known to the Python interpreter. Typically, it will
be in a directory in the \acronym{pythonpath}.
Note that the \acronym{pythonpath} must be set
in the environment of the server before it is started.
This is actually a security feature:
it is not possible to smuggle executable code
into the database. To get code to run in the database
context one needs control over the filesystem.
The database does not introduce a new attack vector.

Once the function $revenue$ is created
and its code is accessible to the database, it
can be executed by an \term{exec} statement:

\begin{sqlcode}
\begin{lstlisting}
exec revenue(9000001, '1929-01-25')
\end{lstlisting}
\end{sqlcode}

\begin{minipage}{\textwidth}
Let's look into the module \term{sales}
to see how the procedure is implemented:

\begin{python}
\begin{lstlisting}
import nowdb
import datetime

def revenue(pClient, pDay):
  try:

    today = nowdb.now2dt(pDay)
    tom = today + timedelta(days=1)

    stmt = "select sum(weight2) from sales "
    stmt += "where edge = 'buys'"
    stmt += "  and origin =" + str(pClient)
    stmt += "  and timestamp >= " + today.strftime(nowdb.TIMEFORMAT)
    stmt += "  and timestamp <  " + tom.strftime(nowdb.TIMEFORMAT)

    with nowdb.execute(stmt) as c:

      if not c.ok():
        return nowdb.makeError(c.code(), c.details()).toDB()

      r = makeRow()
      for row in c: # there is only one row
         r.add2row(INT, row.field(0))

      return r.toDB()

  except Exception as x:
    return nowdb.makeError(USRERR, str(x)).toDB()

def cleanup():
  pass

\end{lstlisting}
\end{python}
\end{minipage}

First, we import the module \term{nowdb}.
This is the equivalent to the \term{now} module
that we imported on client side.
Note, however, that on server side
it is necessary to import the module in this format,
\ie: \term{import nowdb}, not: \term{from nowdb import $\dots$}
Otherwise, \nowdb\ would not be able to initialise the module.
In that case, the database would issue an error 
and not execute the procedure.

The module is initialised, by the way, when the session
terminates (or when the module is first loaded into a session).
This means
that each session, \ie\ each connection to the database,
has its own view of the module. This allows for
global variables, which will have the same lifetime
as the session itself.

Next step in the module is
the definition of the function \term{revenue} with two parameters.
The database will pass the parameters according to the type
information in the interface: the first parameter
will be passed as unsigned integer and the second
will be passed as timestamp.

The entire code of the procedure is encapsulated
in a \term{try/execpt}-block. This, indeed, is
good practice, since otherwise
an exception would terminate the procedure without
passing a result back to the database and without
detailed error information.

In the \term{except}-part, we create
an error with error code `user error' and
the exception string as detailed information.
Notice that we call the $toDB()$ method
and that we in fact return the result of that
method, not the result itself.
The purpose of the method is to separate
the C part and the Python part of the result.
The C part goes back to the database,
the Python part stays in the Python world
to be collected by the Python garbage collector.
Returning a result without calling its $toDB()$
method would pass Python stuff
to the database which would not know how to handle
that.

The main logic of the function
is in the \term{try}-part.
We start by converting the \nowdb-timestamp
into the Python \term{datetime} object \term{today}
using the \term{now2dt} function
(which, by the way,
is also available on client side).
By adding one day to \term{today} we get
\term{tom}(orrow).

With these values and the \term{pClient}
variable, we construct an \sql\ statement,
which is then passed to the \term{execute}
function. Notice that on server-side,
\term{execute} is not a method of another
object or class like \term{connection}.
Indeed, we already are inside the database.
We don't need a connection to talk to it.

Since the \sql\ statement
is a \term{select}, the result we get
back is a cursor. We check that the cursor
is \acronym{ok} and, if so, we work on
its rows using a \term{for} loop.
In fact, we have only
one row -- since we issued a \term{count}
without a \term{group by} clause.

From the cursor row, we create a new row
using the \term{makeRow} function
and add the first (and only) field 
of the cursor's row to it as integer. This row
is returned to the database (using,
of course, its $toDB()$-method).

Note that we could have returned the cursor itself --
since a cursor is also a result type and, as such,
a valid object to be returned by a stored procedure.
We return the row only for the purpose of illustrating
how rows can be created in Python.
\comment{The whole example is quite trivial to be honest.}

The database will then send the result,
that is the row we created,
back to the caller (the \term{exec} statement),
which, in its turn, can handle this result.
When we call the \term{exec} statement
from \term{nowclient} like this:

\cmdline{
nowclient -d retail -Q "exec revenue(9000001, '1929-01-25')"
}

the result would look like this:
\begin{verbatim}
1.9900
\end{verbatim}

\begin{minipage}{\textwidth}
Of course, we can call \term{exec}
also from a Python client and then handle
the result as row, \eg:

\begin{python}
\begin{lstlisting}
with Connection("127.0.0.1", "55505", None, None) as c:
  with c.execute("exec revenue(9000001, '1929-01-25')") as row:
    if not row.ok():
       print "ERROR: %s" % row.details()
       exit(1)
    print "revenue from client 9000001: %d" % row.field(0)
\end{lstlisting}
\end{python}
\end{minipage}

But we have to come back to the server side once again.
There is still a detail that we have not discussed,
namely the $cleanup()$ function at the end of the module.
Modules may (but do not need to) contain this
function. If it is present in the module, it is automatically
called by the database when the session terminates.
For simple code like the one we used
here for illustration, it is not necessary to have a cleanup
function. 

A cleanup is necessary, whenever result types
(cursors, rows, \etc) are stored in global variables.
It is completely legal to store data in global variables.
When the session terminates, the database will reload
the module, so that the next session will start with
fresh instances of those variables. However, when
global variables are holding result types, the underlying
C part of these variables must be released.
Since the database has no knowledge on what is stored
in global variables, the user needs to provide code
to release this memory.

One interesting use case is \term{functional cursors}.
Functional cursors behave like normal cursors,
but they can use the power of the embedded language
to enrich results
using functionality
not available in \sql.
This is discussed in chapter \ref{chpt_pythonemb}.

\section{What's Next?}
This chapter was only a brief introduction
to some important features and the general flavour 
of \nowdb. The remainder of this manual will
discuss the main features more deeply.

The next chapter discusses the \nowdb\ \sql\ dialect.

Chapters \ref{chpt_nowdbd} and \ref{chpt_clienttool}
present the command line tools \tech{nowdbd}
and \tech{nowclient} respectively.

Chapters \ref{chpt_ccpp} -- \ref{chpt_luaclient}
discuss the available native clients in different
languages and \ref{chpt_llc} presents the low-level
C \acronym{api} mainly used to implement
native clients.

The next two chapters, \ref{chpt_pythonemb} and
\ref{chpt_luaemb}, present the server-side
language bindings at more depth.

Chapter \ref{chpt_install} provides detailed information
on installation of server and client on different
platforms.

The next chapters present more features,
namely the loader (\ref{chpt_loader}) and
server-side techniques such as publish and subscribe and
filters (\ref{chpt_pubsub}).

The following chapters 
\ref{chpt_opt} and \ref{chpt_sizing}
discuss technical insight
for application designers and \acronym{dba}s.

Finally, the appendix \ref{chpt_errors} lists server-side
error codes.




\chapter{Data Modelling}\label{chpt_model}
\comment{
stress differences between graph and relational\\
what is good design / best practice for both:
timeseries and graph \\
examples:
wmo (pure timeseries),
retail (timeseries + graph)
}

\chapter{SQL}\label{chpt_sql}
\section{Outline}
\sql\ is a language to store, manipulate
and query data in a database; traditionally
\sql\ is used with relational databases.
In recent years, however, people have
started to use \sql\ also in other contexts,
such as \term{graph} and \term{timeseries}
databases and new patterns are evolving
in the language to better address those
data models.

\sql\ consists of statements that,
in their turn, consist of clauses.
A statement is a piece of \sql\ code
that by itself constitutes a meaningful
action. Statements are
distinguished in

\begin{itemize}
\item \acronym{ddl}:
Statements that create, drop or alter entities
in the database that hold or define
data
like storages, types, edges, indices,
procedures, \etc\

\item \acronym{dml}:
Statements that manipulate data,
\eg\ \term{insert}, \term{update} and
\term{delete}.

\item \acronym{dll}:
Statements that load large volumes of data into the database
or retrieve large volumes of data from the database.

\item \acronym{dql}:
Statements that read data from the database.

\item Miscellaneous:
Statements that do not fall into any of
those categories, in particular
\term{use} and \term{exec}.
\end{itemize}

Clauses are parts of statements;
a \acronym{dql} statement, for instance,
typically has a \term{select} clause and
a \term{from} clause and may have
additional clauses (\term{where},
\term{order by}, \term{group by} and so on).

Some clauses can appear in more than
one type of statement. \term{update}
and \term{delete} statements, typically,
have a \term{where} clause, but no
\term{select} clause.

Clauses can be seen as logical building blocks
of \sql. But they cannot live alone.
It is not possible to execute an isolated \term{where}
clause or an isolated \term{from} clause.
The smallest executable unit is therefore the statement.

Clauses are made of keywords, identifiers, numbers, text,
symbols (such as $, \dot = ( ) * + -$) and
whitespace, \ie\ \acronym{ascii}
10 (line break),
13 (carriage return),
9  (horizontal tab)
and 32 (space).

Keywords and identifiers are mutually
exclusive, that is, if $k$ is a keyword,
$k$ cannot be an identifier at the same time.
\comment{This rule is relaxed in most
\sql\ dialects -- and that is a great relief for users,
because \sql\ has an extraordinary large
number of keywords which sometimes makes the choice
of meaningful identifiers a painful.
At the time of writing, the \nowdb\ parser
does not yet relax this rule, but it will do so
in the future.}
Keywords are defined by the \sql\ specification
and represent syntactic elements for chosing actions
over entities in the database;
identifiers are chosen by the user
and refer to constant values or
entities in the database,
such as edges, types, indices, \etc\

In this specification,
keywords are typeset in boldface
(\eg\ \keyword{select});
identifiers are typeset in italics
(like `mytable' in 
``\keyword{create table} \identifier{mytable}'').

\sql\ is a textual interface.
All statements that are passed to the database
have a textual form. The results produced
by the database, however, are not. They are
binary data which may or may not
contain textual elements.

In \nowdb\ \sql\ statements are strings
of \acronym{utf}-8 characters.
Keywords, identifiers and numbers, however,
must contain only characters
from the \acronym{ascii} subset.
Identifiers are further restricted:
They must start with an \acronym{ascii} 
Latin alphabetic ($a\dots z$ or $A \dots Z$)
and must contain only
alphanumerics or the underscore ($\_$).
Text, by contrast, may contain any
\acronym{utf}-8 character.

Keywords and identifiers are case-insensitive.
There is no difference between
`\acronym{select}', `select' or `Select'
and so on.
Text, however, is case-sensitive;
`hello world' and `hello World'
are not the same thing!

\sql\ is a \term{guest} language
that needs some kind of framework
to support it. One way to provide this
framework is the \nowdb\ client,
which provides two ``channels'' to execute
\sql\ statements in the database,
\ie\ by means of the \tech{-Q} parameter
and by means of standard input.

Another way is a host language
that provides means to pass \sql\ statements
to the database and means to receive
and interpret the results produced by such statements.
For \nowdb, Python, C, Go and Lua
are available as host languages.

The protocol that defines how data are exchanged
between the database and the host system
is not part of this specification.
Currently, native client and server libraries
exist that implement this protocol
without exposing it to the user.
To support open standards, such as
\acronym{odbc} and \acronym{jdbc},
parts of this protocol
will probably be documented
and published in the future

\section{Types}
\nowdb\ \sql\
has a very simple type system,
which is static and safe.
This type system is used to design
the database and to word
\sql\ statements.
We usually refer to it as the \sql\ Static Types.

However, since \sql\ is executed
in a host environment, there is a second
type system to describe the \term{results}
of \sql\ statements.
This other type system is even simpler --
it, in fact, consists of only one type.
This other type system, however, is dynamic.
It is therefore called \sql\ Dynamic Types.

In the following, we first present
the Static System and then the Dynamic System.

\subsection{Static Types}
The static types constitute
the \nowdb\ \sql\ type system in the strict sense.
The static types can be used in \sql\ statements.
Each type is equipped with a declaration form and
type constructors that create instances of this type.

The declaration form is used in \acronym{ddl} statements
to define types, edges, procedure and functions.
In \acronym{dml}, \acronym{dll} and \acronym{dql} statements,
instances of the types are used, \ie\
types are not explicitly declared, but used implicitly
by means of their constructors, which are sufficient
to determine the type uniquely.

In the case of numeric types
(integers, unsigned integers and floats),
\nowdb\ silently corrects type mismatches where possible.
An unsigned integer inserted into a field where
a signed integer is expected, is implicitly converted
to an integer; correspondingly a signed integer
is converted to an unsigned integer if possible.
If the unsigned integer is out of range or
the signed integer is negative, the statement
is rejected with a type error.

Likewise, signed or unsigned integers are converted to floats
if necessary (and possible) and a float might be converted
to an integer (or unsigned integer) if it actually
represents an integer.

The static types are

\ignore{
\bgroup
\renewcommand{\arraystretch}{1.3}
\begin{center}
\begin{tabular}{||c||c||c||c||c||}\hline
Type & Declaration & Range & Constructor & Examples \\\hline\hline
Integer & $int$, $integer$ & $-2^{63} \dots 2^{63}-1$ & $\pm n$ & $-1, +0, +1$\\
        & &                & where $n$ is an unsigned ingeger & \\\cline{1-5}
Unsigned Integer & $uint$, $uinteger$ & $0 \dots 2^{64}-1$ & One digit $(0\dots 9)$ & $0, 1,2, 1024$\\
        & &                & or one digit $(1\dots 9)$ & but not: $01$\\
        & &                & followed by a sequence of digits ($0\dots9$). & \\\cline{1-5}
\end{tabular}
\end{center}
\egroup
}

\begin{minipage}{\textwidth}
\textbf{Integer}\\
Declaration: $int$, $integer$ \\
Values: $-2^{63} \dots 2^{63}-1$ \\
Constructors: $\pm n$, where $n$ is an unsigned integer.\\
Examples: $-1, +0, +1$
\end{minipage}

\begin{minipage}{\textwidth}
\textbf{Unsigned Integer} \\
Declaration: $uint$, $uinteger$ \\
Values: $0 \dots 2^{64}-1$  \\
Constructors: One digit in the range $0\dots 9$
or one digit in range $1\dots 9$ followed by
a sequence of digits ($0\dots9$). \\
Examples: $0, 1, 2, 1024$, but not: $01$.
\end{minipage}

\begin{minipage}{\textwidth}
\textbf{Float} \\
Declaration: $float$ \\
Represents a \term{binary64} \acronym{ieee}-754 floating point number.
For possible values, please refer to the standard or to the table in
\url{https://en.wikipedia.org/wiki/IEEE\_754#Basic\_and\_interchange\_formats}.\\
Constructors: any integer followed by a dot and a sequence of digits,
              optionally followed by $e$ followed by an integer.
              \comment{The exponential form is not yet available.} \\
Examples: $-1.0, 0.0, 1.0, 3.14159, 1.797693e308$ 
\end{minipage}

\begin{minipage}{\textwidth}
\textbf{Time} \\
Declaration: $time$, $date$ \\
Values:  \acronym{utc} 1677-09-21T00:12:44 --
         \acronym{utc} 2262-04-11T23:47:16 \\
Precision: nanosecond. \\
Note, however, that range and precision depend on server configuration.
With less precision, a greater range can be reached.
Please refer to the database configuration guide. \\
Timezone: \acronym{utc} \\
Constructor: any integer or any string following \acronym{iso}-8601 
or any string following a locally defined time format. \\
Examples:\\
1535284617906179393, \\
'1940-12-21', \\
'1904-06-16T11:43:10', \\
'2011-11-11T11:11:11.123456789'
\end{minipage}

\begin{minipage}{\textwidth}
\textbf{Bool} \\
Declaration: $bool$ \\
Values: $true$, $false$
\comment{Currently, Bool cannot be stored in the database.
The future solution won't be to store single Booleans, anyway,
but bit patterns.}
\end{minipage}

\begin{minipage}{\textwidth}
\textbf{Text} \\
Declaration: $text$ \\
Values: \acronym{utf}-8 string with up to 255 bytes\\
Constructor: string enclosed by ' \\
Examples:\\
'hello world',\\
\begin{CJK}{UTF8}{gbsn}
'鎮州臨濟慧照禪師語錄序。' 
\end{CJK} \\
\comment{An important detail is not yet handled:
text that \emph{contains} the character '.
This is important for recursive \sql, \eg\ \\
\term{exec metaquery('select * from myedge 
       where a = $\backslash$'some text$\backslash$'')}}
\end{minipage}

\begin{minipage}{\textwidth}
\textbf{Longtext} \\
Declaration: text \\
Values: \acronym{utf}-8 string with up to 4096 bytes\\
\comment{Longtext is not yet available.}
\end{minipage}

\begin{minipage}{\textwidth}
\textbf{Blob} \\
\comment{Blob is nice to have, but there are currently
no concrete plans to add such a datatype.}
\end{minipage}

\begin{minipage}{\textwidth}
\textbf{NULL} \\
\acronym{null} is a special value available for all types.
It signals the absence of a value for the requested field.
For instance, when a vertex was inserted with a subset
of its attributes, a query on that vertex including
those attributes in the \term{select} clause
will produce \acronym{null} as value in the
columns related to these attributes.
Note that it is not possible to compare any value to \acronym{null}
using $= or \neq$. The special operator \keyword{is} must
be used for that purpose, \eg\
\keyword{where} \identifier{name} \keyword{is not null}.
\end{minipage}

\subsection{Dynamic Types}
Dynamic types are not used in \sql\ statements,
but rather describe the return values of \sql\
statements. As such, they live in the context
of a host language (C, Python, Lua, \etc) and
the concrete implementation
depends on that language.
For more information on concrete implementations
of dynamic types, please refer to the
host language \acronym{api} specifications.

\begin{minipage}{\textwidth}
\textbf{Status}\\
Values: \acronym{ok}, \acronym{nok}\\
The status should additionally provide the values
\begin{itemize}
\item error code
\item a detailed error message (only in case of error)
\end{itemize}
Error codes together with a brief description
of their meaning can be found in \ref{chpt_errors}.
\end{minipage}

\begin{minipage}{\textwidth}
\textbf{Report}\\
Reports consist of up to three values:
\begin{itemize}
\item number of affected rows
(returned by all \acronym{dml} and \acronym{dll} statements)
\item number of errors
(returned by all \acronym{dll} and some \acronym{dml} statements)
\item running time
(returned by most \acronym{dml} and \acronym{dll} statements)
\end{itemize}
\end{minipage}

\begin{minipage}{\textwidth}
\textbf{Row}\\
A row is the result of a projection;
it consists of an array of values
with type information called \term{fields}.
In the host language, one would access a field
typically by an expression of the form:
$row.field(i);$
which would return a tuple $(value,type)$.

A row result consists of one or many rows.
The host language shall provide means
to iterate over collection of rows.
\end{minipage}

\begin{minipage}{\textwidth}
\textbf{Cursor}\\
A cursor is an iterable collection of rows.
The iteration directive is \term{fetch}.
Each fetch may return one or many rows.
A cursor is a server-side resource
and shall be closed using the directive \term{close}.
\end{minipage}

\section{Expressions}
Some \sql\ clauses, like \term{select}, 
\term{where} or \term{insert},
allow the use of \term{expressions}.
Which expressions are valid 
in a specific context varies between
clauses.

The simplest expression,
which is available in all clauses
(that allow expressions),
is a constant value,
\eg\ $true, +4, 3.14159,$ `hello world', \etc\
In some clauses, such as \term{select},
field names are also valid expressions.

Expressions are evaluated when the
\sql\ statement is processed.
The result of the evaluation is
a value with a static type.
In the case of a constant value,
the result is just that constant value.
In the case of a field name,
the result is the value stored in
the corresponding field in the
row that is currently processed.

Expressions may be combined
with operators or functions
to form more complex expressions.
If $\circ$ is a binary operator
and $a$ and $b$ are valid expressions,
then
$a \circ b$
is also a valid expression.
In consequence, $a \circ b$ can
again be used with an operator
to form an even more complex
expression:
$a \circ b \circ c$.
Examples of binary operators are
$+, -, /$ and $\times$.

To enforce a specific order of
evaluation, parentheses may be used.
So, if $a$ is a valid expression,
$(a)$ is too; for instance
$(a+b)/2$
is a valid expression.

Besides binary operators, there are also
unary operators, \ie\ operators that
take only one operand. Those operators
are usually prefixes. An example of
a unary operator is $not$:
if $a$ is a valid expression,
so is $not~a$.

Expressions may also contain
functions. Functions are very
similar to operators, but differ
in syntax. Functions have the
form

\identifier{function}(
\identifier{argument},
\identifier{argument},
$\dots$),

where \term{function} is the name
of the function and the \term{arguments}
are again expressions.
If $\Phi$ is a function
that takes one argument and
$a$ is a valid expression,
then $\Phi(a)$ is a valid expression too.
An example of a function is $log$,
which computes the natural logarithm
for a number $a$. Valid expressions,
hence, are $log(a)$, $log(a)/log(2)$ and
$log(2.718281828)$.

The validity of an expression depends
also on types. Operators and functions
expect arguments of certain types.
Some operators and functions are limited
to a specific type or type family,
\eg\ to numeric types,
to Booleans or to text; when applied
to values of types they do not expect,
the expression evaluates to a type error
and the related \sql\ statement fails.

In some cases, arguments are \term{promoted}
to other types. Binary operators, for instance,
usually expect both operands to be of the same type.
If one differs from the other, one of them
may be promoted.

Promotion for numerical types follows 
the precedence:
$\term{float} > \term{int} > \term{uint}$.
In the following example, we try to
add a \term{float} and an \term{int}:
$3.14159 + 1$. In this case, the \term{int}
is promoted to a \term{float} and the
result is a \term{float},
namely $4.14159$.

In general, if at least one of the
operands or arguments is a \term{float},
all values are promoted to \term{float}.
Likewise, if at least one of the operands
is an \term{int} (but none of them is a \term{float}),
then all others are promoted to \term{int}.
Non-numerical types (like \term{bool} and \term{text}),
however, are never promoted to another type.

\subsection{Operators and Functions}
\subsubsection{Arithmetic}
Arithmetic operators and functions
accept only numerical values.
If not mentioned otherwise,
promotion follows precedence as described above.
The basic arithmetic operators are
\begin{verbatim}
+, -, *, /, %
\end{verbatim}
The first three of them have the obvious meaning.
\comment{There is an annoying ambiguity in the
tokeniser affecting $+$ and $-$. The parser
confuses \tech{+b} in \tech{a+b} with the pattern for
integers. Workaround is to either introduce
a blank before \tech{b}: \tech{a+ b} or to put parentheses
around it: \tech{a+(b)}.}

Division (/), however, is different in that it
has different meanings for different numerical types.
For \term{float} values it performs floating point
division. For integral values (type \term{int} and
\term{uint}), however, it computes the quotient
of the Euclidean division, \eg\ $5/2$ is $2$.
However, $5.0/2$ is $2.5$, because $2$ is promoted
to \term{float} according to the precedence rule.

The fifth operator (\%) is the remainder of
the Euclidean division.
It operates only on integral values and results
in a type error if applied to \term{floats}.

It should be noted that this is not the modulo
operator ($mod$) with which the remainder is
often confused. People often expect the 
remainder to always return a non-negative result.
But that is not the case. The remainder is the
unique number $r$ for which

$a = qb + r, r < b$,

where $q$ is the quotient $a/b$.

In consequence, if $a$ is negative and $b$
is positive, then the quotient and the 
remainder are negative, \eg\
$-7 = -2\times 3 -1$ for $a=-7$ and $b=3$.

If $a$ is positive and $b$ is negative, then
the quotient is negative but the remainder
is positive, \eg\
$7 = -2\times -3 + 1$.

If both, $a$ and $b$, are negative,
then the quotient is positive and 
the remainder is negative, \eg\
$-7 = 2\times -3 -1$.

\paragraph{Power}
The power operator is $^{\wedge}$.
It always promotes all its arguments to \term{float}
and always returns a \term{float}.

Please notice that there is no \term{root} operator.
Instead, the root is computed by raising to the power
of a fraction, \eg:

$\sqrt[2]{n} = n^{\wedge}{(1/2)}$,\\
$\sqrt[3]{n} = n^{\wedge}{(1/3)}$,\\
$\sqrt[4]{n} = n^{\wedge}{(1/4)}$,\\
$\dots$

\paragraph{Log}
As mentioned, the function $log$
implements the natural logarithm.
The function promotes its argument to \term{float}
and always returns a \term{float}.

The natural logarithm finds $x$ in the equation
$n=e^{x}$ for a given $n$, where $e$ is the
\term{Euler-Napier} constant, which is
$\approx 2.718281828$.

Logarithms to any base other than $e$
can be easily found by the formula
$\log(n)/\log(b)$, where $b$ is the
desired base. The binary logarithm
of $n$, for instance, is
$\log(n)/\log(2)$.

\paragraph{Absolute Value}
The function $abs$ computes the
absolute value of its argument, \eg:
$abs(-1)$ is 1.
The function accepts any numerical type
and returns a value of the same type as the argument.

\paragraph{Rounding}
The rounding functions are $round$, $ceil$ and $floor$.
All of them promote their argument to \term{float}
and always return a \term{float} value.

$round(n)$ returns the integer closest to $n$,
which may be $n$ itself; if the next greater and
next smaller integer are equally far away,
\ie\ if the digital part of $n$ is $.5$,
the result is the next greater integer.

$ceil(n)$ returns $n$ if $n$ is an integer or,
otherwise, the next integer;
\ie\ it always rounds up, \eg: $ceil(1.1)$ is 2.

$floor(n)$ returns either $n$ if $n$ is an integer
or, otherwise, the previous integer, \ie\ it always rounds down;
\eg: $floor(1.9)$ is 1.

\subsubsection{Trigonometry}
\nowdb\ provides the trigonometry functions
\identifier{sin} ($sine$),
\identifier{cos} ($cosine$) and
\identifier{tan} ($tangent$)
as well as their inverses
\identifier{asin} ($arcsine$),
\identifier{acos} ($arccosine$) and
\identifier{atan} ($arctangent$) and
the corresponding hyperbolic functions
\identifier{sinh} ($hyperbolic\ sine$),
\identifier{cosh} ($hyperbolic\ cosine$) and
\identifier{tanh} ($hyperbolic\ tangent$)
as well as their inverses
\identifier{asinh} ($hyperbolic\ arcsine$),
\identifier{acosh} ($hyperbolic\ arccosine$) and
\identifier{atanh} ($hyperbolic\ arctangent$).

Example:

\keyword{select} \identifier{sin}(1)
\keyword{from} \identifier{mytable}

results in (approximately) 0.84147098.

All these functions expect a \term{float} argument
and also return a \term{float}.
If the argument is of a numeric type other than \term{float},
the argument is promoted to \term{float} if possible.
The argument is expected to be in units of radians.

The return value of the trigonometric functions
\identifier{sin} and \identifier{cos} is in the range $-1\dots 1$.

The tangent function has singularities at odd multiples of $\pi/2$.
If the argument is too close to one of these singularities,
the result is undefined.

The inverse trigonometric functions
\identifier{asin} and \identifier{acos}
are defined over the domain $-1\dots 1$.
For arguments outside of that domain,
\identifier{asin} and \identifier{acos}
return $NaN$ (\term{not a number}, see \acronym{ieee}-754).
The result of all inverse trigonometric functions
is in the range $-\pi/2\dots\pi/2$.

The hyperbolic function \identifier{acosh}
is valid only for values $\ge 1$.
For smaller values, \identifier{acosh} returns $NaN$.

The hyperbolic function \identifier{atanh}
is valid only for values $\le 1$.
For 1, it returns $\infty$.
For greater values, it returns $NaN$.

\subsubsection{Mathematical Constants}
\nowdb\ defines the mathematical constants
$\pi$ and $e$ (the Euler-Napier number) as
functions \identifier{pi} and, respectively, \identifier{e}.
Both expect no arguments and return a float value.
Example:

\keyword{select} \identifier{pi}(), \identifier{e}()
\keyword{from} \identifier{mytable}

returns (approximately)
3.1415926 and
2.7182818.

\subsubsection{Conversions}
\nowdb\ provides explicit conversion functions
between numerical types. The conversion functions
are   
$tofloat, toint$ and $touint$.
If conversion is applied according to precedence,
the respective function has the obvious result,
\ie\ converting upwards preserves the full value
if possible.

$tofloat(1)$, for instance, is $1.0$ and
$toint(1)$ is $1$.
$toint(18446744073709551615)$, however, is $-1$.
This is because $18446744073709551615$ is
$2^{64}-1$ and, hence, out of range of \term{int}.
Similar effects happen with applications of
$tofloat$ on values out of range.
Please refer to the 
\acronym{ieee}-754 specification for details.

Converting against precedence,
\ie\ converting downwards,
does not always preserve the value.
When converting a \term{float}
value to an integral type, the decimal part
is lost, \eg\ $toint(3.14159)$ is 3.

Converting negative numbers to \term{uint},
does not preserve the value, but the bit pattern.
$touint(-1)$, for instance, is
$18446744073709551615$.

There is also a way to create a textual
representation of a number, namely $totext$.
This function always produces a textual
representation of the constructor for that
number expression,
\eg\ $totext(1.0)$ is '1.0',
$totext(1)$ is '1' and
$totext(+1)$ is '+1'.
\comment{totext is not yet available.
Further missing are the inverse functions
parsefloat, parseint, parseuint, parsebool.}

\subsubsection{Boolean}
Boolean operators and functions
are those resulting in a Boolean value.
They are, hence, not defined by the type
of their arguments, but by their result type.

There are different categories of Boolean
operators (or functions):
boolean operators in the strict sense,
comparisons and special operators.

\paragraph{Boolean Operators in the Strict Sense}
are \keyword{and}, \keyword{or} and \keyword{not}.
What makes them strict Boolean operators is
that their operands must be Boolean values.
Boolean operators can be used in all clauses
where expressions are allowed.
But they are especially important for the
\term{where} clause and will be discussed
in more detail there (please refer to section
\ref{subsec_where}).

\paragraph{Comparisons}
compare values and return
a Boolean value according to the comparison.
Some comparisons are restricted in terms
of the types of their arguments;
others are unlimited. All comparisons, however,
always compare values of the same type.
Comparing incompatible values leads to a type error.
For numeric types, the same promotion rules
as for arithmetic operators apply.

The main comparison operators are
\begin{verbatim}
= < > <= >= != <>
\end{verbatim}
with the obvious meaning for
the first five operators;
The last two operators
are synonym and both implement
the mathematical operator $\neq$.

The operators $=$ and $\neq$ can be used
with any type. $=$ evaluates to $true$
if the values are equal, \ie\
their type is equal (or compatible
via promotion) and their value
is equal. For instance,
$1 = 1$ is $true$;
$1 = 1.0$, after promotion to \term{float},
is also true; but
$1 =$ `1' results in a type error.

Furthermore, `hello' $=$ `hello'
is $true$, while `Hello' $=$ `hello'
is $false$.

Finally, $true = true$ and
$false = false$ are $true$, while
$true = false$ and $false = true$ are
$false$.

The operator $\neq$ is $true$,
if the operands are not equal.
$\neq$, hence, is the negation
of $=$. The expression $a\neq b$ is equivalent to
$not (a=b)$

The \term{inequality} operators
$<,>,\le$ and $\ge$ can only be applied
to numerical types (and result in a
type error otherwise).

All comparison operators return $false$
when applied on the special value \acronym{null}.
That is \acronym{null} is neither equal nor
not equal to any other value. It is also
not greater or less than any other value.
It is incomparable. 

\paragraph{Special Operators}
are similar to comparison operators
but with peculiar operands.
Special operators are
\keyword{is},
\keyword{between},
\keyword{in},
\keyword{having},
\keyword{exists},
\keyword{any} and
\keyword{all}.

\comment{
\keyword{between},
\keyword{having}, 
\keyword{exists},
\keyword{any} and \keyword{all}
are not yet available.}

The operator \keyword{is} must be used
to compare with the special value
\acronym{null}, \eg\ 1 \keyword{is null}
(which is $false$) 
or 1 \keyword{is not null}
(which is $true$).
The expression
$a$ \keyword{is not null} is
equivalent to
\keyword{not}($a$ \keyword{is null}).

The operator \keyword{between} tests
whether a value is in a range.
Its first operand is an expression,
while its second operand is a range,
for instance
1 \keyword{between} $[1,2]$,
which evaluates to $true$,
since \keyword{between} in this form is
\term{inclusive}.

\keyword{between} has a special syntax
for the brackets to indicate
ex- or inclusiveness:
1 \keyword{between} $]1,2]$,
which would evaluate to $false$
because, in this form, the first
element is excluded from the range.
Correspondingly,
\keyword{between} $[1,2[$
would exclude the last element from the range
and
\keyword{between} $]1,2[$
would exclude both.

\keyword{between} is especially interesting
with time periods. We could, for instance,
select all data of the first three months
of year 2010 with the expression:
\keyword{stamp} \keyword{between}
[`2010-01-01', `2010-04-01'[.

The operator \keyword{in} tests
whether a value is in a set.
Its first operand is an expression,
while its second operand is a list
of expressions enclosed by parentheses,
\eg\ 1 \keyword{in} $(1,2,3)$,
which is $true$, or
1 \keyword{in} $(2,3,4)$,
which is $false$.

In a \term{where} clause
the second operator of the \keyword{in} operator
may be a \acronym{dql} statement, \eg\
1 \keyword{in}
(\keyword{select} 1 \keyword{from} \identifier{sales}).
\acronym{dql} statements may also be used
for the second operand of the the operators
\keyword{exists}, \keyword{any} and \keyword{all}.

\keyword{exists} tests whether a set contains
data (\ie\ is not the empty set).
The basic form is \keyword{exists} $(1,2,3)$,
which is $true$.
\keyword{exists} is especially useful with subqueries.

\subsubsection{Conditionals}
\paragraph{The Case Expression}
Conditionals can be constructed by means
of the \keyword{case} expression.
The general syntax is:

\keyword{case}\\
\hspace*{1cm}\keyword{when} \textit{condition$_1$} 
\keyword{then} \textit{value$_1$} \\
\hspace*{1cm}\keyword{when} \textit{condition$_2$} 
\keyword{then} \textit{value$_2$} \\
\hspace*{1cm} $\dots$\\
\hspace*{1cm}\keyword{else} \textit{value$_n$} \\
\keyword{end}

where the conditions are boolean expressions
and the values are any expressions.
For instance:

\keyword{case}\\
\hspace*{1cm}\keyword{when} \identifier{price} $>$ 100.0
\keyword{then} 'expensive' \\
\hspace*{1cm}\keyword{when} \identifier{price} $>$ 50.0
\keyword{then} 'not cheap' \\
\hspace*{1cm}\keyword{when} \identifier{price} $>$ 20.0
\keyword{then} 'considerable' \\
\hspace*{1cm}\keyword{when} \identifier{price} $>$ 5.0
\keyword{then} 'not for free' \\
\hspace*{1cm}\keyword{else} 'acceptable' \\
\keyword{end}

Note that the whole construct enclosed by
the keywords \keyword{case} and \keyword{end}
is an expression. \keyword{when}, \keyword{then}
and \keyword{else} alone do not form expressions;
they can only be used within a \keyword{case}.

A \keyword{case} expression consists of at least
one \keyword{when}/\keyword{then} construct.
Without such a construct, the \keyword{case}
is invalid.

Likewise, a \keyword{case} must contain at most
one \keyword{else}
and that \keyword{else}
must be placed behind all \keyword{when} constructs.
With more than one \keyword{else}
or with an \keyword{else} that is followed
by one or more \keyword{when} constructs,
the \keyword{case} is invalid.

The \keyword{when} constructs are evaluated
in the order in which they appear in the \keyword{case} expression.
The \keyword{case}, thus, evaluates
to the \term{value} branch of the first \keyword{when}
whose condition branch evaluates to \keyword{true}.

If none of the \keyword{when} conditions evaluates to \keyword{true},
the \keyword{case} expression evaluates
to the \keyword{else} \term{value}.
If there is no \keyword{else},
the \keyword{case} expression
evaluates to \keyword{null}.

The \keyword{when} and \keyword{else} values
may be of different types.
For instance:

\begin{minipage}{\textwidth}
\keyword{case}\\
\hspace*{1cm}\keyword{when} \identifier{category} $=$ 1
\keyword{then} 'true' \\
\hspace*{1cm}\keyword{when} \identifier{category} $=$ 2
\keyword{then} 1 \\
\hspace*{1cm}\keyword{when} \identifier{category} $=$ 3
\keyword{then} true \\
\hspace*{1cm}\keyword{else} \keyword{null} \\
\keyword{end}
\end{minipage}

However, care must be taken when using this feature.
Often the context where the \keyword{case} expression
appears requires a certain type, \eg: 

\begin{minipage}{\textwidth}
\keyword{select} 1 +
\keyword{case}\\
\hspace*{2.5cm}\keyword{when} \identifier{category} \keyword{is null}
\keyword{then} 0 \\
\hspace*{2.5cm}\keyword{else} \identifier{price} + category * 0.1 \\
\hspace*{2cm}\keyword{end}
\end{minipage}

This context expects a numerical value.
Note that, if the first \keyword{when} is \keyword{true},
the whole expression will produce a \term{uint};
otherwise, it will produce a \term{float}.
Evaluation will fail with a non-numerical value
like in the following case

\begin{minipage}{\textwidth}
\keyword{select} 1 +
\keyword{case}\\
\hspace*{2.5cm}\keyword{when} \identifier{category} \keyword{is null}
\keyword{then} \keyword{false} \\
\hspace*{2.5cm}\keyword{else} \identifier{price} + category * 0.1 \\
\hspace*{2cm}\keyword{end}
\end{minipage}

This is of special importance in the \keyword{where} clause.
When a \keyword{case} expression is used in a \keyword{where}
like this:

\keyword{where} \keyword{case} $\dots$ \keyword{end}

\ie\ the whole \keyword{where} evaluates to the result
of the \keyword{case} expression, the \keyword{case}
must evaluate to a boolean. Otherwise the effect of
the \keyword{where} clause is undefined.

\paragraph{Coalesce}
\identifier{Coalesce} is a function that accepts
an unspecified number arguments.
\identifier{Coalesce} evaluates to
the first argument that itself
does not evaluate to \keyword{null}.
If none of the arguments is \keyword{not null},
\identifier{coalesce} evaluates to \keyword{null}.
Example:

\identifier{coalesce}(\identifier{category}, 99)

If the field \identifier{category} is \keyword{null},
\identifier{coalesce} evaluates to 99;
otherwise, it evaluates to the current
value of \identifier{category}.

Since \identifier{coalesce} accepts
an undetermined number of arguments,
the following expression is legal:

\identifier{coalesce}(\identifier{species},
                      \identifier{genus},
                      \identifier{family},
                      \identifier{order},
                      \identifier{class},
                      \identifier{phylum},
                      \identifier{kingdom},
                      \identifier{domain},
                      'life')

The following expression evaluates to \keyword{null},
whenever \identifier{category} evaluates to \keyword{null}
(and is as such pointless):

\identifier{coalesce}(\identifier{category}, \keyword{null})

By contrast, this statement always evaluates to 99:

\identifier{coalesce}(\keyword{null}, 99)

The arguments may be of different types.
But, again, care must be taken when using this feature.

\subsubsection{Bitwise Operators}
$\ll, \gg, \&, |,\sim$, xor, bit

\comment{Not yet available.
Bitwise operators, however, are very interesting
to store bools as bitmaps.}

\subsubsection{Time}
\paragraph{Time Component Functions}
Time component functions return a part of the time,
such as the year, the month, the day of the month,
\etc\ as \term{int} according to the European calendar.
Example:

\keyword{select} \identifier{year}(\keyword{stamp})
\keyword{from} \identifier{buys}

\identifier{year} returns the year, \eg\ 2018.

\identifier{month}
returns the month starting from January as 1.

\identifier{mday}
returns the day of the month (1-31).

\identifier{wday}
returns the day of the week with
Monday = 1, Tuesday = 2, $\dots$, Saturday = 6 and
Sunday = 0.

\identifier{yday}
returns the day of the year.

\identifier{hour}
returns the hour of the day from $0\dots 23$.

\identifier{minute}
returns the minute of the hour from $0\dots 59$.

\identifier{second}
returns the second of the minute,
which is almost always in the range $0\dots 59$.
In case of leap seconds, however, 60 may be returned.

\identifier{milli}
returns the milliseconds within the second.

\identifier{micro}
returns the microseconds within the second.

\identifier{nano} 
returns the nanoseconds within the second.

\paragraph{Points in Time}
The following functions return specific points in time.
They take no argument and return a \keyword{time} value.
Example:

\keyword{select} \identifier{now}()

\identifier{now}
returns the current system time.

\identifier{dawn, dusk}
return the earliest (\identifier{dawn}) and
the latest (\identifier{dusk}) point in time
that can be represented.

\identifier{epoch}
returns the system's epoch
(usually `1970-01-01T00:00:00').

\paragraph{Time Formatting}
\comment{Not yet available}

\subsubsection{Geospatial}
\comment{
Not yet available.
Planned are:
geohashing,
distance calculation,
bounding boxes,
box is in another box,
boxes share area,
boxes touch each other,
etc.
}

\subsubsection{Text}
\comment{Not yet available}

\subsection{Aggregates}\label{sec_agg}
Aggregates are functions (and, hence, may be part
of expressions) with very special behaviour and
very special restrictions concerning their usage
context.
While all functions (and operators) we have looked at
so far, are applied to \emph{single} values,
aggregates are applied
to \emph{sets} of values. They produce,
as their name suggests, an aggregation of the
values in the set like the number of elements
in the set (\identifier{count}),
the sum of all the values in the set
(\identifier{sum}) or the average (\identifier{avg})
or the median (\identifier{median}).
They, hence, behave like \term{map} and
\term{reduce} operators.

The set of values to which an aggregate is applied
depends on its usage. Aggregates may be applied
to all data in the result set or to all data
in a \term{group} (please refer to section \ref{sec_group}
for details).

The aggregate arguments are expressions.
There may be restrictions
on the type of expression that can be used
with a particular aggregate.
In general, however, aggregates may
be applied to any kind of expression.
For instance,
\identifier{sum}(1) is equivalent to
\identifier{count}($\ast$) and
\identifier{avg}(\identifier{price}$^{\wedge}2$) would compute
the average of the squares of the values
of field \identifier{price}.

Aggregates can also be part of expressions.
The formula 
\identifier{sum}(\identifier{weight})
/ \identifier{count}($\ast$),
for instance, would be equivalent to
\identifier{avg}(\identifier{weight}).

\subsubsection{count}
The function counts the values in a set.
It takes one argument, which may be any expression,
but few expressions really make sense with \identifier{count}.

The function is applied either to the row (\identifier{count}($\ast$))
or to a part of it (\identifier{count}(\identifier{field}))
or to anything else (\identifier{count}(1)).
However, these expressions are all equivalent.
 
When applied to a field, \identifier{count} is usually combined with
\keyword{distinct} (\comment{which is not yet available}),
otherwise the result would be the same as applying it to the row.
Common examples are therefore:

\keyword{select} \identifier{count}($\ast$) \keyword{from} \identifier{buys}

and

\keyword{select} \identifier{count}(\keyword{distinct} \keyword{origin})
\keyword{from} \identifier{buys}

The return type is always \keyword{uint}.

\subsubsection{sum}
The function takes one argument and
produces the sum of the values
to which its argument evaluates over the result set,
\eg\ the values in a field.

The argument must be numeric.
The return type depends on the type of the input field.

Example:

\keyword{select} \identifier{sum}(\identifier{quantity})
  \keyword{from} \identifier{buys}

produces the sum of the values in the \identifier{quantity} field
for the whole table \identifier{buys}.

\subsubsection{max and min}
The functions take one argument.
Function \term{max} produces the greatest value in the result set
and function \term{min} produces the smallest value in the result set.

The arguments must be numeric.
The return type depends on the type of the input field.
Example:

\keyword{select} \identifier{max}(\identifier{price}),
                 \identifier{min}(\identifier{price})
\keyword{from} \identifier{buys}

\subsubsection{spread}
The function takes one argument.
It produces the difference $max - min$,
where $max$ is the greatest value in the result set
and $min$ is the smallest value in the result set.

The argument must be numeric.
The return type depends on the type of the input field.

Example:

\keyword{select} \identifier{spread}(\identifier{price})
\keyword{from} \identifier{buys}

\subsubsection{avg}
The Function takes one argument.
It produces the arithmetic mean of the values
of this field in the result set, \ie

\[
\left(\sum{x}\right)/n,
\]

where $x$ represents the
field values in the result set and $n$
is the number of rows (\ie\ \identifier{count}($\ast$)).

The argument must be numeric.
The return type is \keyword{float}.

Example:

\keyword{select} \identifier{avg}(\identifier{price})
\keyword{from} \identifier{buys}

\subsubsection{median}
The Function takes one argument.
It finds the central point (or, in case
the number of rows is even, the average
of the two central points), when the values are ordered.

The argument must be numeric.
The return type is \keyword{float}.

Example:

\keyword{select} \identifier{median}(\identifier{price})
\keyword{from} \identifier{buys}

\comment{
No precaution is currently taken for the case
that the result set outgrows available memory.
In that case the query will fail (and probably the queries
in other sessions too).
}

\subsubsection{mode}
The Function takes one argument.
It finds the most frequent value in the result set.

The argument may be of any type.
The return type depends on the type of the input field.

\keyword{select} \identifier{mode}(\identifier{price})
\keyword{from} \identifier{buys}

\comment{
Not yet available
}

\subsubsection{stddev}
The Function takes one argument.
It finds the standard deviation according to the formula:

\[
\sqrt{\frac{\sum_{i=1}^{N}{(x_i - \overline{x})^2}}{N-1}},
\]

where $N$ is the number of rows in the result set,
$x_i$ is the $i$th element and $\overline{x}$ is
the arithmetic mean (\ie\ \identifier{avg}).

The argument must be numeric.
The return type is \keyword{float}.

\keyword{select} \identifier{stddev}(\identifier{weight})
\keyword{from} \identifier{sales}

\comment{
Note that there is not much optimisation done in aggregates.
For instance, \term{median} and \term{stddev} create a
collection of all values, but each one creates its own collection.
That is, the values are collected twice.
}

\ignore{
\subsubsection{integral}
The Function takes two arguments.
It computes the area under the curve,
where the first argument represents
the values on the $x$-axis and
the second represents the values
on the $y$-axis.

The first argument can be any type;
the second argument must be numeric.
The return type is \keyword{float}

Example:

\keyword{select} \identifier{integral}(
\keyword{stamp}, \identifier{weight})
\keyword{from} \identifier{sales}
}

\section{Data Definition}
\subsection{Schema}
The keywords
\term{schema}, \term{database} and \term{scope}
are interchangeable.

\subsubsection{CREATE}
The \term{create schema} statement
creates an empty database physically on disk.
It has the following form:

\keyword{create schema} \identifier{mydb}

This would create all objects necessary
to manage that database.

The following forms are equivalent:

\keyword{create database} \identifier{mydb}\\
\keyword{create scope} \identifier{mydb}

All \keyword{create} clauses can be combined
with the clause \keyword{if not exists}, \eg:

\keyword{create schema} \identifier{mydb} \keyword{if not exists}

The \keyword{if not exists}-clause
suppresses the `duplicate key' error
in case the schema
already exists.
It is a convenient way to avoid
that a \sql\ script is abandoned
in such a situation.

\subsubsection{DROP}
The \term{drop schema} statement
removes a database physically from disk.
It has one of the following forms,
which are all equivalent:

\keyword{drop schema} \identifier{mydb}\\
\keyword{drop database} \identifier{mydb}\\
\keyword{drop scope} \identifier{mydb}

The statement removes all objects and data
belonging to the database `mydb' from disk.

All \keyword{drop} clauses can be combined
with the clause \keyword{if exists}, \eg:

\keyword{drop schema} \identifier{mydb} \keyword{if exists}

The \keyword{if exists}-clause
suppresses the `key not found' error
in case the schema
does not exist.
It is a convenient way to avoid
that a \sql\ script is abandoned
in such a situation.

% \subsubsection{ALTER}

\subsection{Storage}
\subsubsection{CREATE}
The \term{create storage} statement
creates a new storage entity for edges and types
physically on disk.

The simplest form is:

\keyword{create storage} \identifier{mystorage}

The keyword \keyword{storage} may be decorated
with a sizing option:

\keyword{create big storage} \identifier{mystorage}

Valid sizing keywords are:
\keyword{tiny, small, medium, big, large, huge}.

Sizing keywords affect the allocation unit
for disk space. The concrete meaning is not
part of this specification and may change
in the future.

It is also possible to add options 
to a \term{create storage} statement.
Options have the general form

\term{\keyword{set} option = value, option = value}.

Valid options and values are listed in the following table:

\bgroup
\renewcommand{\arraystretch}{1.3}
\begin{center}
\begin{tabular}{||c||c||c||c||}\hline
Option & Values & Meaning & Default \\\hline\hline
\keyword{stress} & \keyword{moderate} & Low ingestion volume with occasional peaks & X \\\cline{2-4}
                 & \keyword{constant} & Constant ingestion of high volume          &   \\\cline{2-4}
                 & \keyword{insane} & Constant ingestion of very high volume       &   \\\hline\hline
\keyword{disk} & \keyword{hdd}  & Disk space is allocated in large chunks          & X \\\cline{2-4}
               & \keyword{ssd}  & Disk space is allocated in small chunks          &   \\\cline{2-4}
               & \keyword{raid} & Currently, no effect                             &   \\\hline\hline
\keyword{compression} & 'zstd'  & zstd is used for compression                     & X \\\cline{2-4}
                      & 'lz4'   & lz4 is used for compression (\comment{not available}) &   \\\cline{2-4}
                      & ''      & Data in this table are not compressed at all     &   \\\cline{1-4}
\end{tabular}
\end{center}
\egroup

The option \keyword{stress} affects the number of threads
allocated to perform ingestion tasks
like compression, sorting and indexing.
How many threads are allocated
is not part of this specification
and may vary between platforms.

The user may decide on the compression algorithm to use.
The standard compression algorithm is \term{zstd},
which is fast, but also has a very good compression ratio.
It is recommended to use \term{zstd} in most cases.
\term{lz4} (\comment{not yet available}) is faster than \term{zstd},
in particular on decompression,
but has a weaker compression ratio.
Finally, no compression at all (empty string)
makes sense on small tables
that are known never to grow beyond some megabyte in size
(or beyond some million edges or vertices).

An example of a \term{create storage} statement with options is

\keyword{create storage} \identifier{mystorage}
\keyword{set} \keyword{stress} = \keyword{constant},
              \keyword{compression} = 'zstd'

\subsubsection{DROP}
The \term{drop storage} statement removes
an existing storage entity
physically from disk.
It has the form:

\keyword{drop storage} \identifier{mystorage}

% \subsubsection{ALTER}

\subsubsection{SHOW, DESCRIBE}
The command \keyword{show storages}
returns a cursor with the names of all
storages defined in the current database,
one storage name per row.

The command \keyword{describe} \identifier{mystore}
returns a cursor with all options set for the storage
\identifier{mystore}, one option name and value
per row.

\subsection{Type}
\term{Types} are user-defined vertex types.
The syntax resembles very much
the \term{create table} syntax
in traditional \sql:

\begin{minipage}{\textwidth}
\keyword{create type} \identifier{product} ( \\
\hspace*{1cm}\identifier{prod\_key} \keyword{uint} \keyword{primary key}, \\
\hspace*{1cm}\identifier{prod\_desc} \keyword{text}, \\
\hspace*{1cm}\identifier{prod\_price} \keyword{float})
\end{minipage}

This creates a type called \term{product}
with three attributes:
\term{prod\_key}, \term{prod\_desc} and \term{prod\_price}.

There is no limit on the number of attributes
a type may have. Indeed, vertices with hundreds
of attributes are not uncommon.

Attributes may have any static type with one exception:
Each type needs a unique primary key and the field
that is primary key must be either
\keyword{uint} or \keyword{text}.

The order in which attributes are declared
determines the \term{canonical order} for this type.

The \keyword{create type} statement has an optional
\keyword{storage} clause of the form

\keyword{storage} = \identifier{mystorage}.

The \keyword{storage} clause comes after the
attribute definition, \eg\

\begin{minipage}{\textwidth}
\keyword{create type} \identifier{client} ( \\
\hspace*{1cm}\identifier{client\_key} \keyword{uint} \keyword{primary key}, \\
\hspace*{1cm}\identifier{client\_name} \keyword{text} \\
) \keyword{storage} = \identifier{client\_store}
\end{minipage}

The storage clause defines the storage
that manages this specific vertex type.
If no storage clause is given, the vertex type
is managed by the default storage for vertices.

\subsubsection{DROP}
The \term{drop type} statement removes a type
and all its vertices
from the database and physically from disk.
It has the form:

\keyword{drop type} \identifier{product}

\subsubsection{SHOW, DESCRIBE}
The command \keyword{show types}
returns a cursor with the names of all
types defined in the current database,
one type name per row.

\keyword{describe} \identifier{mytype}
returns a cursor with the attributes
and attribute types for the vertex type
\identifier{mytype}, one pair of
attribute name, type per row.

\subsection{Edge}
\subsubsection{CREATE}
The \term{create edge} statement defines the layout
of a specific edge type in the database.

There are two variants of the \keyword{create edge} clause.
The simple variant defines a simple edge, which must 
have precisely two fields:
\keyword{origin} and \keyword{destin}, \eg:

\keyword{create edge} \identifier{prod\_category} ( \\
\hspace*{1cm}\keyword{origin} \identifier{product}, \\
\hspace*{1cm}\keyword{destin} \identifier{category})

The types of \keyword{origin} and \keyword{destin}
are vertex types, since they refer
to vertices which are connected by this specific edge.
Stored in the database is the value of the primary key
of the specific type (\eg\ \keyword{uint} for product).

Edges can also be \term{stamped}.
Stamped edges have additionally to the fields
\keyword{origin} and \keyword{destin}
a field \keyword{timestamp} of type \term{time}.
This field does not need to and must not be defined.
It is always the third field of a stamped edge.

A stamped edge can additionally have up to 99
user-defined fields of any static type.
Here is an example:

\keyword{create stamped edge} \identifier{buys} ( \\
\hspace*{1cm} \keyword{origin} \identifier{client}, \\
\hspace*{1cm} \keyword{destin} \identifier{product}, \\
\hspace*{1cm} \identifier{quantity} \keyword{uint}, \\
\hspace*{1cm} \identifier{paid} \keyword{float})

This edge has five fields (in canonical order):
origin, destin, stamp, quantity and paid. 

It is possible to rename the fields origin and destin
according to their real purpose in a concrete
application (\eg\ naming \identifier{origin} \identifier{client}).
This is done by adding an \keyword{as} clause
to the attribute definition, \eg:

\keyword{origin} \identifier{client} \keyword{as} \identifier{client}

The corresponding field can then be referenced as either
\keyword{origin} or \identifier{client}.
\comment{Not yet available!}

The \keyword{create edge} and \keyword{create stamped edge}
statements have an optional
\keyword{storage} clause of the form

\keyword{storage} = \identifier{mystorage}.

The \keyword{storage} clause comes after the attribute definitions.

It defines the storage
that manages this specific edge type.
If no storage clause is given, the edge
is managed by the default storage for edges.

\subsubsection{DROP}
The \term{drop edge} statement removes the edge
(definition and data)
from the database.
It has the form (for both simple and stamped edges):

\keyword{drop edge} \identifier{buys}

\subsubsection{SHOW, DESCRIBE}
The command \keyword{show edges}
returns a cursor with the names of all
edges (simple and stamped)
defined in the current database,
one edge name per row.

\keyword{describe} \identifier{myedge}
returns a cursor with the attributes
and attribute types for the edge 
\identifier{myedge}, one pair of
attribute name, type per row.

\subsection{Index}
\subsubsection{CREATE}
The \term{create index} statement
creates an index physically on disk.
It has the form:

\keyword{create index} \identifier{myidx} \keyword{on} \identifier{mytable}
(\identifier{field1}, \identifier{field2})

where \identifier{mytable} may be a vertex type or an edge.
The fields (``field1'', ``field2'', \etc)
are user-defined fields or edge fields.
Any combination of fields
can be used in index definitions.
It is not recommended, however,
to create indices on fields that are defined as \keyword{float}
or \keyword{time} (\eg\ \keyword{stamp}).

\comment{
There will be the possibility to define indices over ranges
of \keyword{float} fields and periods
of \keyword{time} fields. But that is not yet available.
}

\nowdb\, internally, creates some standard indices,
namely on the primary key of vertices and on origin
and destin of edges.

The \keyword{index} keyword can be decorated with a sizing
indication, \eg:

\keyword{create tiny index} \identifier{myidx} \keyword{on} \identifier{mytable}
(\identifier{field1}, \identifier{field2})

The default sizing is \keyword{small}.

It rarely makes sense to create 
\keyword{big}, 
\keyword{large} or 
\keyword{huge} indices.
It actually makes sense,
when the index
has many data points per key.
More details on this can be found in \ref{chpt_sizing}.

\subsubsection{DROP}
The \term{drop index} statement removes an index physically from disk.
Example:

\keyword{drop index} \identifier{myidx}

\subsection{Procedure}
\subsubsection{CREATE}
The \term{create procedure} statement
creates a procedure interface in the database.
It has the form:

\keyword{create procedure} \identifier{mymodule}.\identifier{myfun}(
                           \identifier{param1} \keyword{uint},
                           \identifier{param2} \keyword{text})
                           \keyword{language} \identifier{lua}

A procedure may have no parameters.
The definition then simplifies to

\keyword{create procedure} \identifier{mymodule}.\identifier{myfun}()
                           \keyword{language} \identifier{lua}

Any number of parameters is allowed and parameters may have
any static type.

Known languages are \identifier{lua} and \identifier{python}.

\subsubsection{DROP}
The \term{drop procedure} statement
drops a procedure interface from the database.
It has the form:

\keyword{drop procedure} \identifier{myfun}

\subsubsection{SHOW, DESCRIBE}
The command \keyword{show procedures}
returns a cursor with the names of all
procedures
defined in the current database,
one procedure name per row.

\keyword{describe} \identifier{myproc}
returns a cursor with the attributes
and attribute types for the procedure
\identifier{myproc}, one pair of
attribute name, type per row.

\subsection{Function}
\comment{Not yet available}

\subsection{Lock}
\subsubsection{CREATE}
The \term{create lock} statement
creates a read-write lock in the current database.
It has the form:

\keyword{create lock} \identifier{mylock}.

\subsubsection{DROP}
The \term{drop lock} statement
drops a lock from the current database.
It has the form:

\keyword{drop lock} \identifier{mylock}

\subsubsection{SHOW}
The command \keyword{show locks}
returns a cursor with the names of all
lock
defined in the current database,
one lock name per row.

\subsection{Event}
\comment{Not yet available}

\subsection{Queue}
\comment{Not yet available}

\subsection{Period}
\term{Period} is not a \term{first-class citizen}
like types, edges, procedures, \etc\
In particular, periods cannot be created or altered.
They evolve as an effect of inserting edges into
the database. However, periods can be identified
and they can be dropped.

Dropping data according to timestamps is an important
feature in timeseries databases.
Without this feature, databases would grow
to an extent that would make efficient queries difficult
or even impossible. Deleting data by means of the
\term{delete} statement, however, is not efficient for large
amounts of data and, in \nowdb\
(but also in other databases), deleting
the data would not solve the problem anyway,
because \term{delete} does not physically remove
the data, but just makes them invisible.

Dropping a period, by contrast, removes all data files 
that contain only data belonging to that period and
removing files is a very efficient
operation on most platforms.

The syntax is:

\begin{minipage}{\textwidth}
\keyword{drop period on} \identifier{mytable} \\
\keyword{where stamp between} $[$'2018-01-01', '2018-04-01'$[$
\end{minipage}

This would drop all data files of table `mytable'
that contain only data between Jan, 1, 2018 (inclusive) and
April, 1, 2018 (exclusive). 

Two remarks are in place. First,
\term{drop period} must have a \term{where} clause
and this clause must contain exactly one condition,
namely \keyword{between} related to the timestamp.

The rationale for this restriction is
to avoid accidentally dropping too many data
using too complex or incomplete \term{drop} statements.

To illustrate that, the following example 
is legal and it drops all data
before a given date (which is very common
for timeseries databases, that often represent gliding
time windows):

\begin{minipage}{\textwidth}
\keyword{drop period on} \identifier{mytable} \\
\keyword{where stamp between}
$[$\identifier{dawn}(), '2018-04-01'$[$
\end{minipage}

Second, \term{drop period} does not guarantee
to drop all data that lie in the period in question --
in fact, it does not even guarantee to drop any data at all.
It guarantees, however, to remove all files
that contain \emph{only} data that lie within the period.
In other words, the behaviour is conservative
and prefers dropping fewer data than possible over
dropping too many data.
On the long run, however, with a consistent dropping policy
old data will be removed and the database
won't grow (except when the periods themselves grow).

\section{Data Manipulation}
\subsection{Insert}
The \term{insert} statement inserts one or more rows
into a given table.
The basic form is

\keyword{insert into} \identifier{mytable} 
                      (\identifier{myfield1},
                       \identifier{myfield2})
     \keyword{values} (value1, value2)

where `myfield1' \etc\ are fields of entity \identifier{mytable}
and `value1' and so on are expressions.

A more concrete example is

\keyword{insert into} \identifier{product} 
                      (\identifier{prod\_key},
                       \identifier{prod\_desc},
                       \identifier{prod\_price})
     \keyword{values} (100001, 'Spinach', 1.99)

For a stamped edge, this would be:

\begin{minipage}{\textwidth}
\keyword{insert into} \identifier{buys} 
                      (\keyword{origin},
                       \keyword{destin},
                       \keyword{stamp},
                       \identifier{quantity},
                       \identifier{price})
                      (\\
\hspace*{2.99cm}       9000001, 100001,
                       '1929-01-22T08:53:22',
                       3, $3\ast 1.99$)
\end{minipage}

When the list of values is complete,
\ie\ covers all fields in the table,
and respects the canonical order,
a shorthand form can be used, \eg:

\keyword{insert into} \identifier{product} 
     \keyword{values} (100001, 'Spinach', 1.99)

It is also possible \comment{(not yet!)} to insert data from
a query, \eg:

\begin{minipage}{\textwidth}
\keyword{insert into} \identifier{buys} ( \\
                      \keyword{origin},
                      \keyword{destin},
                      \keyword{stamp},
                      \identifier{quantity},
                      \identifier{price}) (\\
\hspace*{0.2cm}\keyword{select} \keyword{origin}, 
                         \keyword{destin},
                         \keyword{stamp},
                         \identifier{quantity},
                         \identifier{price} \\
\hspace*{0.35cm}\keyword{from} \identifier{another\_table} \\
\hspace*{0.2cm}\keyword{where} $\dots$)
\end{minipage}

Insert (just as data loading) respects data integrity
on vertices. That is, the primary key of the vertex
must be unique for that type. Otherwise, \term{insert}
fails with the error \term{duplicate key}.

Edges, on the other hand, have no primary key.
Several edges that all look the same can be inserted
without limits. For people coming from the relational
world, this may sound strange. However, edges are
timeseries data and in the timeseries world
it is completely acceptable and
even common to insert the same event for the same
point in time. It is not so clear 
what ``same time'' shall mean in the first place.
Typically a point in time in a timeseries application
is not an exact spot (\eg\ that millisecond or that
minute). More often than not events happen in
time frames; how to identify and correlate single events
is not so much database methodology, but data science.

\nowdb\ does also not enforce the relation between
vertex type and edge. That is, one can insert edges
to which no corresponding vertex exists.
The reason is that the life cycles of timeseries data
and master data are often not in sync.
For one, timeseries data may arrive, before the
respective vertices have been inserted into the database
and, even more typical, timeseries data might
outlive their vertices in the database.
A client, for instance, who does not renew his or her
customer card, may be removed from the database;
the timeseries data in the database, however,
are still there.

\subsection{Update}
The \term{update} statement changes the values
of fields in rows in tables or types.
Its general form is

\keyword{update} \identifier{mytable} \\
\hspace*{0.7cm} \keyword{set} field = value,\\
\hspace*{0.7cm} \keyword{set} field = value \\
\hspace*{0.1cm} \keyword{where} $\dots$

For instance:

\keyword{update} \identifier{product} \\
\hspace*{0.7cm} \keyword{set} \identifier{prod\_price} = 1.89 \\
\hspace*{0.1cm}  \keyword{where} \identifier{prod\_key} = 100001

For details on the \term{where} clause,
please refer to the \acronym{dql} section.

\comment{update has still many unsolved issues.
For instance, when changing primary keys
and other indexed fields, we need to delete
that particular row from the index and add it
again with the new value.
That, however, is quite expensive.
It therefore may take still some time
to make update available :-(}

\subsection{Upsert}

\subsection{Delete}
The \term{delete} statement eliminates single
data points from tables.
Its general form is

\keyword{delete from} \identifier{mytable} \keyword{where} $\dots$

A more concrete example:

\keyword{delete from} \identifier{product}
\keyword{where} \identifier{prod\_key = 100001}

For details on the \term{where} clause,
please refer to the \acronym{dql} section.

It is worth mentioning that delete
does not physically remove data from disk.
It marks the corresponding rows as deleted.

\section{Data Loading}
\subsection{Create}
The \term{create loader} statement creates a user-defined loader.

\comment{Not yet avalailable}

\subsection{Drop}
The \term{drop loader} statement removes a user-defined loader
form the database.

\comment{Not yet avalailable}

\subsection{Load}
The \term{load} statement loads data from an external
data source into the database.
Its general form is

\keyword{load} '/path/to/datafile' \keyword{into} \identifier{mytype\_or\_edge}

The \term{load} statement has optional \term{use} and \term{ignore} clauses

\keyword{use loader} \identifier{myformat}

With this clause, a user-defined loader is applied.
Without this clause the default loader (\acronym{csv})
is applied. That is equivalent to

\keyword{use loader} \identifier{csv}

In the case of a \acronym{csv} loader,
the \keyword{header} option can be used:

\keyword{use header}

or:

\keyword{use csv, header}

which means that the data source has a header and that the loader
will use this header to determine how to load the columns in the data source.
If the data file has a header, but the user decides not to use it,
the \term{ignore} clause can be used:

\keyword{ignore header}

This will ignore the first line of the \acronym{csv}.

Here is a complete example for the default loader:

\keyword{load} '/opt/import/client.csv'
\keyword{into} \identifier{client} \keyword{use header}

The \term{load} statement returns a report on success
that indicates how many rows have been loaded,
how many rows have failed and how long it took.
Notice that \term{load} does not stop on errors.
Instead, errors are written to a file
with the row number where the error occurred.
This way, the faulty lines can be corrected and
reimported later.

Here is how to indicate an error file:

\keyword{load} '/opt/import/transactions.csv'
\keyword{into} \identifier{buys} \\
\hspace*{0.3cm}\keyword{set errors} $=$ '/opt/import/transactions.err'

If no error file is indicated,
the errors are written to standard error.
This rarely makes sense,
since different sessions can load data concurrently.
Standard error would then contain a mix
of errors of all sessions that tried to load data.

\subsection{Dump}
\comment{Not yet available}

\section{Data Querying}
There is only one type of \acronym{dql} statement.
However, this statement is much more complex
than the statements we have seen so far.

A \acronym{dql} statement consists of at least
a \term{select} clause (also called \term{projection} clause)
and, in most cases, a \term{from} clause,
which in itself may contain
a \term{join} clause.
The \acronym{dql} may additionally contain
a \term{where} clause,
a \term{group} clause and
an \term{order} clause.
\comment{having, limit, sample}

\subsection{Select Clause}
The basic form of a \term{select} clause is

\keyword{select} \identifier{expression}, \identifier{expression}, $\dots$

If the statement has a \term{from} clause,
any expression is allowed.
If it consists only of the \term{select} clause,
fields are not allowed.

If the expression involves fields,
it must be fields of the entity (edge or type)
referenced in the \term{from} clause.
If all fields of that entity are selected,
the statement can be simplified to

\keyword{select} \keyword{$\ast$}

It is also possible to refer explicitly to the data source
(which is identified in the \term{from} clause), \eg:

\keyword{select} \identifier{product}.\identifier{prod\_price}

In the \term{from} clause, we can define aliases for data source
and then refer to the data source by this alias, \eg:

\keyword{select} \identifier{p}.\identifier{prod\_price}

where $p$ is an alias defined in the \term{from} clause.
This is especially useful with joins.

Expressions are, of course, not necessarily fields.
They also may be functions, constants or even complex expressions
composed of fields, constants, functions and operators.
The following clauses are all valid:

\begin{minipage}{\textwidth}
\keyword{select} \identifier{true} \\
\keyword{select} 'X' \\
\keyword{select} 3.14159 \\
\keyword{select} \identifier{sum}(\identifier{price}) $/$ \identifier{count}($\ast$)\\
\keyword{select} \identifier{count}($\ast$), \identifier{sum}(\identifier{price})
\end{minipage}

The first three clauses may appear pointless.
Why select constant values from the database?
There are, however, very common use cases
for selecting constant values, in particular
in combination with the \keyword{exists} operator,
but also for \acronym{dql} used within an
\term{insert} statement.

The \term{select} clause may also contain aggregates,
which are applied to partitions of the result set.
On which partition
aggregates are applied, depends 
on the \term{group} clause. If no \term{group} clause
is present, the partition consists of all rows produced
by the \acronym{dql} statement.

The following are valid \term{select} clauses:

\keyword{select} \identifier{count}($\ast$) \\
\keyword{select} \identifier{sum}(\identifier{price}) \\
\keyword{select} \identifier{sum}(\identifier{log}(\identifier{price})) \\
\keyword{select} \identifier{avg}((\identifier{price} $+$
                                   \identifier{price}) $/$ 2) \\
\keyword{select} \identifier{stddev}(\identifier{price})

The freedom of the \term{select} clause
is restricted by other clauses, in particular
the \term{from} clause and the \term{group} clause.
In the \term{select} clause, only
those fields may appear that are actually
part of the entity chosen in the \term{from} clause.
The interdependencies with the \term{group} clause
are more subtle and will be discussed later.

A meaningful example of a \acronym{dql} statement
that consists only of a \term{select} clause is:

\keyword{select} \identifier{now}()

which obtains the current time.
In a \term{select} clause without a \term{from} clause
fields, obviously, have no meaning.
Thus, only constants, functions
and operators are allowed.

There is a subtle difference between \term{select}-only
statements and ``complete'' \acronym{dql} statements.
Complete \acronym{dql} statements always result in a
cursor, while \term{select}-only statements result
in a single row. This has no impact on queries sent
from a client, since rows are always wrapped into
a cursor before they are sent to the client.
In server-side programming, however,
this actually makes a difference
(see chapter \ref{chpt_luaemb}).

It is also possible to define an alias for fields
in the \term{select} clause by means of
the \keyword{as} keyword, \eg:

\keyword{select} \identifier{count}($\ast$) \keyword{as} \identifier{cnt}

The effect is that some client \acronym{api}s
(see for example chapter \ref{chpt_papiclient})
can refer to columns in a cursor by the alias,
\eg\

\begin{python}
\begin{lstlisting}
print(mycursor['cnt'])
\end{lstlisting}
\end{python}

\subsection{From Clause}
The \term{from} clause determines the data source
of the \acronym{dql} statement. The simplest form
of a \term{from} clause is

\keyword{from} \identifier{mytable}

The identifier must refer to an edge or a type.
Valid \term{from} clauses are for instance:

\keyword{from} \identifier{buys} \\
\keyword{from} \identifier{product} \\
\keyword{from} \identifier{client}

Here is a first example of a complete \acronym{dql} statement:

\keyword{select} \keyword{origin},
                 \keyword{destin}, 
                 \keyword{stamp}, 
                 \identifier{quantity},
                 \identifier{price}
\keyword{from} \identifier{buys}

It is possible to define an alias for the data source:

\keyword{from} \identifier{buys} \keyword{as} \identifier{b}

The table \identifier{buys} can now be referred to as ``b''
in all other clauses.
This technique is especially interesting in combination
with joins.

In most \sql\ dialects,
it is common to select data from different
data sources at once, \eg:

\keyword{from} \identifier{product}, \identifier{client}

This form is not supported by \nowdb.
Whenever more than one data source is addressed,
a join has to be used.
\comment{However, there are paths which are
more handy than using joins in many cases.}

\subsection{Join Clause}
A join combines vertices and edges and,
through edges, vertices with each other.
The basic form is:

\keyword{join} \identifier{mytype} \keyword{on} \keyword{edgefield}

To make this more concrete:

\keyword{from} \identifier{buys} 
\keyword{join} \identifier{client} \keyword{on} \keyword{origin}

This would produce an \term{inner join} 
between \identifier{buys} and \identifier{client}.
Joins in \nowdb\ are in fact always inner joins.
In consequence, there is no difference between
\term{left} and \term{right}; one could say,
joins in \nowdb\ are \term{abelian}.

A way to emulate outer joins is using \term{paths}
(see section \ref{subsec_paths} for details).

Since the primary key of a type is known
and there is always exactly one field
which is primary key, the foreign join key
(that of \identifier{client}) needs no
explicit mentioning. It is implicitly clear 
that \keyword{join} \identifier{client} \keyword{on origin}
joins on \keyword{origin} $=$ \identifier{client\_key}.

Every edge connects two vertices.
A join, therefore, consists of at most two sub-joins.
The syntax is straightforward:

\begin{minipage}{\textwidth}
\keyword{from} \identifier{buys}
\keyword{join} \identifier{client} \keyword{on} \keyword{origin} \\
\hspace*{1.9cm}\keyword{join} \identifier{product} \keyword{on} \keyword{destin}
\end{minipage}

With this join, all attributes
from \identifier{buys}, \identifier{client} and \identifier{product}
are available in all clauses.

It may happen that the joined entities
have fields with the same name.
We could have named the primary key in both,
\identifier{client} and \identifier{product},
\identifier{key}, instead of
\identifier{client\_key} and \identifier{prod\_key}.
To distinguish the fields,
one has to use the entity name together with the field name
in the \term{select} clause and (as we will see) in all
other clauses that refer to fields.
Example:

\keyword{select} \identifier{client}.\identifier{key}

Here, using aliases comes in handy, \eg:

\begin{minipage}{\textwidth}
\keyword{from} \identifier{buys} \keyword{as} \identifier{b}
\keyword{join} \identifier{client} \keyword{as} \identifier{c} 
               \keyword{on} \keyword{origin} \\
\hspace*{2.75cm}\keyword{join} \identifier{product} \keyword{as} \identifier{p}
                              \keyword{on} \keyword{destin}
\end{minipage}

In the \term{select} clause, we can now refer to fields
with the alias instead of the full entity name, \eg:

\keyword{select} \identifier{c}.\identifier{key},
                 \identifier{p}.\identifier{key}

\comment{
Joins are not yet available.
}

\ignore{
\subsection{Correlation} 
Correlation combines two or more edges.
\comment{I believe that correlation is much better placed
in support libraries of high-level languages than in \sql.
Some input from an expert would help ;-)}
}

\subsection{Paths}\label{subsec_paths}
\comment{tbd}

\subsection{Where Clause}\label{subsec_where}
The \term{where} clause adds criteria for the selection
of specific rows. \term{where} clauses can be very complex.
Indeed, a \term{where} clause is one complex \term{Boolean} expression.

A very simple \term{where} clause would be:

\keyword{where} \identifier{prod\_key} = 100001

Which evaluates to true for all rows that have $100001$
as \identifier{prod\_key}.

Since \term{where} clauses are Boolean expressions,
it is possible to use the Boolean operators
\keyword{and}, \keyword{or} and \keyword{not}.
The first two operators are binary, \ie\
they expect two operands (which again are
Boolean expressions), while \keyword{not}
is unary and expects one operand (which again
is a Boolean expression).

Example:

\keyword{where} \identifier{destin} = 100001
\keyword{and} \identifier{stamp} $\le$ '2018-04-01'

Before we look at more interesting cases,
let's assume the following set of data:

\begin{minipage}{\textwidth}
\begin{verbatim}
|---------+--------------|
| destin  | timestamp    |
|---------+--------------|
| 100001  | '2018-03-15' |
| 100002  | '2018-03-15' |
| 100001  | '2018-04-15' |
| 100002  | '2018-04-15' |
| 100001  | '2018-05-15' |
| 100002  | '2018-05-15' |
| 100001  | '2018-06-15' |
| 100002  | '2018-06-15' |
|----------+-------------|
\end{verbatim}
\end{minipage}

Here is a tricky \keyword{where} clause:

\keyword{where} \identifier{destin} = 100001
\keyword{and} \identifier{stamp} $<$ '2018-04-01'
\keyword{or} \identifier{stamp} $\ge$ '2018-05-01'

\keyword{or} has precedence over \keyword{and}.
Both, \keyword{or} and \keyword{and}, bind to the left,
\ie\ the first operand is the one in front of the operator.

That means, here, that \keyword{or} is at the top-level:
The clause selects all rows that have \keyword{stamp}
$\ge$ May, 1, and those that have \keyword{destination} $100001$
and \keyword{stamp} less than April, 1.
In other words, what we see is

\begin{minipage}{\textwidth}
\begin{verbatim}
|---------+--------------|
| destin  | timestamp    |
|---------+--------------|
| 100001  | '2018-03-15' |
| 100001  | '2018-05-15' |
| 100002  | '2018-05-15' |
| 100001  | '2018-06-15' |
| 100002  | '2018-06-15' |
|----------+-------------|
\end{verbatim}
\end{minipage}

This might be surprising at the first glance.
But, indeed, most \sql\ dialects follow this convention.

To force another binding, parentheses can be used:

\begin{minipage}{\textwidth}
\keyword{where} \keyword{destin} = 100001 \\
\hspace*{0.45cm}\keyword{and} (\keyword{stamp} $<$ '2018-04-01'
\keyword{or} \keyword{stamp} $\ge$ '2018-05-01')
\end{minipage}

This clause would now select all rows
that have \keyword{destination} $100001$ and
a \keyword{timestamp} that is not in April,
like this:

\begin{minipage}{\textwidth}
\begin{verbatim}
|---------+--------------|
| destin  | timestamp    |
|---------+--------------|
| 100001  | '2018-03-15' |
| 100001  | '2018-05-15' |
| 100001  | '2018-06-15' |
|----------+-------------|
\end{verbatim}
\end{minipage}

Equivalent to this second clause is

\keyword{where} \identifier{destin} = 100001 \\
\hspace*{0.45cm}\keyword{and} \keyword{not} 
(\keyword{stamp} $\ge$ '2018-04-01'
\keyword{and} \keyword{stamp} $<$ '2018-05-01')

\keyword{or} and \keyword{and} have precedence
over \keyword{not}; \keyword{not} binds to the right,
\ie\ \keyword{not} is a \term{prefix}.

Would we again leave out the parentheses,
like this:

\keyword{where} \identifier{destin} = 100001 \\
\hspace*{0.45cm}\keyword{and} \keyword{not} 
\keyword{stamp} $\ge$ '2018-04-01'
\keyword{and} \keyword{stamp} $<$ '2018-05-01'

we would select all rows that have \keyword{destination}
$100001$ and not a \keyword{stamp} greater April
and that are before May, hence:

\begin{minipage}{\textwidth}
\begin{verbatim}
|---------+--------------|
| destin  | timestamp    |
|---------+--------------|
| 100001  | '2018-03-15' |
|----------+-------------|
\end{verbatim}
\end{minipage}

\ignore{
\subsection{While Clause}
The \term{while} clause implements a common requirement
in graph databases, namely to follow a link recursively.
A typical example is social networks where we want to
know if $A$ is connected to $B$ through a path spanning
more than one edge (of the same kind).
Here is an illustration:

\begin{minipage}{\textwidth}
\keyword{select} \keyword{true}
\keyword{from} \identifier{friend} \\
\keyword{where} \keyword{edge} $=$ 'isfriend' \\
\keyword{while} \keyword{destin} $\neq$ 12345 
\end{minipage}

This statement would follow the ``isfriend'' edges
until an edge with destination $12345$ is reached.

\comment{
This is not yet available -- and there are some
aspects to clarify: First, the number of iterations
must be bounded; otherwise we may follow a
cycle for eternity. Second, this, obviously, works
only for edges where origin and destination
have the same type. Interesting
may be cases where the types of origin and
destination alternate, \eg\
\identifier{client} `buys' \identifier{product} and
\identifier{product} \keyword{passive}(`buys') \identifier{client}.
The `operator' \term{passive} here would turn the edge around:
we first go from \identifier{client} to \identifier{product}
and then from \identifier{product} to (another) \identifier{client}.
With this, however, a new difficulty arises:
how to avoid the combinatorial explosion after some,
in fact very few, iterations?
More efficient for such cases is probably just to loop
through all edges.
}
}

\subsection{Group Clause}\label{sec_group}
Grouping partitions the result set
according to a set of \term{keys}.
The result set will then be presented
according to these partitions.
The keys used to partition the set
are defined in the \term{group} clause.
A simple \term{group} clause could be:

\keyword{group by} \keyword{destin}

This clause would partition the result
according to \keyword{destin}.

In the simplest form a statement with
group clause could look like this:

\keyword{select} \keyword{destin}
\keyword{from} \identifier{buys}
\keyword{group by} \keyword{destin}

Let's consider an edge with the
user-defined field \identifier{category}
and the following set of data:

\begin{minipage}{\textwidth}
\begin{verbatim}
|----------+---------+--------------|
| category | destin  | timestamp    |
|----------+---------+--------------|
| 'buys'   | 100001  | '2018-03-15' |
| 'buys'   | 100002  | '2018-03-15' |
| 'buys'   | 100001  | '2018-04-15' |
| 'buys'   | 100002  | '2018-04-15' |
| 'buys'   | 100001  | '2018-05-15' |
| 'buys'   | 100002  | '2018-05-15' |
| 'buys'   | 100001  | '2018-06-15' |
| 'steals' | 100002  | '2018-06-15' |
|----------+----------+-------------|
\end{verbatim}
\end{minipage}

We apply the following grouping to this data set

\keyword{select} \identifier{category},
                 \keyword{destin}
\keyword{from} \identifier{buys}
\keyword{group by} \identifier{category}, \keyword{destin}

This would produce the following output

\begin{minipage}{\textwidth}
\begin{verbatim}
|----------+---------|
| category | destin  |
|----------+---------|
| 'buys'   | 100001  |
| 'buys'   | 100002  |
| 'steals' | 100002  |
|----------+---------|
\end{verbatim}
\end{minipage}

It, hence, produces a set of distinct key values.
One could say that grouping \emph{abstracts} or \emph{reduces}
the data set according to the keys.
This, in itself, is often a useful feature,
\viz\ when we want to apply a certain processing
only once per key. As an aside it may be mentioned
that this kind of grouping is extremely fast in \nowdb;
for performance considerations in general, please
refer to chapter \ref{chpt_opt}.

The real power of grouping becomes evident
when we let the database do the processing.
This is the main purpose of aggregate functions
(please refer to section \ref{sec_agg} for details).

The aggregate functions go to the \term{select} clause
behind the grouping keys:

\keyword{select} \identifier{category},
                 \keyword{destin},
                 \identifier{count}($\ast$)
\keyword{from} \identifier{buys}
\keyword{group by} \identifier{category}, \keyword{destin}

This would now produce:

\begin{minipage}{\textwidth}
\begin{verbatim}
|----------+---------+-------|
| category | destin  | count |
|----------+---------|-------|
| 'buys'   | 100001  |  4    |
| 'buys'   | 100002  |  3    |
| 'steals' | 100002  |  1    |
|----------+---------+-------|
\end{verbatim}
\end{minipage}

The \term{group} clause has strong interdependencies
with the \term{select} clause.
In particular, the \term{select} clause must
contain those fields mentioned in the
\term{group} clause and the order of the fields
must be identical. The only things allowed in
the \term{select} clause besides the grouping keys
are aggregates. \comment{Some of these requirements
will be relaxed in the future.}

The following statements are therefore wrong:

\keyword{select} \identifier{category}, 
                 \identifier{count}($\ast$)
\keyword{from} \identifier{buys}
\keyword{group by} \keyword{category, destin}

\keyword{select} \keyword{destin},
                 \identifier{category}, 
                 \identifier{count}($\ast$)
\keyword{from} \identifier{buys}
\keyword{group by} \identifier{category}, \keyword{destin}

\keyword{select} \identifier{category}, 
                 \keyword{destin},
                 'hello world'
\keyword{from} \identifier{buys}
\keyword{group by} \identifier{category}, \keyword{destin}

Notice that aggregates without grouping
are applied to the whole result set.
For instance, to count all rows in \identifier{buys},
one could say:

\keyword{select} \identifier{count}($\ast$)
\keyword{from} \identifier{buys}

The partition to which \identifier{count} is applied
is here the whole result set.

Syntactically, even this is \acronym{ok}:

\keyword{select} \identifier{category},
                 \keyword{destin},
                 \identifier{count}($\ast$)
\keyword{from} \identifier{buys}

The values shown for \identifier{category} and \keyword{destin},
however, have no relation whatsoever to the \identifier{count}
result. They correspond
to any of the rows in the result set,
the first, the last or any other.
That is unspecified.

\comment{
To be discussed: grouping and having.
}

\comment{
Currently, grouping is only possible for combinations of keys
for which an index exists. That is to say:
the index must be defined over the same set of keys
and the order of the keys must be the same.
Otherwise, when there is no matching index,
the query fails with error.
}

\subsection{Order Clause}
The \term{order} clause sorts the result set
according to a set of sorting criteria (the \term{order keys}).
Its simplest form is just:

\keyword{order by} \identifier{category}, \keyword{destin}

which would present the result set
sorted by \identifier{category} and \keyword{destin}.

\begin{minipage}{\textwidth}
The statement

\keyword{select} \identifier{category}, \keyword{destin}
\keyword{from} \identifier{buys}
\keyword{order by} \identifier{category}, \keyword{destin}
\end{minipage}

applied to the data set already used in section
\ref{sec_group} would produce the following output:

\begin{minipage}{\textwidth}
\begin{verbatim}
|----------+---------+--------------|
| category | destin  | timestamp    |
|----------+---------+--------------|
| 'buys'   | 100001  | '2018-03-15' |
| 'buys'   | 100001  | '2018-04-15' |
| 'buys'   | 100001  | '2018-05-15' |
| 'buys'   | 100001  | '2018-06-15' |
| 'buys'   | 100002  | '2018-03-15' |
| 'buys'   | 100002  | '2018-04-15' |
| 'buys'   | 100002  | '2018-05-15' |
| 'steals' | 100002  | '2018-06-15' |
|----------+----------+-------------|
\end{verbatim}
\end{minipage}

\comment{
Similar to grouping, ordering works only with an index
that has the same keys in the same order.
}

\subsection{Limit and Sample Clause}
\comment{not yet available}

\section{Miscellaneous}
\subsection{Use}
The \term{use} statement sets the database for
the current session. All following statements
 until the next \term{use} statement
will be applied to this database.
Example:

\keyword{use} \identifier{retail}

The statement returns a \term{Status} result.

\subsection{Exec}
The \term{exec} statement executes a stored procedure
whose interface was previously defined.
The simplest form is

\keyword{exec} \identifier{myprocedure}()

In this case \identifier{myprocedure} has no parameters.
A concrete example with parameters may be

\keyword{exec} \identifier{revenue}(9000001, '2018-05-01')

The result of the statement depends on the procedure.

\subsection{Lock}
The \term{lock} statement acquires a lock
for reading or writing. The simplest form is

\keyword{lock} \identifier{mylock}

This would acquire \identifier{mylock} for writing.
This can also be done explicitly:

\keyword{lock} \identifier{mylock} \keyword{for writing}

or, for reading:

\keyword{lock} \identifier{mylock} \keyword{for reading}

When locking for reading, the lock is acquired if no session
is currently holding the lock for writing. That is,
locking for reading is not exclusive. Many sessions
may hold the same lock for reading at the same time.
Locking for writing, however, is exclusive.
A lock may only be acquired for writing,
when it is not held by another session
(for reading or writing) and,
when a lock is held for writing, no other session
can acquire this lock for neither reading nor writing.

If a lock is already held, so that the current
lock request cannot acquire it,
the session blocks until the lock is released.
There is no guarantee of any particular order
in which locks are acquired. If two sessions
are waiting for the same lock, any of the two
will acquire the lock when it is released.

The \term{lock} statement allows for specifying
a timeout by means of an option clause, \eg

\keyword{lock} \identifier{mylock} \keyword{for writing}
\keyword{set timeout} = 100

The timeout is specified in milliseconds.
If the timeout expires, the statement fails with
the error ``timeout expired''.
If the timeout value is 0, the session will not block
in case the lock is already held by another session,
but return immediately with \term{timeout expired}.
If no option is given, the session blocks until it
has successfully acquired the lock.

It is not allowed to acquire a lock already held
by the caller. Attempts to do so are detected and
the error ``selflock'' is generated.
Other kinds of deadlocks are currently not detected.

\subsection{Unlock}
A lock is released by the \term{unlock} statement, \eg:

\keyword{unlock} \identifier{mylock}

If the calling session is not currently holding
the lock, the statement fails with ``does not hold the lock''.
Otherwise, the lock is released and immediately
afterwards acquirable.



\chapter{The NoWDB daemon}\label{chpt_nowdbd}
\comment{tbc}

\chapter{The Client Tool}\label{chpt_clienttool}
\comment{tbc}

\chapter{The Low-Level C Client}\label{chpt_llc}
\section{Outline}

\section{Time}

\section{Connection}

\section{Result}

\subsubsection{Status}
\subsubsection{Cursor}
\subsubsection{Row}
\subsubsection{Report}

\section{Error Handling}

\section{Error Codes}
\bgroup
\renewcommand{\arraystretch}{1.3}
\begin{center}
\begin{longtable}{||l||c||l||}\hline
\textbf{Error Name} & \textbf{Numerical Code} & \textbf{Meaning} \\\hline\endhead\hline
out of memory                             &     -1 & \\\hline\hline
no connection                             &     -2 & \\\hline\hline
socket error                              &     -3 & \\\hline\hline
error on address                          &     -4 & \\\hline\hline
cannot create result                      &     -5 & \\\hline\hline
invalid parameter                         &     -6 & \\\hline\hline
error on read operation                   &   -101 & \\\hline\hline
error on write operation                  &   -102 & \\\hline\hline
error on open  operation                  &   -103 & \\\hline\hline
error on close operation                  &   -104 & \\\hline\hline
use statement failed                      &   -105 & \\\hline\hline
protocol error                            &   -106 & \\\hline\hline
statement or requested resource too big   &   -107 & \\\hline\hline
operating system error (check errno)      &   -108 & \\\hline\hline
time or date format error                 &   -109 & \\\hline\hline
cursor with zerocopy requested            &   -110 & \\\hline\hline
cannot close cursor                       &   -111 & \\\hline\hline
\end{longtable}
\end{center}
\egroup


\chapter{Python Client}\label{chpt_pythonclient}
\section{Outline}
The python client consists of a bunch
of modules all starting with prefix \term{now}.
The main module which defines
the \term{Connection} and the \term{Result} class
is actually called just \tech{now.py}.
Support functions (which are also available)
for server-side python are defined in
\tech{nowutil.py}.

The modules use the package \term{dateutil}
which must be installed on the system in order
to use the \term{now} modules.
(Please refer to chapter
\ref{chpt_install} for details.)

A client program needs to import at least
the \term{now} module. The import may have
any form (\ie\ \term{import now} or
\term{from now import}).

\section{Connections}
The \term{Connection} class represents
a \acronym{tcp/ip} connection to the
database. It provides the following 
methods:
\subsubsection{\_\_init\_\_}
The constructor takes four arguments,
which are all strings:
\begin{itemize}
\item Address:
used to determine the address of the server.
The parameter is passed to the system service
\tech{getaddrinfo} and allows:
an \acronym{ip}v4 address,
an \acronym{ip}v6 address or
a hostname.
Examples: ``127.0.0.1'', ``localhost'',
``myserver.mydomain.org''.

\item Port:
used to determine the port of the server.
The parameter is passed to the system service
\tech{getaddrinfo} and allows:
a port number or a known service name.

\item User:
\comment{currently not used}
\item Password:
\comment{currently not used}
\end{itemize}

\subsubsection{execute(statement)}
The method sends the string \term{statement} to the database
and waits for the database response.
The method returns an instance of the \term{Result} class
or, on internal errors, raises one of the exceptions
\term{ClientError} or \term{DBError}.

\subsubsection{close()}
The method closes the connection and
releases the C objects allocated by
the constructor. If the connection
cannot be closed for any reason,
the exception \term{ClientError} is raised.

\subsubsection{Resource Manager}
\term{Connection} is a \term{resource manager}.
The method \term{close} is therefore
rarely necessary. Instead, \term{Connection} can
be used with the \term{with} statement, \ie:

\begin{python}
\begin{lstlisting}
with Connection("localhost", "55505", None, None) as conn:
    # here goes your code
    # refer to the connect as 'conn'
\end{lstlisting}
\end{python}

\section{Results}
The \term{Result} class represents
the dynamic types described in chapter \ref{chpt_sql}.
Instances of \term{Result} are
created and returned by
the \term{Connection.execute()} method.

The \term{Result} class has the following methods:

\subsubsection{rType()}
The method returns the result type,
either STATUS, REPORT, ROW or CURSOR.

\subsubsection{ok()}
The method return \term{True}
if the instance does not represent
an error and \term{False} otherwise.

\subsubsection{release()}
The method releases the C objects
allocated with the \term{Result} object.
Since \term{Result} is a resource manager,
it is rarely necessary to use explicitly.

\subsubsection{Resource Manager}
Result is a resource manager.
\term{Result} can therefore
be used with the \term{with} statement, \eg:

\begin{python}
\begin{lstlisting}
with conn.execute("select count(*) from sales") as cur:
    # here goes your code
    # 'conn' is a previously created connection
    # 'cur' is the result and
    # if no error has occurred, cur is a cursor
\end{lstlisting}
\end{python}

\subsection{Status}
If \term{Result} is a \term{Status},
two more methods are available:

\subsubsection{code()}
The method returns the \nowdb\ error code.
The error code may be
\begin{itemize}
\item 0: Success, no error has occurred;
\item positive:
An error in the database occurred;
chapter \ref{chpt_errors} provides a list
of server-side error codes.
\item negative:
An error in the client library occurred;
chapter \ref{chpt_llc} provides a list
of client-side error codes.
\end{itemize}

\subsubsection{details()}
The method returns
detailed information on the error
(or \term{None} if no error has occurred.

\subsection{Cursors}
If the result is a cursor,
four more methods are available:

\subsubsection{fetch()}
The method fetches the next bulk
of rows from the database.
After successful completion,
the cursor contains the bulk
of rows, which can be obtain by means of
the method \term{row()}.
Note that the first bulk of rows
is available immediately with
Cursor creation.

\subsubsection{row()}
The method obtains the current
bulk of rows from the cursor.
It returns a \term{Row} result.

\subsubsection{close()}
The method closes the cursor.
Since cursors are server-side
resource, it needs to be closed
explicitly; the method, however,
is usually not called directly,
but implicitly, when the result
is created with a \term{with} statement.

\subsubsection{eof()}
The method returns \term{True}
if the error state of the cursor
is \term{end-of-file} and \term{False}
otherwise.

\subsubsection{Iterator}
\term{Cursor} is an iterator.
Usually, \term{fetch()}, \term{row()}
and \term{eof()} do not need to be called
explicitly in the code.
Instead a \term{for}-loop  can be used, \eg:

\begin{python}
\begin{lstlisting}
with conn.execute("select * from sales") as cur:
    if not cur.ok():
        print "ERROR: %s " % cur.details()
        return
    for row in cur:
        # process cursor
        # row holds a single row
\end{lstlisting}
\end{python}

\subsection{Rows}
If the result is a row,
two more methods are available:

\subsubsection{field(n)}
The method returns the value
of the $n$th field (starting to count
from 0 for the first field).
The value is an \sql\ base type
converted to Python.
Conversion takes place in the obvious way
(integer and unsigned integer to int,
 float to float, text to string, \etc).
An exception are \term{time} fields.
One could expect \term{field} to return
a \term{datetime}, but that is not a case.
Time is returned as a (signed) integer
representing a \acronym{unix} timestamp
with nanosecond precision.
It can be converted with \term{now2db}
(please refer to section \ref{sec_supp}
for details).

\subsubsection{nextRow()}
The method advances to the next row.
If there was one more row, the method
returns \term{True} and the next call
to \term{field(n)} will return the
$n$th field of the next row.
Otherwise, if no more rows are available,
the method returns \term{False}.

Note that, working with a cursor
as iterator, it is not necessary
to use this method. The iterator
produces single rows.
The main use case of \term{nextRow()}
is for user-defined procedures that
return bulks of rows. An example
for this usage is:

\begin{python}
\begin{lstlisting}
with conn.execute("exec myproc()") as row:
    if not row.ok():
        print "ERROR: %s " % row.details()
        return
    while True:
        # use row.field(n) to access fields
        if not row.nextRow():
            break
\end{lstlisting}
\end{python}

\subsection{Reports}
If the result is a report,
three more methods are available:

\subsubsection{affected()}
The method returns the number of affected rows.

\subsubsection{errors()}
The method returns the number of errors.
Note that \acronym{ddl} statements will
return a status result if an error occurred.
In practice, only \acronym{dml} statements provide
this information.

\subsubsection{runTime()}
The method returns the running time of 
the \acronym{dml} or \acronym{dll} statement.

A usage example is

\begin{python}
\begin{lstlisting}
with conn.execute("load '/opt/import/client.csv into client") as rep:
    if not rep.ok():
        print "ERROR: %s " % rep.details()
        return
    print "affected: %d, errors: %d, running time: %dus" % 
             (rep.affected(), rep.errors(), rep.runTime())
\end{lstlisting}
\end{python}

\section{Exceptions and Errors}
\subsubsection{ClientError}
Is raised when an error in the client library occurred
(\eg\ a socket error).

\subsubsection{DBError}
Is raised when an error in the database occurred.

\subsubsection{WrongType}
Is raised in case of type mismatch, \ie\
trying to call a method not available for
this specific return type.

\subsubsection{explain(err)}
The function \term{explain} returns
a description of the error code passed in.

\section{Support Functions}\label{sec_supp}
Support functions are provided by module
\term{nowutil}. Currently the following
support functions are available:

\subsubsection{dt2now(dt)}
The function expects a \term{datetime}
object and converts it to a \nowdb\ timestamp.

\subsubsection{now2dt(tp)}
The function expects a \nowdb\ timestamp
and converts it to a \term{datetime} object.

\subsubsection{utc}
This is a global variable defined in \term{nowutil}.
It is the \acronym{utc} timezone object needed
for many datetime constructors and conversion functions.
It is provided for convenience.

\subsubsection{TIMEFORMAT and DATEFORMAT}
These are global constants providing the
standard \nowdb\ time and date format
strings that can be used, for instance,
in the \term{datetime}
\term{strftime} and \term{strptime} methods, \eg:

\begin{python}
\begin{lstlisting}
    stmt = "select count(*) from sales\
             where timestamp = '" + tp.strftime(TIMEFORMAT) + "'"
\end{lstlisting}
\end{python}


\ignore{
\chapter{Higher Level C and C++ Client}\label{chpt_ccpp}
\comment{tbd}

\chapter{Go Client}\label{chpt_goclient}
\comment{tbd}

\chapter{Lua Client}\label{chpt_luaclient}
\comment{tbd}
}

\ignore{
\chapter{Embedded Python}\label{chpt_pythonemb}
\section{Outline}
The embedded python library consists mainly
of the \term{nowdb} module, which defines
the \term{execute} function and the \term{Result} class.
Support functions (which are also available
for client-side python) are defined in
\tech{nowutil.py}.

The module uses the package \term{dateutil}
which must be installed on the system in order
to use the \term{nowdb} module.
(Please refer to chapter
\ref{chpt_install} for details.)

An embedded Python program needs to import at least
the \term{nowdb} module. The import must have
the form \term{import nowdb}, because the database
interacts with this module through the user-defined
module.

The user may define a module-level function
\term{cleanup()} in his or her module.
This function is called by the database
before the module is unloaded at the end of
the session. A meaningful use case for this
function is provided below in section \ref{sec_func}.

The embedded \acronym{api} is very similar
to that provided by the client library,
but there are also some differences.
The most obvious difference is that
there is no \term{Connection} class.
Instead, there is a module-level function
to execute \sql\ statements.

There are also model-level functions
to create results. In the client library
this is not necessary, because clients
do not need to return results.
Likewise, the \term{Result} class
has some more methods to deal with
the specifics of creating results
and returning them to the database.

\comment{
Currently, we only discuss procedures --
in the future, there will also be functions
(which can be called in \sql\ context)
and those will be discussed here as well.
}

\section{Module-level Functions}

\subsubsection{execute(statement)}
The function calls the parser against the statement and, on success,
executes the parsing result in the database.
The function returns an instance of the \term{Result} class
or, on internal errors, raises the exception \term{DBError}.

\subsubsection{success()}
The function creates a \term{Status} result
that represents \acronym{ok}.

\subsubsection{makeError(code, msg)}
The function creates a \term{Status} result
that represents an error event with error code
\term{code} and detailed information \term{msg}.

\subsubsection{makeRow()}
The function creates a \term{Row} result,
which represents an individual row
or a bulk of rows that are sent to the client
in one go.
The row is initially empty.

\subsubsection{convert(typ, value)}
The function converts \term{value} according
to the \nowdb\ \sql\ type \term{typ}.
Available types are
\acronym{text}, \acronym{time},
\acronym{float}, \acronym{int}, \acronym{uint} and
\acronym{bool},
which are defined as global constants.

\section{Results}
The \term{Result} class represents
the dynamic types described in chapter \ref{chpt_sql}.
Instances of \term{Result} are
created and returned by
the \term{execute()} function
and by the \term{Result} creators discussed above.

The \term{Result} class has the following methods:

\subsubsection{rType()}
The method returns the result type,
either \acronym{status},
\acronym{report}, \acronym{row} or \acronym{cursor}.

\subsubsection{ok()}
The method returns \term{True}
if the instance does not represent
an error and \term{False} otherwise.

\subsubsection{toDB()}
The method returns a \term{void} pointer
to the underlying C structure.
The database expects such a pointer
as result from a procedure call and
wouldn't know what to do
with a Python result instance.
Any result returned to the database
must therefore call \term{toDB()} 
and return the result of that method.
The following example would return
the result of a statement as is:

\begin{python}
\begin{lstlisting}
with execute("select count(*) from sales") as cur:
    return cur.toDB()
\end{lstlisting}
\end{python}

\subsubsection{release()}
The method releases the C objects
allocated with the \term{Result} object.
Since \term{Result} is a resource manager,
it is rarely necessary to use it explicitly.

\subsubsection{Resource Manager}
Result is a resource manager.
\term{Result} can therefore
be used with the \term{with} statement, \eg:

\begin{python}
\begin{lstlisting}
with execute("select count(*) from sales") as cur:
    # here goes your code
    # 'cur' is the result and
    # if no error has occurred, cur is a cursor
\end{lstlisting}
\end{python}

\subsection{Status}
If \term{Result} is a \term{Status},
two more methods are available:

\subsubsection{code()}
The method returns the \nowdb\ error code.
The error code may be 0,
which means that the call was successful,
no error has occurred;
or it may be one of the error codes listed in
chapter \ref{chpt_errors}.

\subsubsection{details()}
The method returns
detailed information about the error
(or \term{None} if no error has occurred).

\subsection{Cursors}
If the result is a cursor,
four more methods are available:

\subsubsection{fetch()}
The method fetches the next bulk
of rows from the database.
After successful completion,
the cursor contains this bulk
of rows, which can be obtained by means of
the method \term{row()}.
Note that the first bulk of rows
is available immediately after
Cursor creation.

\subsubsection{row()}
The method obtains the current
bulk of rows from the cursor.
It returns a \term{Row} result.

\subsubsection{close()}
The method closes the cursor.
Cursors need to be closed
explicitly; the method, however,
is usually not called directly,
but implicitly, when the result
is created by means of a \term{with} statement.

\subsubsection{eof()}
The method returns \term{True}
if the error state of the cursor
is \term{end-of-file} and \term{False}
otherwise.

\subsubsection{Iterator}
\term{Cursor} is an iterator.
Usually, \term{fetch()}, \term{row()}
and \term{eof()} do not need to be called
explicitly.
Instead a \term{for}-loop  can be used, \eg:

\begin{python}
\begin{lstlisting}
with execute("select * from sales") as cur:
    if not cur.ok():
        print "ERROR: %s " % cur.details()
        return
    for row in cur:
        # process cursor
        # row holds a single row
\end{lstlisting}
\end{python}

\subsection{Rows}
If the result is a row,
four more methods are available:

\subsubsection{field(n)}
The method returns the value
of the $n$th field (starting to count
from 0 for the first field).
The value is an \sql\ base type
converted to Python.
Conversion takes place in the obvious way
(integer and unsigned integer to int,
 float to float, text to string, \etc).
An exception are \term{time} fields.
One could expect \term{field} to return
a \term{datetime}, but that is not the case.
Time is returned as a (signed) integer
representing a \acronym{unix} timestamp
with nanosecond precision.
It can be converted with \term{now2dt}
(please refer to chapter \ref{chpt_pythonclient}
for details).

\subsubsection{nextRow()}
The method advances to the next row.
If there was one more row, the method
returns \term{True} and the next call
to \term{field(n)} will return the
$n$th field of that row.
Otherwise, if no more rows are available,
the method returns \term{False}.

Note that, working with a cursor
as iterator, it is not necessary
to use this method. The iterator
produces single rows.
The main use case of \term{nextRow()}
is for user-defined procedures that
return bulks of rows. An example
for this usage is:

\begin{python}
\begin{lstlisting}
with execute("exec myproc()") as row:
    if not row.ok():
        print "ERROR: %s " % row.details()
        return
    while True:
        # use row.field(n) to access fields
        if not row.nextRow():
            break
\end{lstlisting}
\end{python}

\subsubsection{add2Row(typ, value)}
The method adds \term{value}
as \nowdb\ \sql\ type \term{typ} to
the row. This method is intended
for rows that are created in user code.
A usage example is:

\begin{python}
\begin{lstlisting}
with execute("select origin from edge where edge = 'buys'") as cur:
    our = makeRow() # create a row result
    for row in cur:
       with execute("select client_name from client\
                      where client_key = " +
                      str(row.field(0))) as cur2:
            for cow in cur2:
                our.add2Row(UINT, cur.field(0))
                our.add2Row(TEXT, cow.field(0))
                our.closeRow() # terminate row
    return our.toDB() 
\end{lstlisting}
\end{python}

This is a na\"ive join implementation:
For each edge, the function executes a \term{select}
on the \term{client} with the key found in \term{origin}.
The function creates a row ``our''.
Each of the rows contains two fields: the \term{origin}
from edge and the \term{client\_name} from \term{client}.
Finally, the row is returned (calling $toDB()$).

Note that the size of row buffers is limited.
The exact size is not part of this specification.
On adding a value to a row that has already reached
the limits an exception is raised.

\subsubsection{closeRow()}
The method inserts an \term{end-of-row} marker
at the current end of the row.
Without such a marker, the row would be invalid.
Moreover, and instance of the \term{row} type
may contain more than one row.
The marker, in this case, separates the individual
rows within that row buffer.

\subsubsection{fitsRow(typ, value)}
The method checks whether \term{value} of type \term{typ}
still fits into the row. It returns \term{True}
if the value still fits and \term{False} otherwise.

\subsection{Reports}
If the result is a report,
three more methods are available:

\subsubsection{affected()}
The method returns the number of affected rows.

\subsubsection{errors()}
The method returns the number of errors.
Note that \acronym{dml} statements will
return a \term{Status} result if an error occurred.
In practice, only \acronym{dll} statements provide
this information.

\subsubsection{runTime()}
The method returns the running time of 
the \acronym{dml} or \acronym{dll} statement.

A usage example is

\begin{python}
\begin{lstlisting}
with execute("load '/opt/import/client.csv into client") as rep:
    if not rep.ok():
        print "ERROR: %s " % rep.details()
        return
    
    print "affected: %d, errors: %d, running time: %dus" % 
             (rep.affected(), rep.errors(), rep.runTime())
\end{lstlisting}
\end{python}

\section{Exceptions and Errors}
\subsubsection{DBError}
Is raised when an error in the database occurred.

\subsubsection{WrongType}
Is raised in case of type mismatch, \ie\
trying to call a method not available for
this specific return type.

\subsubsection{explain(err)}
The function \term{explain} returns
a description of the error code passed in.

\section{Functional Cursor Library}\label{sec_func}
\comment{tbc}

}

\chapter{Embedded Lua}\label{chpt_luaemb}
\comment{tbd}

\chapter{Detailed Installation Guide}\label{chpt_install}
\comment{tbd}

\chapter{Defining Data Loaders}\label{chpt_loader}
\comment{tbd}

\chapter{Publish and Subscribe}\label{chpt_pubsub}
\comment{tbd}

\chapter{Optimising Queries}\label{chpt_opt}
\section{Terminology}

\section{Fullscan}

\section{Searching}
Index vs. no index,
role of periods 

\section{Grouping and Ordering}
Explain: FRANGE, KRANGE and CRANGE,
grouping and ordering without indices

\section{Joining}


\chapter{Storage Sizing}\label{chpt_sizing}
\comment{tbc}

\chapter{Error Codes}\label{chpt_errors}
\bgroup
\renewcommand{\arraystretch}{1.3}
\begin{center}
\begin{longtable}{||l||c||l||}\hline
\textbf{Error Name} & \textbf{Numerical Code} & \textbf{Meaning} \\\hline\endhead\hline

out of memory                   &      1 & \\\hline\hline
invalid parameter or value      &      2 & \\\hline\hline
no resource available           &      3 & \\\hline\hline
resource is busy                &      4 & \\\hline\hline
requested resource too big      &      5 & \\\hline\hline
locking error                   &      6 & \\\hline\hline
unlock error                    &      7 & \\\hline\hline
end of file                     &      8 & \\\hline\hline
feature not supported           &      9 & \\\hline\hline
bad path                        &     10 & \\\hline\hline
bad name                        &     11 & \\\hline\hline
error in map operation          &     12 & \\\hline\hline
error in unmap operation        &     13 & \\\hline\hline
error in read operation         &     14 & \\\hline\hline
error in write operation        &     15 & \\\hline\hline
error in open operation         &     16 & \\\hline\hline
error in close operation        &     17 & \\\hline\hline
error in remove operation       &     18 & \\\hline\hline
error in seek operation         &     19 & \\\hline\hline
internal error (panic)          &     20 & \\\hline\hline
error reading data catalog      &     21 & \\\hline\hline
error in time operation         &     22 & \\\hline\hline
key not found                   &     26 & \\\hline\hline
duplicated key                  &     27 & \\\hline\hline
duplicated name                 &     28 & \\\hline\hline
error in sync operation         &     30 & \\\hline\hline
error in pthread operation      &     31 & \\\hline\hline
error in sleep operation        &     32 & \\\hline\hline
error in dequeue operation      &     33 & \\\hline\hline
error in enqueue operation      &     34 & \\\hline\hline
worker thread failed            &     35 & \\\hline\hline
timeout                         &     36 & \\\hline\hline
bad block                       &     38 & \\\hline\hline
bad filesize                    &     39 & \\\hline\hline
cannot set max files            &     40 & \\\hline\hline
error in move operation         &     41 & \\\hline\hline
index-related error             &     42 & \\\hline\hline
wrong or unknown version        &     43 & \\\hline\hline
error in compression            &     44 & \\\hline\hline
error in decompression          &     45 & \\\hline\hline
error in compression dictonary  &     46 & \\\hline\hline
error in data store             &     47 & \\\hline\hline
error on table level            &     48 & \\\hline\hline
error on database level         &     49 & \\\hline\hline
error in stat operation         &     50 & \\\hline\hline
error in create operation       &     51 & \\\hline\hline
error in drop operation         &     52 & \\\hline\hline
wrong or unknown magic number   &     53 & \\\hline\hline
error in loader                 &     54 & \\\hline\hline
error in trunc operation        &     55 & \\\hline\hline
error in flush operation        &     56 & \\\hline\hline
error in beet library operation &     57 & \\\hline\hline
error in aggregate function     &     58 & \\\hline\hline
resource not found              &     59 & \\\hline\hline
parser error                    &     60 & \\\hline\hline
error waiting on signal         &     61 & \\\hline\hline
error in signal operation       &     62 & \\\hline\hline
error setting signal set        &     63 & \\\hline\hline
protocol error                  &     64 & \\\hline\hline
cannot create socket            &     65 & \\\hline\hline
error in bind operation         &     66 & \\\hline\hline
error in listen operation       &     67 & \\\hline\hline
cannot accept                   &     68 & \\\hline\hline
server error                    &     69 & \\\hline\hline
cannot find or use address      &     70 & \\\hline\hline
python interpreter error        &     71 & \\\hline\hline
unknown symbol                  &     72 & \\\hline\hline
user error                      &     73 & \\\hline\hline
unknown error                   &   9999 & \\\hline\hline
\end{longtable}
\end{center}
\egroup


\end{document}
