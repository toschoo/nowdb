\section{Outline}
The low-level C \acronym{api} is not intended
for application development. For this purpose
high-level C and a \CC\ \acronym{api}s exist
(please refer to chapter \ref{chpt_ccpp}).
This \acronym{api} aims instead to ease
the development of client language bindings.
The Python, Lua and Go client \acronym{api}s
are built on it.

The \acronym{api} consists of the \tech{libnowdbclient}
library, which is installed in a canonical library directory
such as \tech{/usr/local/lib},
and the \tech{nowclient.h} interface,
which is intalled in a canonical header directory
such as \tech{/usr/local/include}.

The \acronym{api} provides services for
time conversion, connection management,
result handling and error handling.

\section{Time}
The \acronym{api} defines the \nowdb\ time type,
\tech{nowdb\_time\_t} as \tech{uint64\_t}.
Values of this type represent a \acronym{unix}
timestamp with nanosecond precision.
The smallest and greatest possible values
are defined as 
\begin{itemize}
\item \acronym{nowdb\_time\_dawn}
(`1677-09-21T00:12:44')
and
\item \acronym{nowdb\_time\_dusk}
(`2262-04-11T23:47:16')
respectively.
\end{itemize}

The standard formats are defined as
\begin{itemize}
\item \acronym{nowdb\_time\_format}
\tech{"\%Y-\%m-\%dT\%H:\%M:\%S"} and
\item \acronym{nowdb\_date\_format}
\tech{"\%Y-\%m-\%d"}. 
\end{itemize}

The following time conversion functions are available:

\subsubsection{nowdb\_time\_fromUnix}
The function receives a \acronym{posix} \tech{struct timespec}
and returns a \tech{nowdb\_time\_t}. The conversion never fails.

\subsubsection{nowdb\_time\_toUnix}
The function receives a \tech{nowdb\_time\_t} 
and a pointer to a \acronym{posix} \tech{struct timespec},
which must not be \acronym{null}.
The function returns an \tech{int} representing
an error code (see chapter \ref{chpt_errors}).
The conversion fails if the pointer to the \term{timespec}
structure is \acronym{null} or when the
\tech{nowdb\_time\_t} value is out of range
(which can happen only with custom time configurations).

\subsubsection{nowdb\_time\_parse}
The function receives
\begin{itemize}
\item a time string (which must not be \acronym{null}),
\item a format string (which must not be \acronym{null}) and
\item a pointer to a \tech{nowdb\_time\_t}
(which must not be \acronym{null}).
\end{itemize}

The function returns an \tech{int} representing
an error code (see chapter \ref{chpt_errors}).

The function attempts to parse the time string
according to the format string and, if successful,
stores the result under the address passed in as third parameter.

\subsubsection{nowdb\_time\_show}
The function receives
\begin{itemize}
\item a \tech{nowdb\_time\_t}
\item a format string (which must not be \acronym{null}),
\item a \tech{char} buffer (which must not be \acronym{null}) and
\item a \tech{size\_t} indicating the size of the buffer.
(which must not be \acronym{null}).
\end{itemize}

The function returns an \tech{int} representing
an error code (see chapter \ref{chpt_errors}).

The function writes a string representation
of the \tech{nowdb\_time\_t} value
according to the format string
into the \tech{char} buffer.
The function fails (among other reasons),
if the buffer is not big enough to hold the result.

\subsubsection{nowdb\_time\_get}
The function expects no argument and returns a \tech{nowdb\_time\_t}
representing the current time. On failure,
the function returns \acronym{nowdb\_time\_dawn}
(which is certainly not \emph{now}).

\section{Connection}
The type \tech{nowdb\_con\_t} represents a
\acronym{tcp/ip} connection to the database.
The type is defined as pointer to an
\term{anonymous struct} and cannot be
allocated by the user.

Note that connections are not \term{threadsafe}.
Threads must not access connection objects
concurrently.

For connection management the following services
are available:

\subsubsection{nowdb\_connect}
The function receives six arguments:
\begin{itemize}
\item a pointer to a \tech{nowdb\_con\_t}
(which must not be \acronym{null}),
\item a string
(which must not be \acronym{null})
defining the \term{node}
on which the database is running and which
may be an \acronym{ip} \tech{v4} or \tech{v6} address or
a hostname.
\item a string
(which must not be \acronym{null})
defining the service which
is usually the port to which the database is listening,
\item a string
(which may be \acronym{null})
defining the user (\comment{currently not used}),
\item a string
(which may be \acronym{null})
defining the password of that user (\comment{currently not used}) and
\item an \tech{int} representing connection flags.
\end{itemize}

The function returns an \tech{int} representing
a client error code (see section \ref{sec_clnterrors}).

The function attempts to connect to the database server.
On success (when the return value is \acronym{ok}),
the \tech{nowdb\_con\_t} pointer is guaranteed
to point to a valid connection object.
Otherwise, no memory is allocated.

The flags represent connection options.
Currently, the following options are available:
\begin{itemize}
\item \acronym{nowdb\_flags\_text}:
results for this session will be sent
in a textual (\acronym{csv}) format.
\comment{Not available}

\item \acronym{nowdb\_flags\_le}:
results for this session will be sent
in binary \term{little endian} format.

\item \acronym{nowdb\_flags\_be}:
results for this session will be sent
in binary \term{big endian} format.
\comment{Not available}
\end{itemize}

\subsubsection{nowdb\_connection\_close}
The function receives a \tech{nowdb\_con\_t}
and returns an \tech{int} representing
a client error code (see section \ref{sec_clnterrors}).

The function terminates the connection.
On success, it is guaranteed that all
resources have been freed.
Otherwise, when an error code is returned,
resources are still allocated and must be
freed using \tech{nowdb\_connection\_destroy}.

\subsubsection{nowdb\_connection\_destroy}
The function receives a \tech{nowdb\_con\_t}.
The function frees all resources allocated to this connection.
The function never fails and
is declared as \tech{void}.

\section{Result}
The \acronym{api} implements the four
\nowdb\ dynamic types:

\begin{itemize}
\item \acronym{nowdb\_result\_status}
\item \acronym{nowdb\_result\_report}
\item \acronym{nowdb\_result\_cursor}
\item \acronym{nowdb\_result\_row}
\end{itemize}

Three anonymous structs are defined to
represent these types:
\begin{itemize}
\item \tech{nowdb\_result\_status} represents
\begin{itemize}
\item\acronym{nowdb\_result\_status} and
\item\acronym{nowdb\_result\_report},
\end{itemize}
\ie\ the services defined for \term{status}
also accept an object of type \term{report}
and \textit{vice versa};
\item \tech{nowdb\_result\_cursor} 
represents \acronym{nowdb\_result\_cursor};
\item \tech{nowdb\_result\_row} 
represents \acronym{nowdb\_result\_row}
\end{itemize}

\subsection{Execution}
\subsubsection{nowdb\_exec\_statement}
The function receives
\begin{itemize}
\item a connection
\item a string representing an \sql\ statement
\item a pointer to a result object
\end{itemize}

None of those must be \acronym{null}.

The function returns an \tech{int} representing
a client error code.

The function sends the \sql\ statement
to the database and waits for the result.
On success,
the \tech{nowdb\_result\_t} pointer is guaranteed
to point to a valid result object.
Otherwise, no additional memory is allocated.

\subsubsection{nowdb\_exec\_statementZC}
This function is a \term{zerocopy} variant
of \tech{nowdb\_exec\_statement}.
\tech{nowdb\-\_\-exec\-\_\-statement} copies all
data received from the database to a private
buffer that belongs to the result object.
This allows user programs to process
interleaving queries, \ie\
statements can be executed,
while others still have pending results.

In some cases, this is not necessary,
in particular when the result is just
a status that is checked once immediately
after \tech{exec} has returned.
In such cases
\tech{nowdb\-\_exec\-\_statementZC}
might be more efficient, because
it does not copy the result data into a private buffer,
but leaves them in the connection object.
The next query, hence, will overwrite
those data.

Note that \tech{nowdb\_exec\_statementZC}
is not allowed when the result
may be a cursor.

\subsection{Status and Report}
\subsubsection{nowdb\_result\_errcode}
The function receives a \tech{nowdb\_result\_t}
and returns its error code which is a server-side
error code (or 0 for success).

\subsubsection{nowdb\_result\_eof}
The function receives a \tech{nowdb\_result\_t}
and returns an \tech{int} which is $\neq 0$
if the error code is \term{end-of-file}
and 0 otherwise.

\subsubsection{nowdb\_result\_details}
The function receives a \tech{nowdb\_result\_t}
and returns a constant string providing details
on the error represented by the status.
If no error has occurred, the result string
is \tech{"OK"}.

\subsubsection{nowdb\_result\_report}
The function is declared as \tech{void}.
The function receives a \tech{nowdb\_result\_t}
and three more parameters:
\begin{itemize}
\item a pointer to a \tech{uint64\_t}
which must not be \acronym{null}
and to which the number of affected rows
is written; 
\item a pointer to a \tech{uint64\_t}
which must not be \acronym{null}
and to which the number of errors
is written; 
\item a pointer to a \tech{uint64\_t}
which must not be \acronym{null}
and to which the running time in microseconds
is written.
\end{itemize}

\subsubsection{nowdb\_result\_destroy}
The function is declared as \tech{void}.
The function receives a \tech{nowdb\_result\_t}
and frees all resources assigned to it.

\subsection{Cursor}
\subsubsection{nowdb\_cursor\_open}
The function receives a \tech{nowdb\_result\_t}
and a pointer to a \tech{nowdb\_cursor\_t}
(which must not be \acronym{null}).
It returns an \tech{int} representing
a client error code.
The function \emph{casts} the result
passed in to the pointer address.
The function fails if the result
does not not represent a valid result
or if that result was created
with \term{zerocopy} option.
Note that \tech{nowdb\_exec\_statementZC}
is not allowed when the result
may be a cursor.

\subsubsection{nowdb\_cursor\_errcode}
The function receives a \tech{nowdb\_cursor\_t}
and returns its error code.

\subsubsection{nowdb\_cursor\_details}
The function receives a \tech{nowdb\_cursor\_t}
and returns a constant string providing error details.
If no error has occurred, the result string
is \tech{"OK"}.

\subsubsection{nowdb\_cursor\_eof}
The function receives a \tech{nowdb\_cursor\_t}
and returns an \tech{int} which is $\neq 0$
if the error code is \term{end-of-file}
and 0 otherwise.

\subsubsection{nowdb\_cursor\_ok}
The function receives a \tech{nowdb\_cursor\_t}
and returns an \tech{int} which is $\neq 0$
if the error code is \acronym{ok}
and 0 otherwise.

\subsubsection{nowdb\_cursor\_id}
The function receives a \tech{nowdb\_cursor\_t}
and returns a \tech{uint64\_t} which
represents the identifier under which this
cursor is known in the server.

\subsubsection{nowdb\_cursor\_fetch}
The function receives a \tech{nowdb\_cursor\_t}
and returns an \tech{int} which
represents a client error code.
On success, the cursor fetches the next
bulk of rows from the server.
If there are no more rows in the server,
the error code of the cursor passes to \term{end-of-file}.

\subsubsection{nowdb\_cursor\_row}
The function receives a \tech{nowdb\_cursor\_t}
and returns \tech{nowdb\_row\_t}.
Note that the result is not a copy
of the rows in the cursor, but a reference
to those.
It is therefore not necessary to destroy
the returned rows.
On the other hand, the rows are lost
when either \tech{fetch} or \tech{close}
is called on the cursor. 

\subsubsection{nowdb\_cursor\_close}
The function receives a \tech{nowdb\_cursor\_t}
and returns an \tech{int} which represents an
error code.
It sends a close request for this cursor
to the database and, on success,
frees all resources assigned to this cursor.
Otherwise, if the close request fails,
the cursor must be destroyed using 
\tech{nowdb\_result\_destroy} and
casting the cursor to a \tech{nowdb\_result\_t}

\subsection{Row}
\subsubsection{nowdb\_row\_next}
The function receives a \tech{nowdb\_row\_t}
and returns an \tech{int} that represents
a client error code.
It advances to the next row.
If the current row was already the last one,
\term{end-of-file} is returned.

\subsubsection{nowdb\_row\_rewind}
The function is declared as \tech{void}.
The function receives a \tech{nowdb\_row\_t}.
It resets the row struct to the first row.

\subsubsection{nowdb\_row\_field}
The function receives a \tech{nowdb\_row\_t}
and two more parameters, namely
\begin{itemize}
\item an \tech{int}, $n$, indicating
that we want to obtain the $n$th field
starting to count from 0
for the first field in the row;
\item a pointer to an \tech{int}
which must not be \acronym{null}
and is set to the type of the field.
\end{itemize}
The function returns the address
of the first byte of the $n$th field
or \acronym{null} on error.

Valid types are
\begin{itemize}
\item \acronym{nowdb\_typ\_uint}
\item \acronym{nowdb\_typ\_int}
\item \acronym{nowdb\_typ\_float}
\item \acronym{nowdb\_typ\_time}
\item \acronym{nowdb\_typ\_text}
\item \acronym{nowdb\_typ\_bool}
\end{itemize}

\subsubsection{nowdb\_row\_copy}
The function receives and returns
a \tech{nowdb\_row\_t} and
copies the row passed in allocating
new memory. Rows created with copy
must be destroyed using \tech{nowdb\_result\_destroy}
casting the row to the generic result type.

\subsubsection{nowdb\_row\_write}
The function receives a \acronym{file} pointer and
a \tech{nowdb\_row\_t}.
It returns an \tech{int} representing
a client error code.
The function writes a textual representation
of the row(s) passed in to the file.

\section{Error Handling}
\subsubsection{nowdb\_err\_explain}
The function receives a (client or server) error code
and returns a constant string explaining the error.

\begin{minipage}{\textwidth}
\section{Error Codes}\label{sec_clnterrors}
\bgroup
\renewcommand{\arraystretch}{1.3}
\begin{center}
\begin{longtable}{||l||c||l||}\hline
\textbf{Error Name} & \textbf{Numerical Code} & \textbf{Meaning} \\\hline\endhead\hline
out of memory                             &     -1 & \\\hline\hline
no connection                             &     -2 & \\\hline\hline
socket error                              &     -3 & \\\hline\hline
error on address                          &     -4 & \\\hline\hline
cannot create result                      &     -5 & \\\hline\hline
invalid parameter                         &     -6 & \\\hline\hline
error on read operation                   &   -101 & \\\hline\hline
error on write operation                  &   -102 & \\\hline\hline
error on open  operation                  &   -103 & \\\hline\hline
error on close operation                  &   -104 & \\\hline\hline
use statement failed                      &   -105 & \\\hline\hline
protocol error                            &   -106 & \\\hline\hline
statement or requested resource too big   &   -107 & \\\hline\hline
operating system error (check errno)      &   -108 & \\\hline\hline
time or date format error                 &   -109 & \\\hline\hline
cursor with zerocopy requested            &   -110 & \\\hline\hline
cannot close cursor                       &   -111 & \\\hline
\end{longtable}
\end{center}
\egroup
\end{minipage}
