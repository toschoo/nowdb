\section{Outline}
\ignore{
- why lua
- what it provides
- how it works
}
The main server interface
besides \sql\ is Lua.
While \sql\ is intentionally kept simple in \nowdb,
it is in fact a mere data access language
without control structures and variables,
Lua is a full-fledged (but still simple)
programming language that can be used to
implement complex processing logic.
The fact that the Lua environment is hosted
on the server implies that the overhead
for communicating with the storage engine
and getting data in and out is very small
compared to the client/server round-trip.
Typical use cases for server-side Lua programming
are therefore:

\begin{itemize}
\item implementing complex queries that require
      repeated data access or complex joins;
\item building algorithms that are beyond simple queries
      like path finding or recommendation engines; 
\item data preparation for data science tasks
      like grouping, statistical sampling or
       feature extraction; 
\item background tasks to compute statistics or
      perform housekeeping and backups
      and other \acronym{dba} jobs;
\item active server tasks like publish and subscribe,
      monitoring or alerts;
\item tasks that depend on in-memory state.
\end{itemize}

The server-side \acronym{api} is similar
to its client-side cousins. It provides
an interface to execute \sql\ statements with
a number of convenience interfaces for common
use cases; a polymorphic result type
to communicate \sql\ results from the database
to user code and
from user code to the client; services for
time manipulation and error handling routines.

The \nowdb\ Lua environment itself
lives in the session.
When a client connects
(and the Lua support
is activated in the \nowdb\ daemon),
the Lua interpreter for this session is initialised
and the Lua standard modules 
as well as the main Lua module are loaded
into the session.
Further modules, in particular the toplevel modules
containing the user code for stored procedures,
are loaded when needed, that is, on the first
execution of a specific stored procedure.

The main Lua module defines the table 
\term{nowdb} which contains all
functions and constants defined by
the \nowdb\ Lua environment.
This table is a global variable,
so that this functionality is immediately available
when the session starts.
It is not necessary to import this module explicitly
in user code.

In order to find the user-defined Lua code,
\nowdb\ inspects the environment variable
\tech{NOWDB\_LUA\_PATH}. \nowdb\ expects
this variable to contain
semicolon-separated definitions of the form

\tech{db:path}

where \term{db} is the name of a database and
\term{path} is an ordinary \acronym{posix} path.
\nowdb\ will then search for a specific module
in the path given for the database currently
in use (defined by the most recent
\sql\ \keyword{use} statement).

For instance, if a procedure in database \term{mydb}
was created as

\keyword{create procedure} \identifier{mymodule.myprocedure}()
\keyword{language} \identifier{lua}

\nowdb\ will search for the key \term{mydb} in
\tech{NOWDB\_LUA\_PATH} and search in the \term{path}
indicated after the colon for a file called
\term{mymodule.lua}.

It is possible to define a wildcard that matches all databases.
The wildcard is \tech{*}, \eg:

\tech{*:/path/to/lua}

If such an entry exists 
in \tech{NOWDB\_LUA\_PATH},
but not an entry for the current database,
then modules will be searched in this path.

The rationale for providing a wildcard is that
new databases may be created on the fly in an already
running \nowdb\ daemon with an environment variable
that does not yet contain the name of this new database.
Toplevel modules for such new databases
should then be stored in the wildcard path.

An issue with this approach is that names
of toplevel modules in the wildcard path must be unqiue;
this is not true for modules of different
databases living in different paths.
When the session switches from one database to the other,
by issuing a new \keyword{use} statement,
the modules of the old database will be unloaded.

In general, database-specific paths are always 
searched before the wildcard path
and the first module found will be
loaded.

\ignore{
what happens when I first use one database and then
the other potentially with equally named modules?
}

\section{Execute}
Most interactions with the \nowdb\ server
are performed by one of the varieties of
the \term{execute} function.
The most fundamental is \term{nowdb.execute},
which receives a string containing an \sql\ statement,
returns a result or raises an error (\ie\ calls
the Lua \term{error} function), \eg:

\begin{lua}
\begin{lstlisting}
local cur = nowdb.execute([[select * from product]])
\end{lstlisting}
\end{lua}

In some situations, it might be inconvenient
that a function raises an error.
For this cases, there is a protected
variant called \term{nowdb.pexecute}.
It also expects an \sql\ statement
as input, but returns two return values:
an error code and a result or string.
If the error code is \term{nowdb.OK},
the second return value
is a polymorphic result type
(a status, report, cursor or row).
Otherwise, it is a string describing
the error with more details, \eg:

\begin{lua}
\begin{lstlisting}
local ok, cur = nowdb.pexecute([[select * from product]])
if ok ~= nowdb.OK then
   -- error handling
end
-- handle cursor
\end{lstlisting}
\end{lua}

Another variant of \term{execute} is
the function \term{nowdb.execute\_}.
This is a convenience interface for
situations where the programmer does
not want to deal with a result at all.
The function returns no result, but
raises an error if something goes wrong.
typical use cases may be:

\begin{lua}
\begin{lstlisting}
nowdb.execute_([[create type product(
                   key uint primary key,
                   desc       text,
                   base_price float)]])

nowdb.execute_([[insert into product (key, desc, base_price)
                             values  (1010, 'cool product', 1.49)]])
\end{lstlisting}
\end{lua}

Often queries can only have one result row or even
just one value. In such cases, the functions above
are not convenient, because they force the programmer
to write all the boilerplate code for cursors.
There are two functions to handle this kind of
situations: \term{nowdb.onerow} and
\term{nowdb.onevalue}.

\term{nowdb.onerow} accepts an \sql\ statement
and returns a Lua array representing one
result row, \eg:

\begin{lua}
\begin{lstlisting}
local r = nowdb.onerow([[select * from product where key = 12345]])
\end{lstlisting}
\end{lua}

where $r$ is a Lua array.
If no data are available for the specific query,
the array is empty.

\term{nowdb.onevalue} behaves like \term{onerow},
but returns only one single value, \eg:

\begin{lua}
\begin{lstlisting}
local d = nowdb.onevalue([[select desc from product where key = 12345]])
\end{lstlisting}
\end{lua}

where $d$ is a single Lua value.

Both functions raise an error when something goes wrong.

One might be tempted to write code like this

\begin{lua}
\begin{lstlisting}
local p = nowdb.onevalue([[select pi()]])
\end{lstlisting}
\end{lua}

This, however, would fail, since queries
that have only a projection clause (\ie\ no \term{from} close)
do not produce a cursor, but a row (see chapter \ref{chpt_sql}).
\term{onerow} and \term{onevalue}, however, expect
statements that produce a cursor and fail otherwise.
The correct way to write the above code is

\begin{lua}
\begin{lstlisting}
local r = nowdb.execute([[select pi()]])
local p = r.field(0)
r.release()
\end{lstlisting}
\end{lua}

Since this is a lot of boilerplate,
there is a function to handle such cases,
namely \term{nowdb.eval} which evaluates
an \sql\ expression. It expects a string that
contains only the expression, \eg\
$pi()$, $1+2$, $now()$, \etc\
The above code can therefore be written as

\begin{lua}
\begin{lstlisting}
local p = nowdb.eval('pi()')
\end{lstlisting}
\end{lua}

\section{Results}
The \term{execute} functions return
a polymorphic result type (see dynamic types
in chapter \ref{chpt_sql}).
A result always has the following methods:
\begin{itemize}
\item \term{resulttype}()
      returns a numeric code indicating
      the type of this specific result.
      Result types are:
      \begin{itemize}
      \item \term{nowdb.NOTHING}
      \item \term{nowdb.STATUS}
      \item \term{nowdb.REPORT}
      \item \term{nowdb.ROW}
      \item \term{nowdb.CURSOR}
      \end{itemize}
\item \term{ok}()
      returns a boolean indicating whether
      the result represents an error condition;
      if the return value is $true$,
      the result does not represent an error,
      otherwise, it does.
      The programmer can then inspect the error code
      and the error details method discussed in the
      following subsection.
\item \term{release}()
      releases the C resources associated with the result.
      It is not strictly necessary to release these resources
      explicitly, since the Lua \acronym{gc} takes care of that.
      However, it is a good policy to release results as soon
      as they are not needed anymore; The \acronym{gc} is not
      aware of the size of the C resources and may decide to
      postpone the next cycle although there is already
      a lot of unused memory on the C side.
\end{itemize}

\subsection{Status and Error Handling}
The simplest result type is
the status which represents an error condition.
It is provide two methods, namely
\term{errcode}() and \term{errdetails}().
The former returns a numeric error condition (see appendix
\ref{chpt_errors}), the latter a string
describing the details of the error.
Important error codes that are often used to
decide on how to proceed in the code are

\begin{itemize}
\item \term{nowdb}.\acronym{ok}: successful completion;
\item \term{nowdb}.\acronym{eof}: no more data available;
\item \term{nowdb}.\acronym{nomem}: the server ran out of memory;
\item \term{nowdb}.\acronym{toobig}: the requested resource is too big;
\item \term{nowdb}.\acronym{keynof}: key not found;
\item \term{nowdb}.\acronym{dupkey}: duplicated key;
\item \term{nowdb}.\acronym{timeout}: a timeout occurred;
\item \term{nowdb}.\acronym{notacur}: a cursor method was invoked on a result,
                            that is not a cursor;
\item \term{nowdb}.\acronym{notarow} a row method was invoked on a result,
                            that is not a row;
\item \term{nowdb}.\acronym{usrerr}: error in user code;
\item \term{nowdb}.\acronym{selflock}: attempt to acquire a lock that the caller
                             is currently holding;
\item \term{nowdb}.\acronym{deadlock}: deadlock detected;
\item \term{nowdb}.\acronym{notmylock}: attempt to release a lock that the caller
                              is currently not holding;
\end{itemize}

The programmer may decide to return an error result from
her function. That is not strictly necessary:
if the user-defined function does not return anything,
\nowdb\ will send \term{nowdb.OK} back to the client.
If user code raises an error, \nowdb\ will convert
this error into an error message which is sent back
to the client with error code \term{nowdb}.\acronym{usrerr}.

For explicit error handling
the environment provides functions to
create status codes, namely:

\term{nowdb.success}() creates a status with error code
\term{nowdb}.\acronym{ok};
\term{nowdb.error}(rc, msg) creates a status with error code $rc$ and
error message $msg$.

The programmer can use the function
\term{nowdb.raise}(rc, msg) to immediately terminate
execution. The client will then receive an error
with the error code \acronym{usrerr} and a message string
that contains the error code $rc$ and the message $msg$.
In some situations, it may be preferrable
to explicitly return an error status, since in that case
the calling client receives the intended numerical
error code instead of \acronym{usrerr}.

To ease program flow in face of exceptions,
there are the function \term{nowdb.bracket} and
\term{nowdb.pbracket}.
The functions take three arguments,
which are all functions themselves, called
\term{before}, \term{after} and \term{body}.
Ignoring details of parameter passing
(which in fact is bit more complex),
the implementation of \term{pbracket} looks
like the code snippet:

\begin{lua}
\begin{lstlisting}
function nowdb.bracket(before, after, body)
  local r = before()
  local ok, t = pcall(body,r)
  after(r)
  return ok, t
end
\end{lstlisting}
\end{lua}

Typically $before$ is a function that obtains a resource
and $after$ closes this resource. $body$ is a function
that runs between $before$ and $after$ using the resource.
$bracket$ guarantees that $after$ is called
even when $body$ fails. A use case may be:

\begin{lua}
\begin{lstlisting}
local function before() 
  nowdb.execute_("lock mylock")
end
local function after()
  nowdb.execute_("unlock mylock")
end
local k = nowdb.bracket(before, after, function()
  local x = nowdb.onevalue([[select max(key) from mytable]])
  nowdb.execute_(string.format(
        [[insert into mytable (key) values (%d)]], x+1))
  return x
end)
\end{lstlisting}
\end{lua}

In this code segment, $bracket$ is used to obtain
and increment a unique key from a table.
The resource in this case is a lock held in the database and
just represented by the state $nil$, \ie\
$before$ does not return a value and $after$ and $body$
do not expect values. The point is, however,
that the lock is released independent of $body$
raising an exception or not.

The difference between $bracket$ and $pbracket$ is
that $pbracket$ returns all return values from \term{pcall}
including the leading boolean ($ok$);
$bracket$, by contrast, removes the Boolean
and reraises the error in case the Boolean was $false$

\subsection{Reports}
\comment{tbc}

\subsection{Rows}
Row results represent single lines of a result set.
They appear naturally with cursors which provide
methods to iterate through the the rows of a result set
and this is basically the only way where they
appear on client side. On server-side there are two
other occasions where rows appear:
a stored procedure may return a row (or bunch of rows).
On client side, this is not the case, because \nowdb\
transforms rows into cursors before they are sent to
the client. On the server, however, rows are returned in
their ``raw'' form.
The second occasion is projections
without a \keyword{from} clause.

A row contains one or more typed values.
There two methods that provide access to these values:
\term{field} and \term{typedfield}.

Both take on argument, namely the index of the field
starting to count from 0. the difference is that
\term{field} returns only the value at that position
in the row. \term{typedfield}, by contrast,
returns a \nowdb\ type code and the field.
The \nowdb\ type codes are:

\begin{itemize}
\item \term{nowdb}.\acronym{nothing}
\item \term{nowdb}.\acronym{text}
\item \term{nowdb}.\acronym{date}
\item \term{nowdb}.\acronym{time}
\item \term{nowdb}.\acronym{float}
\item \term{nowdb}.\acronym{int}
\item \term{nowdb}.\acronym{uint}
\item \term{nowdb}.\acronym{bool}
\end{itemize}

Here is a simple usage example:

\begin{lua}
\begin{lstlisting}
local row = nowdb.execute("select pi(), e()")
print(string.format(
  [[pi is            : %.8f
    Euler's number is: %.8f]],
  row.field(0), row.field(1)))
row.release()
\end{lstlisting}
\end{lua}

The method \term{countfields}() returns
the number of fields in a row.
Here is a usage example that also
illustrates the use of \term{typedfield}:

\begin{lua}
\begin{lstlisting}
for i in 0, row.countfields() do
  local t, v = row.typedfield(i)
  print(string.format(
    "field %d is of type %d", i, t))
end
\end{lstlisting}
\end{lua}

There are two helper functions to convert
type code into type strings and vice versa,
namely \term{nowdb.nowtypename},
which expects a type code and returns a string
(\eg\ \term{nowdb}. \acronym{float} is converted
to 'float') and
\term{nowdb.nowtypebyname}, which expects
a string and, if this string is recognised
as the name of the type, returns the corresponding
type code; valid type strings are
'text', 'date', 'time', 'uint', 'uinteger',
'int', 'integer', 'float', 'bool', 'boolean',
'null' and 'nil'.

Rows are an important means to return 
results back to the calling client.
A row is created with the function
\term{nowdb.makerow}, which takes
no arguments and returns an empty row.

The method \term{add2row} can
be used to add a value to the row.
The method takes two arguments:
the first is the \nowdb\ type of the value
the second is the value itself.
A row is closed (\ie\ an \term{end-of-row} marker
is inserted) with the method
\term{closerow}, which takes no arguments
and returns nothing.

The following code snippet is an example:

\begin{lua}
\begin{lstlisting}
function getconstants()
  local row = nowdb.makerow()
  row.add2row(nowdb.FLOAT, 2.718281828)
  row.add2row(nowdb.FLOAT, 3.141592653)
  row.closerow()
  return row
end
\end{lstlisting}
\end{lua}

A very common use case is to create a row from
an array. There is a special function to do this:
\term{nowdb.array2row}, which takes two arguments
and returns a row.

The first argument is an array of \nowdb\ types,
the second argument is an array of values.
The two arrays shall have the same number
of elements, otherwise, an error is raised.
The function creates a row and adds the values
with the matching type in the first array, \eg:

\begin{lua}
\begin{lstlisting}
function getconstants()
  local r = {2.718281828, 3.141592653}
  return nowdb.array2row({nowdb.FLOAT, nowdb.FLOAT}, r)
end
\end{lstlisting}
\end{lua}

The inverse of \term{array2row} is the row method
\term{row2array}, which receives a row and returns
two arrays: the first containing the values,
the second containing the types.
The following code example, hence, is
somewhat pointless, but it illustrates the
behaviour:

\begin{lua}
\begin{lstlisting}
local row = getconstants() -- constants in row
local vs, ts = row.row2array() -- row to arrays
return nowdb.array2row(ts, vs) -- arrays back to row
\end{lstlisting}
\end{lua}

Another convenience interface is \term{nowdb.makeresult},
which creates a row from a single value. The function
takes two arguments, the type and the value itself
and returns a row containing this value as its
only field. A very common use case for this function is to
to communicate a single value back to the calling client.
Here is a, perhaps, somewhat artificial example:

\begin{lua}
\begin{lstlisting}
function integral(ts, fld, k, t0, t1)
  local stmt = string.format(
    [[select stamp/%d, %s from %s
       where origin = %d
         and stamp >= %d and stamp < %d
       order by stamp
    ]], nowdb.hour, fld, ts, k, t0, t1)
  local cur = nowdb.execute(stmt)
  local x_1 = 0
  local x_2 = 0
  local x = 0
  local first = true
  for row in cur.rows() do
      if not first then x_1 = x_2 end
      x_2 = row.field(0)
      if not first then
         local d = x_2 - x_1
         local y = row.field(1)
         x = x + d*y
      else
         first = false
      end
  end
  cur.release()
  return nowdb.makeresult(nowdb.FLOAT, x)
end
\end{lstlisting}
\end{lua}

The function computes an approximation
to the area under the curve
formed by the time series `ts' with origin $k$
and timestamp as the $x$-value and field 'fld'
as the $y$-value.

\subsection{Cursors}
Cursors are iterators over data in the database,
either a vertex type or an edge.
A successful execution of a \keyword{select}
statement almost always returns a cursor;
the only exception being projections without
a \keyword{from} statement.

The main functionality provided by cursors
is the \term{rows} method, an iterator factory
for the rows in the
result set of the statement.
The basic pattern is

\begin{lua}
\begin{lstlisting}
local cur = nowdb.execute(stmt)
for row in cur.rows()
    -- do something
end
cur.release()
\end{lstlisting}
\end{lua}

A more elaborate example is the \term{integral}
function in the previous section.

Internally, \term{rows} uses the methods
\term{open}, which starts the iteration,
\term{fetch}, which obtains the next bunch
of rows from the storage engine and
\term{nextrow}, which advances to
the next row within the current bunch of rows.
For user code, it is rarely necessary to use
these methods directly, but it may be useful
in some situations.

The simple pattern above can be built from
these building blocks in the following way:

\begin{lua}
\begin{lstlisting}
local cur = nowdb.execute(stmt)
local rc, msg = cur.open()
if rc ~= nowdb.OK then nowdb.raise(rc, msg) end
rc, msg = cur.fetch()
if rc ~= nowdb.OK then
   nowdb.raise(rc, msg)
end
while true do
  -- do something
  -- use row methods (field, typedfield, etc.)
  -- to obtain access to data
  local rc = cur.nextrow()
  if rc ~= nowdb.OK then
     local rc, msg = cur.fetch()
     if rc == nowdb.EOF then break end
     if rc ~= nowdb.OK then
        nowdb.raise(rc, msg)
     end
  end
end
cur.release()
\end{lstlisting}
\end{lua}

\term{open} and \term{fetch} return an error code
and, if this error code is not \term{nowdb}.\acronym{ok},
an error message. The methods have effect on
the internal state of the cursor, but do not return
the state directly. In fact, the \term{row} methods
like \term{field} and \term{typedfield} can be called
directly on the cursor.

Data are immediately available,
when the cursor was opened and \term{fetch}
was called on it.
The internal offset which points to the current row
within the bunch of row no held by the cursor is
incremented with the method \term{nextrow}.
The method returns either
\term{nowdb}.\acronym{eof},
\term{nowdb}.\acronym{notacur},
or \term{nowdb}.\acronym{ok}.
Details are not available for this method.

\begin{minipage}{\textwidth}
The logic, hence, is:
\begin{enumerate}
\item open
\item fetch
\item process the data
\item nextrow until \acronym{eof}
\item if \acronym{eof} then fetch
\item if \acronym{eof} again, terminate the process
\item otherwise, go to step 3 
\end{enumerate}
\end{minipage}

Contrary to rows, cursors cannot be created
``out of thin air''; cursors can only be created
as results of a \keyword{select} statement.
But they can be returned to the calling client.
Here is a simple difference engine that
illustrates this technique:

\begin{lua}
\begin{lstlisting}
function differences(ts, fld, k, t0, t1)
  local stmt = string.format(
    [[select stamp/%d, %s from %s
       where origin = %d
         and stamp >= %d and stamp < %d
       order by stamp
       ]], nowdb.hour, fld, ts, k, t0, t1)
  local y_1 = 0
  local y_2 = 0
  local first = true
  local cur = nowdb.execute(stmt)
  for row in cur.rows() do
      if not first then
         y_1 = y_2
      end
      y_2 = row.field(1)
      if not first then
         local d = y_2 - y_1
         nowdb.execute_(string.format(
           [[insert into babbage (origin, destin, stamp, diff)
                          values (%d, 0, %d, %f)
           ]], k, row.field(0), d))
      else
         first = false
      end
  end
  return nowdb.execute(string.format(
         [[select * from babbage
            where origin = %d]], k))
end
\end{lstlisting}
\end{lua}

It may not be satisfactory in many cases
that we have to store the results in another table,
before we can pass them back to the client.
Instead, we would like to emulate a cursor, so that
the client code can iterate over the results
generated by the server as if it was a single cursor.

In such cases, we use the cursor to maintain state,
while each fetch from the client would trigger
additional computations over the data produced
by the cursor. This behaviour is implemented
in the \nowdb\ Lua support package (see section
\ref{sec_luasupp}). But it can also be built
directly from the \nowdb\ module.
The main building block is coroutines and,
indeed, there is a convenience interface
for \term{execute} that we have not yet mentioned,
namely \term{nowdb.cocursor}.
The function accepts an \sql\ statement and
returns a coroutine that yields a single row
on each \term{resume}.

We could iterate over all rows like this:

\begin{lua}
\begin{lstlisting}
local co = nowdb.cocursor(stmt)
while true do
  if coroutine.status(co) == 'dead' then break end
  ok, row = coroutine.resume(co)
  if not ok then nowdb.raise(nowdb.USRERR, row) end
  -- do something with row
end
\end{lstlisting}
\end{lua}

There is a function that does exactly this:
\term{nowdb.corows}. This function returns
and iterator, so that we can simplify to:

\begin{lua}
\begin{lstlisting}
local co = nowdb.cocursor(stmt)
for row in nowdb.corows(co) do
  -- do something with row
end
\end{lstlisting}
\end{lua}

Assuming that the producer as well as some of the other variables
are held in a module-wide variable,
we could implement a \term{fetch} for the difference engine
that works roughly like this:

\begin{lua}
\begin{lstlisting}
function fetchbabbage()
  local y_1 = 0
  local res = nowdb.makerow()
  for i = 1,100 do
    if coroutine.status(co) == 'dead' then break end
    ok, row = coroutine.resume(co)
    if not ok then nowdb.raise(nowdb.USRERR, row) end
    if not first then
       y_1 = y_2
    end
    y_2 = row.field(1)
    if not first then
       local d = y_2 - y_1
       res.add2row(nowdb.UINT, k)
       res.add2row(nowdb.TIME, row.field(0))
       res.add2row(nowdb.FLOAT, d)
       res.closerow()
    end
  end
  return res
end
\end{lstlisting}
\end{lua}

The function creates a bunch of
up to hundred rows each time it is called
and sends them to the client.
The rows are computed in the same way
as in the previous implementation of
the difference engine;
the variables $y\_2$, $first$ and
the coroutine \term{co} are stored
in a module-wide variable.
Of course, in production code,
that state would be stored
in a Lua table with some kind
of key to identify the query
currently processed. But that
is not essential for the point
here: that we can emulate cursors
using global or module-wide state.

\section{Time}
The \nowdb\ Lua \acronym{api}
provides services to convert
data from \nowdb\ format to
the Lua time format (which is also
used in the \term{os} module)
and back to the \nowdb\ format.
The conversion functions are
called \term{nowdb.from} and \term{nowdb.to};

\term{nowdb.from} expects a \nowdb\ timestamp,
\ie\ an integer, and returns a Lua table with
the fields

\begin{itemize}
\item year: the year as 4-digit number;
\item month: the month (1--12);
\item day: the day of th month (1--31);
\item wday: the day of the week starting at sunday with 1;
\item yday: the day of the year starting with January, 1, as 1;
\item hour: the hour of the day (0--23);
\item min: the minute of the hour (0--59);
\item sec: the second of the minute (0--60);
\item nano: the nanosecond of the second (0--999999999).
\end{itemize}

The opposite conversion \term{nowdb.to} expects
such a table and returns a \nowdb\ timestamp.
Note that the table must have all fields
(except \term{wday}, \term{yday} and \term{nano}), otherwise,
an error will be raised.

Lua time tables can be created on the fly
by \term{nowdb.time} or \term{nowdb.date}.
\term{nowdb.time} expects up to seven parameters:
\term{year}, \term{month}, \term{day},
\term{hour}, \term{minute}, \term{second}
and \term{nano}, but only the first three
are mandatory. The \term{nowdb.date} function
is just a wrapper around \term{nowdb.date} and
equivalent to \term{nowdb.time} with only
the first three parameters.

Lua time tables can also be transformed into time or date
strings using one of the functions
\term{nowdb.timeformat} or
\term{nowdb.dateformat}.
The first function converts the timestamp
into a canonical time string and the latter
converts it into a canonical date string.
\term{timeformat} expects all fields in the
table to be present (with the same exception
as \term{nowdb.to});
for \term{dateformat} only \term{year},
\term{month} and \term{day} are necessary.

The function \term{nowdb.getnow}() returns
the current system time as \nowdb\ timestamp
(as always in \acronym{utc}).

\nowdb\ timestamps can be rounded by use of
\term{nowdb.round}, which expects a \nowdb\ timestamp
and a unit (which is an unsigned integer)
and returns a new \nowdb\ timestamp.
Time units are
\term{nowdb.second},
\term{nowdb.minute},
\term{nowdb.hour},
\term{nowdb.day} and
\term{nowdb.year}.
Care should be taken with larger units;
This function is not aware of leap seconds
and other time subtleties. If precision is needed,
\sql\ should be used to perform time arithmetic.

\section{Lua Support Modules}\label{sec_luasupp}
The \nowdb\ Lua Support Package is a collection
of useful modules to extend the server-side
scripting capabilities and to ease common tasks.
The collection contains currently:

\begin{itemize}
\item The \acronym{ipc} Module:
      a collection of inter-session
      communication means, in particular
      \begin{itemize} 
      \item locking
      \item events and
      \item queues
      \end{itemize} 
\item Unique Identifiers:
      services to obtain and manage unique identifiers.

\item The Result Cache Library:
      services for result caches with various
      expiration strategies;

\item Virtual Cursors:
      a module to ease stateful computation and,
      in particular, cursor-like processing (request-and-fetch).
      \comment{not yet...}
       
\item Numnow:
      Is a Lua implementation of NumPy
      base on the open source project Lunum.
      \comment{not yet...}
\item Stats
      \comment{not yet...}
\item The Sampling Library
      \comment{not yet...}
\item Feature Extraction
      \comment{not yet...}
\end{itemize}

