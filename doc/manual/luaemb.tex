\section{Outline}
The \nowdb\ Lua environment lives in a session.
When a client connects
(and the Lua support
is activated in the \nowdb\ daemon),
the Lua interpreter for this session is initialised
and the Lua standard modules 
as well as the main Lua module are loaded
into the session.
Further modules, in particular the toplevel modules
containing the user code for stored procedures,
are loaded when needed, that is, on the first
execution of a specific stored procedure.

The main Lua module defines the table 
\term{nowdb} which contains all
functions and constants defined by
the \nowdb\ Lua environment.
This table is a global variable,
so that this functionality is immediately available
when the session starts.
It is not necessary to import this module explicitly
in user code.

In order to find the user-defined Lua code,
\nowdb\ inspects the environment variable
\tech{NOWDB\_LUA\_PATH}. \nowdb\ expects
this variable to contain
semicolon-separated definitions of the form

\tech{db:path}

where \term{db} is the name of a database and
\term{path} is an ordinary \acronym{posix} path.
\nowdb\ will then search for a specific module
in the path given for the database currently
in use (defined by the most recent
\sql\ \keyword{use} statement).

For instance, if a procedure in database \term{mydb}
was created as

\keyword{create procedure} \identifier{mymodule.myprocedure}()
\keyword{language} \identifier{lua}

\nowdb\ will search for the key \term{mydb} in
\tech{NOWDB\_LUA\_PATH} and search in the \term{path}
indicated after the colon for a file called
\term{mymodule.lua}.

It is possible to define a wildcard that matches all databases.
The wildcard is \tech{*}, \eg:

\tech{*:/path/to/lua}

If such an entry exists in \tech{NOWDB\_LUA\_PATH},
modules will be searched here \emph{and} in the path given
for the database currently in use.

The rationale for providing a wildcard is that
new databases may be created on the fly in an already
running \nowdb\ daemon with an environment variable
that does not yet contain the name of this new database.
Toplevel modules for such new databases
should then be stored in the wildcard path.

An issue with the approach is that names
of toplevel modules must be unqiue;
if modules with the same name exist
in the wildcard path and in a database-specific path,
one of those will be shadowed by the other.
In general, the database-specific path will be always
searched first and the module found first will be
loaded.

\ignore{
what happens when I first use one database and then
the other potentially with equally named modules?
}

\section{Execute}
Once the session has started the \nowdb\ functionality
is available just like that of standard modules.

\section{Results}
\subsection{Status and Error Handling}
\subsection{Reports}
\subsection{Rows}
\subsection{Cursors}

\section{Time}

\section{Lua Support Modules}
