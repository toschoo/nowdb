\section{Getting Started}

The easiest way to get started
is to use the \nowdb\ docker containing
the database server and clients.

\comment{more instructions of how to get it $\dots$}

The docker does not start the database by itself.
You need to start it explicitly. There is a script
called \tech{nowstart.sh} in the root of the docker
that does that.
(You may want to adapt this script to your specific needs!)

Here is one way to start the docker:

\cmdline{
docker run --rm -p 55505:55505 $\backslash$\\
\hspace*{2cm} -v /opt/dbs:/dbs -v /var/log:/log $\backslash$\\
\hspace*{2cm} -d nowdbdocker /bin/bash -c "/nowstart.sh"
}

This command creates the docker container and starts it.
The parameters are
\begin{itemize}
\item \tech{--rm}
instructs the docker daemon to remove
the container immediately after it will have stopped.

\item \tech{-p 55505:55505} binds the host port \term{55505}
to the same docker port.

\item \tech{-v} maps the host path
\tech{/opt/dbs} to the docker path \tech{/dbs} and
the host path \tech{/var/log} to the docker path \tech{/log}.

\item \tech{-d} means the docker runs in the background
(\term{detached}).

\item \tech{nowdbdocker} is the name of the docker image.

\item \tech{/bin/bash} is the command to be executed
within the docker; \tech{-c} passes a command
to be executed to \tech{bash},
namely \tech{/nowstart.sh}.
\end{itemize}

The script \tech{nowstart.sh}, contains the
instructions to start the \nowdb\ daemon.

Looking into the script,
we see it first sets some environment variables:

\cmdline{
export LD\_LIBRARY\_PATH=/lib:/usr/lib:/usr/local/lib
}

This sets the search path for shared libraries.

\cmdline{
export PYTHONPATH=/pynow:\$PYTHONPATH
}

This sets the search path for Python modules
(we will discuss that later).

Then the script starts the daemon iself:

\cmdline{
nowdbd -b /dbs -y 2>/log/nowdbd.log
}

The script passes two options to the daemon:
the base directory where all databases
managed by this particular daemon live (\tech{-b /dbs})
and the \tech{-y} switch, which activates
server-side Python support.

Now the daemon is listening to port 55505
and is ready to respond to database requests.
The daemon starts by printing a welcome banner
to standard output:

\begingroup
\small
\begin{minipage}{\textwidth}
\begin{verbatim}
+---------------------------------------------------------------+ 
 
  UTC 2018-10-09T12:03:21.631000000
 
  The server is ready
 
    - with python support enabled
 
+---------------------------------------------------------------+ 
  nnnn   nnnn          nnnn    nnnnnn       nnnnnn       nnnnnn  
    wi  i   wi       i      i    iw           iw           er   
    wi i     wi     n        n    iw         e  wi        e    
    wii      wi    wi        iw    iw       e    wi      e        
    wi       wi    wi        iw     iw     e      wi    e        
    wi       wi     n        n       iw   e        wi  e        
    wi       wi      i      i         iw e          wie          
   nnnn     nnnn       nnnn             n            n            
+---------------------------------------------------------------+ 

connections: 128
port       : 55505
domain     : any
base path  : /dbs
\end{verbatim}
\end{minipage}
\endgroup

\begin{minipage}{\textwidth}
Here are some more options provided by
the \nowdb\ daemon:

\begin{itemize}
\item \tech{-b} The base directory,
where databases are stored.
(default is the current working directory).

\item \tech{-s} the binding domain, default:
any. If set to a host or a domain, the server
will accept only connections from that host
or domain. Example: \tech{-s localhost} does
only accept connections from the server.
The host or domain can be given as name (\term{localhost})
or as \acronym{ip} address (\tech{127.0.0.1});

\item \tech{-p} the port to which
the server will listen; default is 55505,
but any other (free) port may be used.
 
\item \tech{-c} size of the connection pool.
If the argument is 0, the connection pool grows
indefinitely; otherwise, for \tech{-c n},
$n$ being a positive integer,
the server will create a connection pool
up to $n$ sessions and, all sessions in
the pool are used, refuse to accept
more connections. The default pool size is 128;
\comment{Notice that due to a memory leak
in the Python interpreter, there is no way
to instruct the server to accept indefinitely
many connections, but to only maintain $n$
in the pool.}

\item \tech{-q} runs in quiet mode
(\ie\ no debug messages are printed to standard error);
\item \tech{-n} does not print the starting banner;
\item \tech{-y} activate server-side Python support;
\item \tech{-l} activates server-side Lua support;
\item \tech{-V} prints version information to standard output;
\item \tech{-?} or \tech{-h} prints a help message to standard error.
\end{itemize}
\end{minipage}

\section{First Steps}
Once \term{nowdbd} is running, we can connect to pass
queries to the server. There is a tool to do this from the command line
called \term{nowclient}. Here is a usage example:

\cmdline{
nowclient -d retail -Q "select count(*) from sales where customer=12345"
}

In this form, the client will try to connect to a server
running on the same host and listening to port 55505.
Furthermore, it will request to use the database \term{retail}
and send the query indicated by the \tech{-Q} parameter.

\begin{minipage}{\textwidth}
If successful, the client will print some
processing information to standard error and the query result
to standard output, \eg:

\cmdline{
executing "use retail" \\
OK \\
executing "select count(*) from sales where customer=12345" \\
59
}
\end{minipage}

\begin{minipage}{\textwidth}
With option \tech{-q} we suppress the processing information.
We would then only see the result:

\cmdline{
59
}
\end{minipage}

\begin{minipage}{\textwidth}
Here are more options supported by the client tool:
\begin{itemize}
\item \tech{-s} 
The server address or name, \eg\ \tech{myserver.mydomain.org} or
\tech{127.0.0.1}. Default is \tech{127.0.0.1};

\item \tech{-p}
the port to which the database is listening. Default: 55505;

\item \tech{-d}
the database to which we want to connect. 
Default: no database at all, which means
that we cannot send queries without naming a database.
Below we will look at alternatives to using this parameter;

\item \tech{-Q}
the query we want the database to process;

\item \tech{-t}
print some (server-side) timing information to standard error;

\item \tech{-q}
quiet mode: don't print processing information to standard error.

\item \tech{-V} prints version information to standard output;
\item \tech{-?} or \tech{-h} prints a help message to standard error.
\end{itemize}
\end{minipage}

The client tool is able to read from standard input;
this way, more than one query can be processed by one call
to \term{nowclient}.
The following command processes the same query as the one above,
but uses standard input instead of the options \tech{-d} and \tech{-Q}:

\cmdline{
echo "use retail;select count(*) from sales where customer=12345;" $| \backslash$ \\
\hspace*{2cm} nowclient
}

Notice that, using standard input,
we need to terminate single \sql\ statements
by a semicolon. This is even true for the last statement.
Leaving the semicolon out would lead to an error.

Of course, we can do much more useful things than just
getting rid of the options.
The main point of reading from standard input is
that we can put \sql\ statements into a file and
$cat$ it to nowclient. A useful example may be:

\begin{sqlcode}
\begin{lstlisting}
drop schema retail if exists;
create schema retail; use retail;

create large table sales set stress=constant;
create table statistics;

create medium index idx_sales_eds on sales (edge, destin);
create index idx_sales_eor on sales (edge, origin);

create type product (
        prod_key uint primary key,
        prod_desc text,
        prod_price float
);
create type client (
        client_key uint primary key,
        client_name text
);
create edge buys (
        origin client,
        dest product,
        weight uint,
        weight2 float
);
load '/opt/data/products.csv' into product use header;
load '/opt/data/clients.csv' into client use header;
load '/opt/data/sales.csv' into sales;
\end{lstlisting}
\end{sqlcode}

Let's assume we had this code in a file called
\tech{create\_retail.sql}; then we could
send it to \term{nowclient}:

\cmdline{
cat create\_retail.sql | nowclient
}

which would create the retail database.

The script shows some of the peculiarities of \nowdb.
The beginning is quite regular \sql:

\begin{sqlcode}
\begin{lstlisting}
drop schema retail if exists;
create schema retail; use retail;
\end{lstlisting}
\end{sqlcode}

The first line drops the database retail,
\ie\ it removes all its data physically
from disk. The \term{if exists} part
is included to avoid an error 
(and hence the termination of the script)
in the case
the database does not yet exist.

In the second line the schema `retail'
is created.
The third statement (still in the second line)
instructs \nowdb\ to use the newly created
schema `retail' in all following statements.

The next line is a bit uncommon:

\begin{sqlcode}
\begin{lstlisting}
create large table sales set stress=constant;
\end{lstlisting}
\end{sqlcode}

The statement creates a table called `sales';
\nowdb, however, is not a relational database.
There are no  
`tables' with the meaning of that term
in the relational world.
In fact, tables are just units of storage 
for edges. \comment{A better naming convention
would probably be `tablespace'. The first name,
by the way, was \term{context} $\dots$}

The statement explicitly says that we
want a \emph{large} table and that there
will be \emph{constant} stress (\ie\ ingestion load)
on that table. The \term{create table} statement,
hence, is much more oriented to storage
and processing details than to the logical structure
of the table (as it would be in a relational database).

A similar is true for the index creation:

\begin{sqlcode}
\begin{lstlisting}
create medium index idx_sales_eds on sales (edge, destin);
\end{lstlisting}
\end{sqlcode}

This statement creates a \emph{medium} index on table \term{sales}
with the fields `edge' and `destination' as index keys.
Index sizing is a difficult topic and will be discussed
in chapter \ref{chpt_sizing}.
As a rule of thumb, it is almost always better to assume
small sizing. Large indices have large storage nodes
and those nodes are very often read from and written
to disk. Only when we know that our index will have
many data points per key, it is advisable
to use larger index sizing. The default sizing (used in the next
line) indicatively is \term{small}.

The next two blocks of code create the types
`product' and `client'.
Types describe vertices.
Each database has a set
of vertices that can be connected
by means of edges to form graphs.
The graphs structure all data
in our specific application.
The vertex types thus describe the universe of discourse.
Technically, vertices are stored in
\term{column-oriented} tables (which are
invisible to the user).
\comment{Well, not entirely true at the moment,
but it is the direction to go -- sooner or later
the statement will be true $\dots$}

The attribute types used in the script
are \term{uint}, \term{float} and \term{text}.
The first is a 64bit unsigned integer;
the second is a 64bit floating point number
(a.k.a. \term{double} in languages like C);
\term{text} is a string of up to 255 bytes,
which represent \acronym{utf}-8 characters.
For more information on \sql\ types,
please refer to chapter
\ref{chpt_sql}.

Every vertex type needs a \term{primary key} and
that primary key must consist of only one attribute.
There are no composed keys like in relational
databases.

The next block defines an edge.
Edges are links between vertices.
Contrary to vertices, edges have a fixed structure
with up to seven fields:
\term{edge}, \term{origin}, \term{destination},
\term{label}, \term{timestamp},
\term{weight} and \term{weight2}.

The first field \term{edge} identifies the edge type.
This appears somewhat redundant (from the relational perspective),
but it is indeed necessary, because
edges of different types can be stored in the same table.
The type of the \term{edge} field is \term{text}
and cannot be changed. It is therefore not mentioned in
the script. However, it is always present.

The next two fields, \term{origin} and \term{destination}
(usually shortened to \term{destin} or even \term{dest})
refer to the vertices that are linked by this edge.
The types of \term{origin} and \term{destin} are defined
by the user. The types are restricted to vertex types.
In the script, the types are
\term{client} and \term{product},
which we defined before.

The next field is \term{label}.
Labels can be used
to identify inherent relations between edges.
The type of the label field is defined by the user.
It may be either \term{uint} or \term{text}.
Since we have not defined the label field in the script above,
the label would not be available in this specific edge type.
A useful example using labels is given further below.

With the next field, the timeseries aspect of \nowdb\
comes in. Indeed, edges have a \term{timestamp}.
In our retail application,
edges of type \term{buys}, indicate that a client
bought a certain product \emph{at a given time}.

The timestamp is always present in all edges and
its type cannot be changed.
It always has the type \term{time},
which (usually) is a positive or negative
offset from the \acronym{unix} \term{epoch}
with nanosecond precision.

Finally, edges are weighted. Each edge can be weighted
in two dimensions (\term{weight} and \term{weight2}).
The number $2$ is somewhat arbitrary.
It is in fact a compromise between space and convenience.
One should not interpret too much into it.

The \term{weight} fields can have any basic type
(\term{uint}, \term{int}, \term{float}, \term{text},
\term{time} or \term{bool}).

The fields whose type can be defined by the user,
can also be renamed. We could say, for instance,
\tech{weight uint as quantity}.
The field \term{weight} would then be accessible as
\term{quantity} in \sql\ statements. 
\comment{Renaming is not yet possible.}

In the script above we define four of the fields,
but six will be accessible by the application,
because \term{edge} and \term{timestamp} are always available.
But \term{label} will remain invisible as if it did not exist.

In the final section,
the script loads data from three different \acronym{csv}s
into the database. \nowdb\ provides loaders
for different formats. \acronym{csv} is just one example.
There are many more and the loader even allows 
users to define
their own formats using Apache Avro.
With Avro it is possible to define binary formats, which
can be much faster than textual formats such as \acronym{csv}.
Using a serialisation system like Avro also eases
interoperability of the database with
external systems and applications and it
significantly eases version management should data formats
change over time (what they always do). 
For more details on data loaders, please refer to chapters
\ref{chpt_sql} and \ref{chpt_loader}.
\comment{Avro is not yet available.}

Loaders, in general, are usually much more efficient than
the \sql\ \term{insert} statement.
The drawback of \term{insert} is that each statement
needs the whole cycle of \sql\ parsing and execution,
while loaders only need one cycle. Since a data source
can contain millions or even billions of rows,
loading is way more efficient than inserting in most cases.

The first two \acronym{csv}s in the script contain vertices.
As such they need to have a header and we need to instruct
the loader of how to interpret the data.
Since edges always have the same format, they don't need a header
and we do not give any instructions of how to interpret the data.

\section{First Queries}
The alternative to loading data is, of course, the conventional
\term{insert} statement:

\begin{sqlcode}
\begin{lstlisting}
insert into client(9000001, 'Popeye the Sailor');
insert into product(100001, 'Spinach, 450g net', 1.99);
\end{lstlisting}
\end{sqlcode}

These two statements insert a client and a product respectively.
We can also name the attributes explicitly, like:

\begin{sqlcode}
\begin{lstlisting}
insert into product(prod_key, prod_desc, prod_price)
              (100002, 'Candy Cigarettes, 20', 2.49);
\end{lstlisting}
\end{sqlcode}

Now we insert a bunch of edges:

\begin{sqlcode}
\begin{lstlisting}
insert into sales (edge, origin, destin, timestamp, weight, weight2)
           ('buys', 9000001, 100001, '1929-01-17T09:35:12', 1, 1.99)
insert into sales (edge, origin, destin, timestamp, weight, weight2)
           ('buys', 9000001, 100001, '1929-01-19T10:15:01', 2, 3.98)
insert into sales (edge, origin, destin, timestamp, weight, weight2)
           ('buys', 9000001, 100001, '1929-01-20T17:12:55', 3, 5.97)
insert into sales (edge, origin, destin, timestamp, weight, weight2)
           ('buys', 9000001, 100001, '1929-01-22T08:27:32', 1, 1.99)
insert into sales (edge, origin, destin, timestamp, weight, weight2)
           ('buys', 9000001, 100001, '1929-01-25T12:09:59', 1, 1.99)
insert into sales (edge, origin, destin, timestamp, weight, weight2)
           ('buys', 9000001, 100001, '1929-01-26T21:19:44', 2, 3.98)
insert into sales (edge, origin, destin, timestamp, weight, weight2)
           ('buys', 9000001, 100002, '1929-01-22T08:27:51', 1, 2.49)
\end{lstlisting}
\end{sqlcode}

\begin{minipage}{\textwidth}
Worth noticing here is the time format,
which follows \acronym{iso}-8601.

The format is
\begin{itemize}
\item 4 digits for the year and hyphen
\item 2 digits for the month and hyphen
\item 2 digits for the day of month
\item `T' to mark the beginning of the time section
\item 2 digits for the hour and colon
\item 2 digits for the minute and colon
\item 2 digits for the second and,
\item if finer grain is necessary,
a dot followed by up to 9 digits
for the nanoseconds.
\end{itemize}
\end{minipage}

This format can be used anywhere in \nowdb\ \sql.
However, it is also possible to define custom
date and time formats. How to do this is discussed
in chapter \ref{chpt_sql}. \comment{That's not yet possible.}

Worth noticing is also the first date.
It was on Jan, 17, 1929 that Popeye had his first
appearance in a newspaper of the \term{King Features}
Syndicate.

Now that we have inserted some data
into our database, we are able to perform $selects$, \eg:

\cmdline{
nowclient -d retail -Q
"select count(*) from sales $\backslash$ \\
\hspace*{4.7cm} where edge='buys' $\backslash$ \\
\hspace*{4.7cm} and origin=9000001"
}

which would give us 7 and would count Popeye's visits to the supermarket.
We can also count how often Spinach was bought:

\begin{sqlcode}
\begin{lstlisting}
select count(*) from sales
 where edge='buys'
   and destin=100001
\end{lstlisting}
\end{sqlcode}

which shows us 6.
Or we can ask how much Popeye bought and paid per type of product:

\begin{sqlcode}
\begin{lstlisting}
select edge, destin, count(*), sum(weight), sum(weight2)
  from sales 
 where edge='buys' 
   and origin=9000001 
 group by edge, destin
\end{lstlisting}
\end{sqlcode}

\begin{minipage}{\textwidth}
which would give us:
\begin{verbatim}
buys;100001;6;10;19.9000
buys;100002;1;1;2.4900
\end{verbatim}
\end{minipage}

Notice that the output
of the client tool does not resemble the classical
pretty-printed output produced by most database
client tools today. The advantage of such output is
that it is easier for humans to read.
The \acronym{csv}-like output shown above, however,
is better for interoperability, for instance,
when we want to combine it through pipes
with other programs, like this:

\cmdline{
nowclient -d retail -Q "select * from sales" | cut -d";" -f2 | $\dots$
}

On the other hand, there are tools that
produce more readable output from \acronym{csv} input,
such as \tech{csvlook} from the \tech{csvkit} package.\footnote{Have
a look at
\url{https://github.com/jeroenjanssens/data-science-at-the-command-line}}
To obtain a traditional pretty-printed output we could do the following
(assuming that \tech{query.sql} contains the query we used above):

\cmdline{
cat query.sql | nowclient | csvformat -d";" | $\backslash$ \\
    header -a 'edge,product,count,quantity,price' | csvlook
}

and would obtain for the grouping query used above:

\begin{minipage}{\textwidth}
\begin{verbatim}
|----------+---------+-------+----------+---------|
|  edge    | product | count | quantity | price   |
|----------+---------+-------+----------+---------|
|  buys    | 100001  | 6     | 10       | 19.9000 |
|  buys    | 100002  | 1     | 1        | 2.4900  |
|----------+---------+-------+----------+---------|
\end{verbatim}
\end{minipage}

Here is a more typical time series query illustrating
the advantage of the pretty printer:

\begin{sqlcode}
\begin{lstlisting}
select destin, timestamp, weight, weight2
  from sales 
 where edge='buys' 
   and origin=9000001 
 order by timestamp
\end{lstlisting}
\end{sqlcode}

\begin{minipage}{\textwidth}
which, with the same technique as above, shows:
\begin{verbatim}
|----------+---------------------+----------+---------|
|  product | timestamp           | quantity | price   |
|----------+---------------------+----------+---------|
|  100001  | 1929-01-17T09:35:12 | 1        | 1.9900  |
|  100001  | 1929-01-19T10:15:01 | 2        | 3.9800  |
|  100001  | 1929-01-20T17:12:55 | 3        | 5.9700  |
|  100001  | 1929-01-22T08:27:32 | 1        | 1.9900  |
|  100002  | 1929-01-22T08:27:51 | 1        | 2.4900  |
|  100001  | 1929-01-25T12:09:59 | 1        | 1.9900  |
|  100001  | 1929-01-26T21:19:44 | 2        | 3.9800  |
|----------+---------------------+----------+---------|
\end{verbatim}
\end{minipage}

We can also select from vertices,
but instead of a table, we use the type:

\begin{sqlcode}
\begin{lstlisting}
select prod_price from product
 where prod_key = 100001;
\end{lstlisting}
\end{sqlcode}

which shows
\begin{verbatim}
1.99
\end{verbatim}

and

\begin{sqlcode}
\begin{lstlisting}
select client_name from client
 where client_key = 9000001;
\end{lstlisting}
\end{sqlcode}

which gives
\begin{verbatim}
Popeye the Sailor
\end{verbatim}

\comment{
This is as it should be.
Unfortunately, there are still heavy inconsistencies
with vertices.
The syntax is currently `select field from vertex as type',
which is ugly and pointless
and will be corrected \acronym{asap}.
}

Much more typical for \nowdb, however,
is to use vertices together with edges. Edges connect vertices
and can therefore be seen as the `relations' in \nowdb.
What we typically want is either find the vertex
at the other end of the edge (\eg\ find the destination
for a given origin) or to look at edges with the attributes
of the vertices added to them.

Both patterns are, in \sql, instances of \term{joins}.
An instance of the first pattern would be:

\begin{sqlcode}
\begin{lstlisting}
select timestamp, prod_desc, prod_price
  from sales join product on destin
 where edge = 'buys'
   and origin = 9000001;
\end{lstlisting}
\end{sqlcode}

\begin{minipage}{\textwidth}
\begin{verbatim}
|---------------------+----------------------+---------|
| timestamp           | product              | price   |
|---------------------+----------------------+---------|
| 1929-01-17T09:35:12 | Spinach, 450g net    | 1.9900  |
| 1929-01-19T10:15:01 | Spinach, 450g net    | 1.9900  |
| 1929-01-20T17:12:55 | Spinach, 450g net    | 1.9900  |
| 1929-01-22T08:27:32 | Spinach, 450g net    | 1.9900  |
| 1929-01-25T12:09:59 | Spinach, 450g net    | 1.9900  |
| 1929-01-26T21:19:44 | Spinach, 450g net    | 1.9900  |
| 1929-01-22T08:27:51 | Candy Cigarettes, 20 | 2.4900  |
|---------------------+----------------------+---------|
\end{verbatim}
\end{minipage}

Note the difference in the price column.
In the previous query we used \term{weight2} of sales,
which (as you may have realised)
is the multiplication of the product price and the value
in \term{weight}, which, in its turn,
represents the number of items.
Here, however, we use the value in \term{prod\_price}
which is the base price of one unit of that product.

We can, of course, combine joins with grouping:

\begin{sqlcode}
\begin{lstlisting}
select edge, destin, count(*), sum(prod_price)
  from sales join product on destin
 where edge = 'buys'
   and origin = 9000001
 group by edge, destin;
\end{lstlisting}
\end{sqlcode}

\begin{minipage}{\textwidth}
\begin{verbatim}
|----------+---------+-------+---------|
|  edge    | product | count | price   |
|----------+---------+-------+---------|
|  buys    | 100001  | 6     | 11.9400 |
|  buys    | 100002  | 1     | 2.4900  |
|----------+---------+-------+---------|
\end{verbatim}
\end{minipage}

Note again the sum of the price which, here,
is just the sum of the base price per unit of the product.

The point about the first joining pattern
is that it joins only one of the two vertices
with the edge. The second pattern is a bit more complex:

\begin{sqlcode}
\begin{lstlisting}
select timestamp, prod_desc, client_name
  from sales
  join product on destin
  join client on origin
 where edge = 'buys'
   and origin = 9000001;
\end{lstlisting}
\end{sqlcode}

The result of this query would be:

\begin{minipage}{\textwidth}
\begin{verbatim}
|---------------------+----------------------+-------------------|
| timestamp           | product              | client            |
|---------------------+----------------------+-------------------|
| 1929-01-17T09:35:12 | Spinach, 450g net    | Popeye the Sailor |
| 1929-01-19T10:15:01 | Spinach, 450g net    | Popeye the Sailor |
| 1929-01-20T17:12:55 | Spinach, 450g net    | Popeye the Sailor |
| 1929-01-22T08:27:32 | Spinach, 450g net    | Popeye the Sailor |
| 1929-01-25T12:09:59 | Spinach, 450g net    | Popeye the Sailor |
| 1929-01-26T21:19:44 | Spinach, 450g net    | Popeye the Sailor |
| 1929-01-22T08:27:51 | Candy Cigarettes, 20 | Popeye the Sailor |
|---------------------+----------------------+-------------------|
\end{verbatim}
\end{minipage}

\comment{
Unfortunately, joins are not yet available :-(
}

An edge field that we have not yet used is the \term{label}.
The \term{label} is intended to create a connection
between edges that are inherently related. In our example, we see
that at one day Popeye bought two different things:
spinach and candy cigarettes.
That was on Jan, 22.

\begin{minipage}{\textwidth}
These two edges, hence, relate to the same visit at the supermarket.
We could link these edges using a label. The edge would then be created
as, for instance:
\begin{sqlcode}
\begin{lstlisting}
create edge buys (
        origin client,
        dest product,
        label text,
        weight uint,
        weight2 float
);
\end{lstlisting}
\end{sqlcode}
\end{minipage}

The label text would correspond to a token generated by
the cashpoint when a customer starts the check-out
and all products scanned during the procedure
would have the same token.
We could then insert edges using that token as label:

\begin{sqlcode}
\begin{lstlisting}
insert into sales (edge, origin, destin, timestamp,
                            label, weight, weight2)
   ('buys', 9000001, 100001, '1929-01-22T08:27:32',
                      'tx-19290122082731', 1, 1.99)
insert into sales (edge, origin, destin, timestamp,
                            label, weight, weight2)
   ('buys', 9000001, 100002, '1929-01-22T08:27:51', 
                      'tx-19290122082731', 1, 2.49)
\end{lstlisting}
\end{sqlcode}

\begin{minipage}{\textwidth}
We now can select edges according to the label:

\begin{sqlcode}
\begin{lstlisting}
select destin, timestamp, weight, weight2
  from sales 
 where label = 'tx-19290122082731'
 order by timestamp
\end{lstlisting}
\end{sqlcode}
\end{minipage}

\begin{minipage}{\textwidth}
This query would result in:

\begin{verbatim}
|----------+---------------------+----------+---------|
|  product | timestamp           | quantity | price   |
|----------+---------------------+----------+---------|
|  100001  | 1929-01-22T08:27:32 | 1        | 1.9900  |
|  100002  | 1929-01-22T08:27:51 | 1        | 2.4900  |
|----------+---------------------+----------+---------|
\end{verbatim}
\end{minipage}

\section{The Python Client}
Until here we always used the client \emph{tool}
to perform queries.
That is certainly an important use case.
Much more typical, however, is to develop application code
that needs a client \acronym{api} to connect to the database.

\nowdb\ comes with a native client \acronym{api}
that is available in different languages, among others
C, \CC, Go, Python and Lua.
\comment{Only Python is currently available.
There is a low-level \acronym{api} for C,
but that is not for developing applications,
but for developing \acronym{api}s.}
Here, we will have a quick look at the Python \acronym{api}.

The module implementing the Python \nowdb\ \acronym{api}
is called \term{now.py} and must be imported into the client program.
For the Python interpreter to find this module,
it must be in a directory in the \acronym{pythonpath}.
There may be different ideas on how to install python modules.
The \nowdb\ installation will copy all \nowdb-related modules
to one specific folder and add this folder to the
\acronym{pythonpath}. But you also may install
the \nowdb\ Python \acronym{api} using \tech{pip}.
Then, everything is handled by the Python environment
and you don't need to care about these things.
For more details on installation, please refer
to chapter \ref{chpt_install}.

Anyway, here is a simple Python program:

\begin{python}
\begin{lstlisting}
import now

with now.Connection("localhost", "55505", None, None) as c:
   with c.execute("use retail") as r:
       if not r.ok():
          print "cannot use retail: %s" % r.details()
          exit(1)

   with c.execute("select count(*), sum(weight), avg(weight) \
                     from sales where edge='buys'") as cur:
       if not cur.ok():
          print "ERROR: %s" % cur.details()
          exit(1)
       for row in cur:
           print "count: %d, sum: %d, avg: %.2f" %
                 (row.field(0), row.field(1), row.field(2))
\end{lstlisting}
\end{python}

The program first creates a \term{Connection}
to a database listening on port 55505 on `localhost'.
It then executes a \term{use} statement on this connection
to indicate the database towards which
the following statements are directed.

The result of a \term{use} statement is a \term{status}.
A status is either \acronym{ok} or an error
whose details can be obtained by means of the method
\term{details()}, which returns a string.
Results also support the method \term{code()},
which would return a numeric error code.
It is often useful to know the error code to decide
what to do programmatically (abandon the program,
retry, try something else, \etc)

Results are \term{resource managers}. That means
that they can be used inside a \term{with} statement.
\term{with} assures that all resources (in this case
the result) are freed before the control leaves
the \term{with}-block even if an exception is raised.

The program then executes a query.
This time the execution returns a \term{cursor} (`cur').
The program checks whether the cursor is
in a good state. The statement may have
failed on the server side. The cursor would
then be in a state that is not \acronym{ok}.
In that case, the program prints the error details
and exits with return code 1.

Otherwise, if the result was fine,
it iterates over the cursor
printing for each row the fields 1-3.
Cursors, indeed, are iterators
that allow simple iteration using \term{for}.

\begin{minipage}{\textwidth}
Here is an example of an \term{insert} statement
(we assume that the connection, `c', was already established):

\begin{python}
\begin{lstlisting}
with c.execute("insert into edge (edge, origin, product, timestamp) \
                                 ('complains', 9000001, 100002, \
                                  '1929-01-23T08:45:00')") as rep:
       if not rep.ok():
          print "ERROR: %s" % cur.details()
          exit(1)
       else:
          print "rows affected: %d" % rep.affected()
          print "running time : %d" % rep.runTime()
\end{lstlisting}
\end{python}
\end{minipage}

In the case of data manipulation (\acronym{dml})
or data loading (\acronym{dll}) statements,
the result (if there was not error) is a report
A report has the methods \term{affected()},
which indicates the number of rows affected by this statement,
\term{runTime()},
which indicates the running time of the statement in microseconds,
and \term{errors()},
which, in the case of \term{load},
indicates the number of rows that resulted in an error.

For more details on the Python client,
please refer to chapter \ref{chpt_pythonclient}.
The other client \acronym{api}s are described
in chapters \ref{chpt_ccpp}, \ref{chpt_goclient} and
\ref{chpt_luaclient}.

\section{Python in the Database}
Like many other databases,
\nowdb\ supports
stored procedures and stored functions,
\ie\ code that is executed within the database.
This code can be written in several languages.
Currently Python and Lua are supported.
\comment{Lua is not yet available.}

The major advantage of Python is the huge
number of libraries available,
in particular for mathematics,
data science and machine learning.
But the poor design of the Python interpreter
imposes some limitations in concurrency.
(Please refer to chapter \ref{chpt_pythonemb}
for details.)
It is therefore recommended to use Python
in the database where it is strongest,
\ie\ for higher math and data science.
For `normal' \acronym{dba}
jobs or simple computations that do not require
sophisticated math libraries, the use one of
the light-weight languages,
such as Lua, is recommended.

The difference between stored procedures
and stored functions is that stored functions
run inside an \sql\ context. They
can be used in a \term{select} clause,
for instance. They are not allowed, however,
to execute whatever they want.
In a \term{select} clause,
they are not allowed to
execute \term{updates} or \term{inserts} for instance.
It would indeed be very strange when
a \term{select} would change the database.

Stored procedures, on the other hand,
cannot run in \sql\ context. They are
executed explicitly by the \term{exec}
statement. In exchange for this limitation,
they get a lot of power:
stored procedures are allowed to run
any \sql\ code they (or their programmers)
like to including \acronym{dml}, \acronym{dll}
and even \acronym{ddl}.

Stored procedures are composed of two
elements: there interface (or \term{signature})
and their implementation.
The interface defines how the procedure is
to be called; the implementation defines
what it does.

The interface is create by the \sql\ statement
\term{create procedure}, \eg:

\begin{sqlcode}
\begin{lstlisting}
create procedure sales.revenue(pClient uint, pDay time) language python
\end{lstlisting}
\end{sqlcode}

This statement defines the interface of a procedure
called `revenue' that takes two arguments:
\begin{itemize}
\item $pClient$ which is an unsigned integer and
\item $pDay$ which is a timestamp.
\end{itemize}
Furthermore, the procedure is written in Python and
it lives in the module `sales'.
The statement does not define a return value for the procedure.
But all procedures in \nowdb\ return a \term{result type},
that is polymorphic type that
can be either a status, a report, a row or a cursor.

The module `sales' must be located in a directory that is
known to the Python interpreter. Typically, it will
be in a directory in the \acronym{pythonpath}.
Note that the \acronym{pythonpath} must be set
in the environment of the server before it is started.
This is actually a security feature:
it is not possible to smuggle executable code
into the database. To get code to run in the database
context one needs control over the filesystem.
The database does not introduce a new attack vector.

Once the function $revenue$ is created
and its code is accessible to the database, it
can be executed by an \term{exec} statement:

\begin{sqlcode}
\begin{lstlisting}
exec revenue(9000001, '1929-01-25')
\end{lstlisting}
\end{sqlcode}

\begin{minipage}{\textwidth}
Let's look into the module \term{sales}
to see how the procedure is implemented:

\begin{python}
\begin{lstlisting}
import nowdb
import datetime

def revenue(pClient, pDay):
  try:

    today = nowdb.now2dt(pDay)
    tom = today + timedelta(days=1)

    stmt = "select sum(weight2) from sales "
    stmt += "where edge = 'buys'"
    stmt += "  and origin =" + str(pClient)
    stmt += "  and timestamp >= " + today.strftime(nowdb.TIMEFORMAT)
    stmt += "  and timestamp <  " + tom.strftime(nowdb.TIMEFORMAT)

    with nowdb.execute(stmt) as c:

      if not c.ok():
        return nowdb.makeError(c.code(), c.details()).toDB()

      r = makeRow()
      for row in c: # there is only one row
         r.add2row(INT, row.field(0))

      return r.toDB()

  except Exception as x:
    return nowdb.makeError(USRERR, str(x)).toDB()

def cleanup():
  pass

\end{lstlisting}
\end{python}
\end{minipage}

First, we import the module \term{nowdb}.
This is the equivalent to the \term{now} module
that we imported on client side.
Note, however, that on server side
it is necessary to import the module in this format,
\ie: \term{import nowdb}, not: \term{from nowdb import $\dots$}
Otherwise, \nowdb\ would not be able to initialise the module.
In that case, the database would issue an error 
and not execute the procedure.

The module is initialised, by the way, when the session
terminates (or when the module is first loaded into a session).
This means
that each session, \ie\ each connection to the database,
has its own view of the module. This allows for
global variables, which will have the same lifetime
as the session itself.

Next step in the module is
the definition of the function \term{revenue} with two parameters.
The database will pass the parameters according to the type
information in the interface: the first parameter
will be passed as unsigned integer and the second
will be passed as timestamp.

The entire code of the procedure is encapsulated
in a \term{try/execpt}-block. This, indeed, is
good practice, since otherwise
an exception would terminate the procedure without
passing a result back to the database and without
detailed error information.

In the \term{except}-part, we create
an error with error code `user error' and
the exception string as detailed information.
Notice that we call the $toDB()$ method
and that we in fact return the result of that
method, not the result itself.
The purpose of the method is to separate
the C part and the Python part of the result.
The C part goes back to the database,
the Python part stays in the Python world
to be collected by the Python garbage collector.
Returning a result without calling its $toDB()$
method would pass Python stuff
to the database which would not know how to handle
that.

The main logic of the function
is in the \term{try}-part.
We start by converting the \nowdb-timestamp
into the Python \term{datetime} object \term{today}
using the \term{now2dt} function
(which, by the way,
is also available on client side).
By adding one day to \term{today} we get
\term{tom}(orrow).

With these values and the \term{pClient}
variable, we construct an \sql\ statement,
which is then passed to the \term{execute}
function. Notice that on server-side,
\term{execute} is not a method of another
object or class like \term{connection}.
Indeed, we already are inside the database.
We don't need a connection to talk to it.

Since the \sql\ statement
is a \term{select}, the result we get
back is a cursor. We check that the cursor
is \acronym{ok} and, if so, we work on
its rows using a \term{for} loop.
In fact, we have only
one row -- since we issued a \term{count}
without a \term{group by} clause.

From the cursor row, we create a new row
using the \term{makeRow} function
and add the first (and only) field 
of the cursor's row to it as integer. This row
is returned to the database (using,
of course, its $toDB()$-method).

Note that we could have returned the cursor itself --
since a cursor is also a result type and, as such,
a valid object to be returned by a stored procedure.
We return the row only for the purpose of illustrating
how rows can be created in Python.
\comment{The whole example is quite trivial to be honest.}

The database will then send the result,
that is the row we created,
back to the caller (the \term{exec} statement),
which, in its turn, can handle this result.
When we call the \term{exec} statement
from \term{nowclient} like this:

\cmdline{
nowclient -d retail -Q "exec revenue(9000001, '1929-01-25')"
}

the result would look like this:
\begin{verbatim}
1.9900
\end{verbatim}

\begin{minipage}{\textwidth}
Of course, we can call \term{exec}
also from a Python client and then handle
the result as row, \eg:

\begin{python}
\begin{lstlisting}
with Connection("127.0.0.1", "55505", None, None) as c:
  with c.execute("exec revenue(9000001, '1929-01-25')") as row:
    if not row.ok():
       print "ERROR: %s" % row.details()
       exit(1)
    print "revenue from client 9000001: %d" % row.field(0)
\end{lstlisting}
\end{python}
\end{minipage}

But we have to come back to the server side once again.
There is still a detail that we have not discussed,
namely the $cleanup()$ function at the end of the module.
Modules may (but do not need to) contain this
function. If it is present in the module, it is automatically
called by the database when the session terminates.
For simple code like the one we used
here for illustration, it is not necessary to have a cleanup
function. 

A cleanup is necessary, whenever result types
(cursors, rows, \etc) are stored in global variables.
It is completely legal to store data in global variables.
When the session terminates, the database will reload
the module, so that the next session will start with
fresh instances of those variables. However, when
global variables are holding result types, the underlying
C part of these variables must be released.
Since the database has no knowledge on what is stored
in global variables, the user needs to provide code
to release this memory.

One interesting use case is \term{functional cursors}.
Functional cursors behave like normal cursors,
but they can use the power of the embedded language
to enrich results
using functionality
not available in \sql.
This is discussed in chapter \ref{chpt_pythonemb}.

\section{What's Next?}
This chapter was only a brief introduction
to some important features and the general flavour 
of \nowdb. The remainder of this manual will
discuss the main features more deeply.

The next chapter discusses the \nowdb\ \sql\ dialect.

Chapters \ref{chpt_nowdbd} and \ref{chpt_clienttool}
present the command line tools \tech{nowdbd}
and \tech{nowclient} respectively.

Chapters \ref{chpt_ccpp} -- \ref{chpt_luaclient}
discuss the available native clients in different
languages and \ref{chpt_llc} presents the low-level
C \acronym{api} mainly used to implement
native clients.

The next two chapters, \ref{chpt_pythonemb} and
\ref{chpt_luaemb}, present the server-side
language bindings at more depth.

Chapter \ref{chpt_install} provides detailed information
on installation of server and client on different
platforms.

The next chapters present more features,
namely the loader (\ref{chpt_loader}) and
server-side techniques such as publish and subscribe and
filters (\ref{chpt_pubsub}).

The following chapters 
\ref{chpt_opt} and \ref{chpt_sizing}
discuss technical insight
for application designers and \acronym{dba}s.

Finally, the appendix \ref{chpt_errors} lists server-side
error codes.


