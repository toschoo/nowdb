\section{Outline}
This chapter presents the \term{simple client},
\term{nowclient},
which comes with basic \nowdb\ installation
and is available in the \nowdb\ docker.
The advantage of this client is that it is
extremely simple, it has a very small code base,
it is fast and very easy to use.
On the other hand, it does not implement
advanced features like an interactive user interface
or pretty printing query results.
There is a more complete client implemented
in Go \comment{(or Rust?)} that can be downloaded
\comment{here}.

The simple client is a program to send a
single statement or a sequence of statements
to the \nowdb\ server. The result,
in case of \acronym{dql} statements,
is presented in a \acronym{csv}-like format
on \term{stdout}
(with semicolon instead of comma).

For a single \acronym{sql} statement,
the program provides the command line parameter
\tech{-Q} (see next section for details).
Without this option, \term{nowclient}
expects the statement(s) to be sent to the server
on \term{stdin}. The program could, hence,
be called in the form:

\cmdline{
"use retail; select count(*) from buys;" | nowclient
}

This line would send two statements to the server,
a \term{use} statement and a \term{select}.
Note that both statements are terminated with
a semicolon. This is mandatory, when sending
queries through \term{stdin}.

In this example, no command line options are used.
The program will therefore fall back to
its default behaviour.
The following section discusses
the command line options.

\clearpage
\section{Command Line Parameters}
Options known to \term{nowclient} are:

\begin{itemize}
\item \tech{-s}:
The server address or name, \eg\ \tech{myserver.mydomain.org} or
\tech{127.0.0.1}. Default is \tech{127.0.0.1};

\item \tech{-p}:
the port to which the database is listening. Default: 55505;

\item \tech{-d}:
the database to which we want to connect. 
Default: no database at all, which means
that we cannot send queries without naming a database.
Below we will look at alternatives to using this parameter;

\item \tech{-Q}:
the query we want the database to process;
the query is added as a string, \eg:

\cmdline{
nowclient -d retail -Q "select count(*) from buys"
}

It is not necessary to end the statement with semicolon.
This is only necessary when statements are sent through
\term{stdin};

\item \tech{-t}:
print some (server-side) timing information to standard error;

\item \tech{-q}:
quiet mode: don't print processing information to standard error;

\item \tech{-V}: prints version information to standard output;
\item \tech{-?} or \tech{-h}: prints a help message to standard error.
\end{itemize}

