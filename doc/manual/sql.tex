\section{Types}
\subsection{Static Types}
The static types constitute the \nowdb\ \sql\ type system.
The static types can be used in \sql\ statements.
The declaration form is used in \acronym{ddl} statements
to define types, edges, procedure and functions.
In \acronym{dml}, \acronym{dll} and \acronym{dql} statements,
instances of the types are used, \ie\
types are not explicitly declared, but used implicitly
by means of their constructors.

\begin{minipage}{\textwidth}
\textbf{Integer}\\
Declaration: int, integer \\
Values: $-2^{63} \dots 2^{63}-1$ \\
Constructors: $\pm n$, where $n$ is an unsigned integer.\\
Examples: $-1, +0, +1$
\end{minipage}

\begin{minipage}{\textwidth}
\textbf{Unsigned Integer} \\
Declaration: uint, uinteger, unsigend integer \\
Values: $0 \dots 2^{64}-1$  \\
Constructors: any sequence of digits ($0\dots9$). \\
Examples: $0, 1, 2$ 
\end{minipage}

\begin{minipage}{\textwidth}
\textbf{Float} \\
Declaration: float \\
Represents a binary64 \acronym{ieee}-754 floating point number.
For possible values, please refer to the standard or to the table in
\url{https://en.wikipedia.org/wiki/IEEE\_754#Basic\_and\_interchange\_formats}.\\
Constructors: any integer followed by a dot and a sequence of digits,
              optionally followed by $e$ followed by an integer.
              \comment{The exponential form is not yet available.} \\
Examples: $-1.0, 0.0, 1.0, 3.14159, 1.797693e308$ 
\end{minipage}

\begin{minipage}{\textwidth}
\textbf{Time} \\
Declaration: time, date \\
Values:  \acronym{utc} 1677-09-21T00:12:44 --
         \acronym{utc} 2262-04-11T23:47:16 \\
Precision: nanosecond. \\
Note, however, that range and precision depend on server configuration.
With less precision, a higher range can be reached.
Please refer to the database configuration guide. \\
Timezone: \acronym{utc} \\
Constructor: any integer or any string following \acronym{iso}-8601 
or any string following a locally defined time format. \\
Examples:\\
1535284617906179393, \\
'1940-12-21', \\
'1904-06-16T11:43:10', \\
'2011-11-11T11:11:11.123456789'
\end{minipage}

\begin{minipage}{\textwidth}
\textbf{Bool} \\
Declaration: bool \\
Values: true, false
\end{minipage}

\begin{minipage}{\textwidth}
\textbf{Text} \\
Declaration: text \\
Values: \acronym{utf}-8 string with up to 255 bytes\\
Constructor: string enclosed by ' \\
Examples:\\
'',\\
'hello world',\\
\begin{CJK}{UTF8}{gbsn}
'鎮州臨濟慧照禪師語錄序。'
\end{CJK}
\end{minipage}

\begin{minipage}{\textwidth}
\textbf{Longtext} \\
Declaration: text \\
Values: \acronym{utf}-8 string with up to 4096 bytes\\
\end{minipage}
\begin{minipage}{\textwidth}
\textbf{Blob} \\
\end{minipage}

\subsection{Dynamic Types}
Dynamic types are not used in \sql\ statements,
but rather describe the return values of \sql\
statements. As such, they live in the context
of a host language (C, Python, Lua, \etc).
There concrete implementation, therefore,
depends on that language and any language
functioning as either client or server-side language
needs to implement these types.
For more information on concrete implementation
of dynamic types, please refer to the
host language \acronym{api} specifications.

\begin{minipage}{\textwidth}
\textbf{Status}\\
Values: \acronym{ok}, \acronym{nok}\\
In the case of error (\acronym{nok}), \nowdb\ provides:
\begin{itemize}
\item an error code
\item a detailed error message
\end{itemize}
Error codes together with a brief description
of their meaning can be found in \ref{chpt_errors}.
\end{minipage}

\begin{minipage}{\textwidth}
\textbf{Report}\\
Reports consist of up to three values:
\begin{itemize}
\item number of affected rows
(returned by all \acronym{dml} and \acronym{dll} statements)
\item number of errors
(returned only by \acronym{dll} statements)
\item running time
(returned by most \acronym{dml} and \acronym{dll} statements)
\end{itemize}
\end{minipage}

\begin{minipage}{\textwidth}
\textbf{Row}\\
A row is the result of a projection;
it consists of an array of values
with type information called fields.
A row result consists of one or many rows.
\end{minipage}

\begin{minipage}{\textwidth}
\textbf{Cursor}\\
A cursor is an iterable collection of rows.
The iteration directive is \term{fetch}.
Each fetch may return one or many rows.
A cursor is a server-side resource
and shall be closed using the directive \term{close}.
\end{minipage}

\section{Data Definition}
\subsection{Schema}
\subsection{Table}
\subsection{Type}
\subsection{Edge}
\subsection{Index}
\subsection{Procedure}
\subsection{Function}
\subsection{Period}

\section{Data Manipulation}
\subsection{Insert}
\subsection{Update}
\subsection{Delete}

\section{Data Loading}

\section{Data Query}

\section{Miscellaneous}
\subsection{Use}
\subsection{Exec}
