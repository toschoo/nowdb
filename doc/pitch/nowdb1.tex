\documentclass{beamer}
%\usetheme{Bergen}
%\usetheme{Copenhagen}
% \usetheme{Darmstadt}
% \usetheme{Frankfurt}
%\usetheme{Luebeck}
\usetheme{Madrid}
\usecolortheme{beetle}
%\usecolortheme{dove}
%\usecolortheme{fly}
%\usecolortheme{seagull}
%\useinnertheme{rectangles}
%\useinnertheme{circles}
%\useinnertheme{inmargin}
%\useinnertheme{rounded}
%% =======================================================
%% (c) Tobias Schoofs
%% =======================================================
%% Commands 4 Programmers
%% =======================================================

%include lhs2TeX.fmt
%include lhs2TeX.sty

\usepackage[pdftex]{graphicx}
\usepackage{ucs}
\usepackage[utf8x]{inputenc} 
\usepackage{tabto}
\usepackage[russian,portuguese,german,english]{babel}
\usepackage{CJK}
\usepackage{amsfonts}
\usepackage{amsfonts}

\usepackage{amsmath}
\usepackage{amssymb}
\usepackage{amsthm}
\usepackage{amscd}

\usepackage{siunitx}

\usepackage{listings}
\usepackage{longtable}

\usepackage{tikz}
\usepackage{pgfplots}

\usepackage{relsize}
\usepackage{xcolor}

\long\def\ignore#1{}

\newcommand{\acronym}[1]{\textsc{#1}}

\newcommand{\term}[1]{\textit{#1}}
\newcommand{\tech}[1]{{\ttfamily #1}}
\newcommand{\latin}[1]{\textit{#1}}
\newcommand{\speech}[1]{\textit{#1}}

\newcommand{\ie}{\textit{i.e.}}
\newcommand{\eg}{\textit{e.g.}}
\newcommand{\etc}{\textit{etc.}}
\newcommand{\viz}{\textit{viz.}}
\newcommand{\vs}{\textit{vs.}}

\newcommand{\sql}{\acronym{sql}}

\newcommand{\code}[1]{{\ttfamily #1}}
\newcommand{\cmdline}[1]{{\ttfamily #1}}

\newenvironment{sqlcode}{
\small
\begin{minipage}{\textwidth}
\lstset{language=sql,
        keepspaces=true,
        showspaces=false,
        showstringspaces=false}
}{
\end{minipage}
}

\newenvironment{python}{
\small
\begin{minipage}{\textwidth}
\lstset{language=python,
        keepspaces=true,
        showspaces=false,
        showstringspaces=false}
}{
\end{minipage}
}

\newcommand{\keyword}[1]{\textbf{#1}}
\newcommand{\identifier}[1]{\textit{#1}}

\newcommand{\Rom}[1]{\uppercase\expandafter{\romannumeral #1\relax}}

\newcommand{\CC}{C\nolinebreak[4]\hspace{-.05em}\raisebox{.3ex}{\relsize{-2}{\textbf{++}}}}
\newcommand\csharp{C\nolinebreak[4]\hspace{-.02em}\raisebox{.3ex}{\relsize{-1}{\#}}}

\newcommand{\comment}[1]{\textcolor{red}{#1}}

\newcommand{\nowdb}{\textsc{n}o\textsc{wdb}}

\newcommand{\connect}[2]{
  \draw [-,color=black] (#1) to (#2)
}

% tikz
% \newcommand{\drawDataPoint}{\draw circle (0.2)}
% \drawDataGroup

% \renewcommand{\gcd}{\textsc{gcd}}

\title{What is \nowdb?}
\author{Tobias Schoofs}
\begin{document}

% TITLE
\frame{\titlepage}

% OVERVIEW 
\begin{frame}
\frametitle{Overview}
\begin{itemize}
\item What?
\item Why?
\item How?
\item Vision!
\item Where are we now?
\item Market?
\item Why the funny name?
\end{itemize}
\end{frame}

% WHAT
\begin{frame}
\frametitle{What?}
\begin{equation}
\nowdb = (graph +  timeseries + \acronym{sql}) \times (Python + Lua)
\end{equation}
\end{frame}

% WHY TIMESERIES
\begin{frame}[shrink]
\frametitle{Why Timeseries?}
\begin{itemize}
\item Always more data are timeseries \emph{to some extent}\\
      important driver: \acronym{i}o\acronym{t}\\
      but also: ``traditional'' industries:
      \begin{itemize} 
      \item energy,
      \item manufacturing
      \item retail
      \item \etc
      \end{itemize}
\item Avalailable timeseries databases are too narrow,\\
      \ie\ \emph{pure} timeseries data.\\
      There are projects trying to solve this issue,
      \eg:
      $Timescale = Postgres + Timeseries$ \\
      $\dots$ but it's still a bit slow

\item Timeseries data have special requirements:
      \begin{itemize}
      \item Fast ingestion ($> 1M/s$)
      \item Scalability of queries ($> 1B metrics$)
      \item Special handling of time dimension
      \item Timeseries are less rigid
            (time points may be lost or duplicated)
      \end{itemize}
\end{itemize}
\end{frame}
\begin{frame}

% WHY GRAPH
\frametitle{Why Graph?}
\begin{itemize}
\item The main point is:\\
      Make timeseries \st{great again} more widely applicable!
\item Natural candidates: Relational and Graph.
\item Graph is more flexible:\\
      one can add new edges without changing the structure of entities
\item Graph is less rigid:\\
      edges pointing to nowhere or\\
            pointing to duplicated data are no issue
\item Graph may be easier to grasp:\\
      and this way reduce common errors in data modelling.\\
      (This, however, is pure speculation on my side $\dots$)
\end{itemize}
\end{frame}

% WHY SQL
\begin{frame}
\frametitle{Why \acronym{sql}?}
\begin{itemize}
\item \acronym{sql} is standard.
\item \acronym{sql} is actively developed with new ideas flowing in.
\item \acronym{sql} is known by virtually everybody in the industry.
\item \acronym{sql} is easy to learn (even though tricky to master \dots)
\item \acronym{sql} is designed for integration with other languages and tools.
\item Thousands of tools support \acronym{sql}.
\item \acronym{sql} is not bad anyway \dots
\item Well \dots
\item Next slide!
\end{itemize}
\end{frame}

% Why Python?
\begin{frame}[shrink]
\frametitle{Why Python?}
\includegraphics[width=0.3\linewidth]{pythonlogo.png}
\begin{itemize}
\item An important application domain for \nowdb\ is data science.
\item There will be strong demand for tools, libraries, languages supporting
\begin{itemize}
\item Calculus
\item Numerical analysis
\item Function analysis and function plotting
\item Linear Algebra (vectors, matrices, \etc)
\item Graph(!)-oriented data structures
\item Complex Analysis
\item \dots
\end{itemize}
\item Millions of lines of source code implementing this kind of stuff
      are available in (or through) Python
\item Python is \textbf{the} data science language.\\
      (There is also $R$, but $R$ has serious drawbacks
       as general purpose language and for embedding in another system.)
\item It is in particular interesting to have Python available
      on server side.
      This will reduce client/server round trips
      and network traffic.
\end{itemize}
\end{frame}

% Why Lua?
\begin{frame}[shrink]
\frametitle{Why Lua?}
\begin{itemize}
\item Unfortunately, nothing is perfect and also
      Python has its drawbacks.
\item The main issue (from our perspective) is
      the design of the Python interpreter:
      it does not allow running many interpreters ($=$sessions)
      in parallel in one process ($=$\nowdb\ server).
\item In consequence:
      Python is not good for tasks that frequently run
      in parallel.
\item We therefore need a second language for more
      down-to-earth, day-to-day \acronym{dba} jobs.
\item Lua appears to be a good candidate because:
\begin{itemize}
\item Lua is well known in the data science community
\item Lua already had its appearance on the database stage
      (namely in the redis \term{in-memory} database.
\item Lua is designed as configuration language
      (\acronym{dba} jobs typically involve configuration work)
\item Lua is designed for integration with the C language.
\end{itemize}
\end{itemize}
\begin{flushright}
\includegraphics[width=0.16\linewidth]{lualogo.png}
\end{flushright}
\end{frame}

% HOW
\ignore{
datamodel example with
simple edges:
- product of category or
- client has address
timeseries edges:
- client buys product
on joins
select client.buys.destin
}
\ignore{
2 storage types
1) tables for edges
   optimised for fast ingestion
   data packed to favour typical queries
   special handling of time
2) vertices
   store vertex types
   store consistent data (no duplicates)
   optimised for large datatypes (many attributes)
   with potentially many NULL values
   "column-oriented"
3 types of indices:
1) for primary keys
   - b-trees to store limited amount of data per key
   - used mainly for vertices (store vids and page numbers)
2) page indices ("secondary indices")
   - store page numbers per key in a second-level b-tree
   - for mass data per key (i.e. timeseries)
3) labels
   - store a known number of values per key
   - used to quickly find related edges
}
\ignore{
Python in the database...
}
% VISION
\ignore{
- data management platform scalable
  to billions of metrics on one server
- easy to handle for people with db experience
- server- and client-side modules for data analysis
- integrated with big-data tools for reliable ingestion
  (e.g. Kafka)
- integration with GUIs, e.g. Grafana
}
% WHERE ARE WE NOW 
% MARKET
% NORBERT WIENER
\end{document}
